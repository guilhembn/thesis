\ifdefined\included
\else
\documentclass[a4paper,11pt,twoside]{StyleThese}
\usepackage{amsmath,amssymb, amsthm}             % AMS Math
\usepackage[T1]{fontenc}
\usepackage[utf8x]{inputenc}
\usepackage{babel}
\usepackage{datetime}

\usepackage{silence}

\WarningFilter{minitoc(hints)}{W0023}
\WarningFilter{minitoc(hints)}{W0028}
\WarningFilter{minitoc(hints)}{W0030}

\usepackage{lmodern}
\usepackage{tabularx}
%\usepackage{tabular}
\usepackage{multirow}
\usepackage{xspace}

\usepackage{hhline}
\usepackage[left=1.5in,right=1.3in,top=1.1in,bottom=1.1in,includefoot,includehead,headheight=13.6pt]{geometry}
\renewcommand{\baselinestretch}{1.05}

% Table of contents for each chapter

\usepackage[nottoc, notlof, notlot]{tocbibind}
\usepackage{minitoc}
\setcounter{minitocdepth}{2}
\mtcindent=15pt
% Use \minitoc where to put a table of contents

\usepackage{aecompl}

% Glossary / list of abbreviations

\usepackage[intoc]{nomencl}
\iftoggle{ThesisInEnglish}{%
\renewcommand{\nomname}{Glossary}
}{ %
\renewcommand{\nomname}{Liste des Abréviations}
}

\usepackage{etoolbox}
\renewcommand\nomgroup[1]{%
  \item[\bfseries
  \ifstrequal{#1}{A}{Number Sets}{%
  \ifstrequal{#1}{G}{Agents Beliefs and Action Models}{%
  \ifstrequal{#1}{N}{Navigation}{%
  \ifstrequal{#1}{O}{Ontology}{%
  \ifstrequal{#1}{R}{Referring Expression Generation}{%
  \ifstrequal{#1}{Z}{Controllable and Uncontrollable Agents Task Planning}{}}}}}}%
]}

\makenomenclature



% My pdf code

\usepackage{ifpdf}

\ifpdf
  \usepackage[pdftex]{graphicx}
  \DeclareGraphicsExtensions{.jpg}
  \usepackage[pagebackref,hyperindex=true]{hyperref}
  \usepackage{tikz}
  \usetikzlibrary{arrows,shapes,calc}
\else
  \usepackage{graphicx}
  \DeclareGraphicsExtensions{.ps,.eps}
  \usepackage[dvipdfm,pagebackref,hyperindex=true]{hyperref}
\fi

\graphicspath{{.}{images/}}

%% nicer backref links. NOTE: The flag ThesisInEnglish is used to define the
% language in the back references. Read more about it in These.tex

\iftoggle{ThesisInEnglish}{%
\renewcommand*{\backref}[1]{}
\renewcommand*{\backrefalt}[4]{%
\ifcase #1 %
(Not cited.)%
\or
(Cited in page~#2.)%
\else
(Cited in pages~#2.)%
\fi}
\renewcommand*{\backrefsep}{, }
\renewcommand*{\backreftwosep}{ and~}
\renewcommand*{\backreflastsep}{ and~}
}{%
\renewcommand*{\backref}[1]{}
\renewcommand*{\backrefalt}[4]{%
\ifcase #1 %
(Non cité.)%
\or
(Cité en page~#2.)%
\else
(Cité en pages~#2.)%
\fi}
\renewcommand*{\backrefsep}{, }
\renewcommand*{\backreftwosep}{ et~}
\renewcommand*{\backreflastsep}{ et~}
}

% Links in pdf
\usepackage{color}
\definecolor{linkcol}{rgb}{0,0,0.4} 
\definecolor{citecol}{rgb}{0.5,0,0} 
\definecolor{linkcol}{rgb}{0,0,0} 
\definecolor{citecol}{rgb}{0,0,0}
% Change this to change the informations included in the pdf file

\hypersetup
{
bookmarksopen=true,
pdftitle="Planning For Both Robot and Human: Anticipating and Accompanying Human Decisions",
pdfauthor="Guilhem BUISAN", %auteur du document
pdfsubject="Thèse", %sujet du document
%pdftoolbar=false, %barre d'outils non visible
pdfmenubar=true, %barre de menu visible
pdfhighlight=/O, %effet d'un clic sur un lien hypertexte
colorlinks=true, %couleurs sur les liens hypertextes
pdfpagemode=None, %aucun mode de page
pdfpagelayout=SinglePage, %ouverture en simple page
pdffitwindow=true, %pages ouvertes entierement dans toute la fenetre
linkcolor=linkcol, %couleur des liens hypertextes internes
citecolor=citecol, %couleur des liens pour les citations
urlcolor=linkcol %couleur des liens pour les url
}

% definitions.
% -------------------

\setcounter{secnumdepth}{3}
\setcounter{tocdepth}{2}

% Some useful commands and shortcut for maths:  partial derivative and stuff

\newcommand{\pd}[2]{\frac{\partial #1}{\partial #2}}
\def\abs{\operatorname{abs}}
\def\argmax{\operatornamewithlimits{arg\,max}}
\def\argmin{\operatornamewithlimits{arg\,min}}
\def\diag{\operatorname{Diag}}
\newcommand{\eqRef}[1]{(\ref{#1})}

\usepackage{rotating}                    % Sideways of figures & tables
%\usepackage{bibunits}
%\usepackage[sectionbib]{chapterbib}          % Cross-reference package (Natural BiB)
%\usepackage{natbib}                  % Put References at the end of each chapter
                                         % Do not put 'sectionbib' option here.
                                         % Sectionbib option in 'natbib' will do.
\usepackage{fancyhdr}                    % Fancy Header and Footer

% \usepackage{txfonts}                     % Public Times New Roman text & math font
  
%%% Fancy Header %%%%%%%%%%%%%%%%%%%%%%%%%%%%%%%%%%%%%%%%%%%%%%%%%%%%%%%%%%%%%%%%%%
% Fancy Header Style Options

\pagestyle{fancy}                       % Sets fancy header and footer
\fancyfoot{}                            % Delete current footer settings

%\renewcommand{\chaptermark}[1]{         % Lower Case Chapter marker style
%  \markboth{\chaptername\ \thechapter.\ #1}}{}} %

%\renewcommand{\sectionmark}[1]{         % Lower case Section marker style
%  \markright{\thesection.\ #1}}         %

\fancyhead[LE,RO]{\bfseries\thepage}    % Page number (boldface) in left on even
% pages and right on odd pages
\fancyhead[RE]{\bfseries\nouppercase{\leftmark}}      % Chapter in the right on even pages
\fancyhead[LO]{\bfseries\nouppercase{\rightmark}}     % Section in the left on odd pages

\let\headruleORIG\headrule
\renewcommand{\headrule}{\color{black} \headruleORIG}
\renewcommand{\headrulewidth}{1.0pt}
\usepackage{colortbl}
\arrayrulecolor{black}

\fancypagestyle{plain}{
  \fancyhead{}
  \fancyfoot{}
  \renewcommand{\headrulewidth}{0pt}
}

%\usepackage{MyAlgorithm}
%\usepackage[noend]{MyAlgorithmic}
\usepackage{algorithm}
\usepackage[noend]{algpseudocode}
\usepackage{comment}
\usepackage[ED=EDSYS-Robo, Ets=INSA]{tlsflyleaf}
%%% Clear Header %%%%%%%%%%%%%%%%%%%%%%%%%%%%%%%%%%%%%%%%%%%%%%%%%%%%%%%%%%%%%%%%%%
% Clear Header Style on the Last Empty Odd pages
\makeatletter

\def\cleardoublepage{\clearpage\if@twoside \ifodd\c@page\else%
  \hbox{}%
  \thispagestyle{empty}%              % Empty header styles
  \newpage%
  \if@twocolumn\hbox{}\newpage\fi\fi\fi}

\makeatother
 
%%%%%%%%%%%%%%%%%%%%%%%%%%%%%%%%%%%%%%%%%%%%%%%%%%%%%%%%%%%%%%%%%%%%%%%%%%%%%%% 
% Prints your review date and 'Draft Version' (From Josullvn, CS, CMU)
\newcommand{\reviewtimetoday}[2]{\special{!userdict begin
    /bop-hook{gsave 20 710 translate 45 rotate 0.8 setgray
      /Times-Roman findfont 12 scalefont setfont 0 0   moveto (#1) show
      0 -12 moveto (#2) show grestore}def end}}
% You can turn on or off this option.
% \reviewtimetoday{\today}{Draft Version}
%%%%%%%%%%%%%%%%%%%%%%%%%%%%%%%%%%%%%%%%%%%%%%%%%%%%%%%%%%%%%%%%%%%%%%%%%%%%%%% 

\newenvironment{maxime}[1]
{
\vspace*{0cm}
\hfill
\begin{minipage}{0.5\textwidth}%
%\rule[0.5ex]{\textwidth}{0.1mm}\\%
\hrulefill $\:$ {\bf #1}\\
%\vspace*{-0.25cm}
\it 
}%
{%

\hrulefill
\vspace*{0.5cm}%
\end{minipage}
}

\let\minitocORIG\minitoc
\renewcommand{\minitoc}{\minitocORIG \vspace{1.5em}}

\usepackage{multirow}
%\usepackage{slashbox}

\newenvironment{bulletList}%
{ \begin{list}%
	{$\bullet$}%
	{\setlength{\labelwidth}{25pt}%
	 \setlength{\leftmargin}{30pt}%
	 \setlength{\itemsep}{\parsep}}}%
{ \end{list} }

\theoremstyle{definition}
\newtheorem{definition}{Definition}
\renewcommand{\epsilon}{\varepsilon}

% centered page environment

\newenvironment{vcenterpage}
{\newpage\vspace*{\fill}\thispagestyle{empty}\renewcommand{\headrulewidth}{0pt}}
{\vspace*{\fill}}

\usepackage{tablefootnote}

\theoremstyle{plain}
\newtheorem{constraint}{Constraint}[section]

\algnewcommand\algorithmicforeach{\textbf{for each}}
\algnewcommand\algorithmicin{\textbf{in}}
\algdef{S}[FOR]{ForEach}[2]{\algorithmicforeach\ #1\ \algorithmicin\ #2\ \algorithmicdo}

\usepackage{listings}
\lstdefinestyle{customPlan}{
  language=C,
  commentstyle=\itshape\color{green!25!black},
}
\usepackage{pdfpages}

\sloppy
\begin{document}
\fi


\chapter*{Introduction}
\addstarredchapter{Introduction} %Sinon cela n'apparait pas dans la table des matières
\markboth{INTRODUCTION}{}
Humans use machines for a long time. On one hand, what started as simple tools quickly gained in complexity and are now robots able to autonomously act on the world with little to no human supervision. On the other hand, humans are more and more dependent on these machines, both for everyday and more specific tasks.
However, both acting on the environment and having some autonomy can lead to incidents and injuries if a robotic system is not well made or if a human has not received a specific training. This is why currently in the industry we see a complete physical separation between human and robots, or some annoying light and sound systems when robots share human environments. Indeed, while the Fitts's HABA MABA \cite{fitts_human_1951} distinction is getting blurrier, some major interaction challenges have not been tackled yet.

In this thesis we propose to explore a way to bring the human and robot closer in order to make them perform tasks in shared environments in a safer and more usable manner than existing systems. To do so, we claim that robots must be able to make their decisions based not only on their own perception of the environment but also on their estimation of the beliefs of their human partner. Moreover, the robot must be able to plan taking into account that the other agent will also plan, act and react to the actions of the robot.

\section*{Human Robot Interaction}
\markright{Human Robot Interaction}
%%% AI less error posible vs. mimicry the human
%%% HRI: mimicry the human, trying cognitive architecture on robot to check models, or less error possible-> in two ways, mimicrying the human (that is the thing we know working the best currently), or trying other stuff... We try other stuff
The literature on \acrlong{hri} covers a wide range of approaches and visions. As in artificial intelligence, a distinction can be made between systems trying to act like humans and systems trying to act rationally. Moreover, some approaches also try to implement human cognitive models on robot to validate them. %\cite{act-R}.

In this manuscript, the approach chosen is to make a system acting rationally. Besides, we define the \acrfull{hri} as Goodrich and Schultz as being \textit{the field of study dedicated to understanding, designing, and evaluating robotic systems for use by or with humans}. Thus, the goal is to make a robotic system allowing to, when used by or with humans, to perform a task in the most effective, efficient and satisfactory way. To put it otherwise, we aim at making the most usable (as defined in ISO 9241-11) robotic system.

It is worth mentioning that some work focusing on the goal described above still mimicry some human behaviors. Indeed, the best working interaction example we currently have access to is humans collaborating with humans. In this thesis, even if the human behavior may be taken as an inspiration, the argument that the robot should act in a certain way because the human does so will not be used. However, we present ways of using human tasks and actions modeling to improve robot decisions and planning.



\section*{Motion and Task Planning}
\markright{Motion and Task Planning}
Both humans and robots are considered as agents. An agent is an entity able to modify its environment (and its own state) by performing actions. Now if an agent is given an objective (a specific environment state) and has an estimation of what its actions will change in this environment, it can try to figure out a succession of actions leading to that objective. This process is called planning. We can identify two main types of planning. 

First, task (or symbolic) planning models the world into facts and agents' actions as changes over these facts. In this thesis, we explore how human robot task planning can benefit from communication actions feasibility and cost estimation (Chapter~\ref{chapter:comm}). Moreover, we propose a task planning scheme aiming at emulating human planning process to generate human robot collaborative conditional plans (Chapter~\ref{chapter:doublehtn}). Then, motion (or geometrical) planning represents the world in a more refined way, accounting for geometrical models of agents and the environment. Actions are represented as trajectories, moving objects or (parts of) agents across space. In this thesis, we will especially deal with navigation, a sub part of motion planning, and study in Chapter~\ref{chapter:navigation} how planning for both the human and robot trajectories allows to elaborate more efficient trajectories when the robot is in proximity to the human.

\section*{Summary of the Thesis}
\markright{Summary of the Thesis}
We are interested in how planning for both the human and the robot can improve the interaction while doing collaborative tasks. We start in Chapter~\ref{chapter:sota} by giving context to this thesis by presenting task and navigation planning, what are the current challenges in human robot interaction and how we can model human actions and provide shared plans. Then, in Chapter~\ref{chapter:navigation} we explore an approach to robot navigation planning where both the robot and the human trajectories are computed at position control rate. This approach uses an optimization scheme allowing to define constraints representing the interaction between the trajectories. We use it to enhance the mutual manifestness of the robot and show through a user study how it improves the efficiency of the robot navigation in narrow crossing scenarios. We also present how we used this approach on other robots and on a complete robotic system deployed in the wild during the \acrfull{mummer} project. Alleviating from inherent ephemeral and implicit nature of navigation interactions, we continue exploring human and robot planning in task planning.

In Chapter~\ref{chapter:comm}, our objective is to use the observations made in navigation and to apply them in the symbolic domain, more explicitly. More precisely, we want to consider communication as actions the robot must plan in order to have the best interaction possible. Planning communications requires planning for both agents. This lead to two contributions. First, we present an efficient algorithm running over an ontology resolving the content of communications aiming at referring to an object of the environment to the hearer. This problem is called referring expression generation and, while it has been studied for over thirty years, we show that our approach is not only the fastest one to date but also the most suitable for human robot interaction scenarios. Then, we used this efficient method for resolving communications content to estimate the feasibility and cost of such communication actions during task planning. Using this approach allows to avoid plans that would have been unrecoverable during the execution and to find more efficient ones.

In Chapter~\ref{chapter:doublehtn} we propose a hierarchical task planning scheme that is not only able to update human and robot beliefs separately throughout the planning process, but also able to reason on distinct human and robot action models. While the robot action model is close to ones used in \acrlong{htn} planning, the human one is thought to be made through a task modeling approach as done in \acrlong{hci}. Actions are represented as functions over the beliefs and we specify rules on which actions may update or reason on which beliefs. We implemented this scheme in a prototype planner which we named \acrfull{hatpehda} and showed that it allows to represent and to plan for intricate human robot collaborative scenarios. 

Finally, in Chapter~\ref{chapter:integration}, we present some interesting details about how \acrshort{hatpehda} can be integrated in a robotic architecture. We end by presenting a novel human robot collaborative task: the director task. Inspired from psychology studies, it induces several challenges for human robot interaction. We introduce a full robotic architecture able to cope with the nominal cases and in which \acrshort{hatpehda} has been integrated.

\subsection*{List of Publications}
\markright{List of Publications}
\subsubsection*{Published}
\begin{itemize}
\item Buisan, G., Sarthou, G., Bit-Monnot, A., Clodic, A., \& Alami, R. (2020, August). Efficient, situated and ontology based referring expression generation for human-robot collaboration. In \textit{2020 29th IEEE International Conference on Robot and Human Interactive Communication (RO-MAN)} (pp. 349-356). IEEE.

\item Buisan, G., Sarthou, G., \& Alami, R. (2020, November). Human aware task planning using verbal communication feasibility and costs. In \textit{International Conference on Social Robotics} (pp. 554-565). Springer, Cham.

\item Buisan, G., \& Alami, R. (2021, March). A Human-Aware Task Planner Explicitly Reasoning About Human and Robot Decision, Action and Reaction. In \textit{Companion of the 2021 ACM/IEEE International Conference on Human-Robot Interaction} (pp. 544-548).
\end{itemize}

\subsubsection*{Accepted}
\begin{itemize}
\item Belhassein, K.*, Buisan, G.*, Clodic, A., \& Alami, R. Towards Methodological Principles for User Studies in Human-Robot Interaction. To be published in \textit{ACM Transactions on Human-Robot Interaction Journal}.
\end{itemize}

\subsubsection*{Submitted}

\begin{itemize}
\item Buisan, G., Compan, N., Caroux, L., Clodic, A., Carreras, O., Vrignaud, C., \& Alami, R. Evaluating the Impact of Time to Collision Constraint and Head Gaze on Usability for Robot Navigation in a Corridor. Submitted to \textit{IEEE Transactions on Human-Machine Systems Journal}.

\item Buisan, G., Favier, A., Mayima, A., \& Alami, R. HATP/EHDA: A Robot Task Planner Anticipating and Eliciting Human Decisions and Actions. Submitted to the \textit{IEEE International Conference on Robotics and Automation (ICRA) 2022}.
\end{itemize}





 

\ifdefined\included
\else
\bibliographystyle{acm}
\bibliography{These}
\end{document}
\fi