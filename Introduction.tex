\ifdefined\included
\else
\documentclass[a4paper,11pt,twoside]{StyleThese}
\usepackage{amsmath,amssymb, amsthm}             % AMS Math
\usepackage[T1]{fontenc}
\usepackage[utf8x]{inputenc}
\usepackage{babel}
\usepackage{datetime}

\usepackage{silence}

\WarningFilter{minitoc(hints)}{W0023}
\WarningFilter{minitoc(hints)}{W0028}
\WarningFilter{minitoc(hints)}{W0030}

\usepackage{lmodern}
\usepackage{tabularx}
%\usepackage{tabular}
\usepackage{multirow}
\usepackage{xspace}

\usepackage{hhline}
\usepackage[left=1.5in,right=1.3in,top=1.1in,bottom=1.1in,includefoot,includehead,headheight=13.6pt]{geometry}
\renewcommand{\baselinestretch}{1.05}

% Table of contents for each chapter

\usepackage[nottoc, notlof, notlot]{tocbibind}
\usepackage{minitoc}
\setcounter{minitocdepth}{2}
\mtcindent=15pt
% Use \minitoc where to put a table of contents

\usepackage{aecompl}

% Glossary / list of abbreviations

\usepackage[intoc]{nomencl}
\iftoggle{ThesisInEnglish}{%
\renewcommand{\nomname}{Glossary}
}{ %
\renewcommand{\nomname}{Liste des Abréviations}
}

\usepackage{etoolbox}
\renewcommand\nomgroup[1]{%
  \item[\bfseries
  \ifstrequal{#1}{A}{Number Sets}{%
  \ifstrequal{#1}{G}{Agents Beliefs and Action Models}{%
  \ifstrequal{#1}{N}{Navigation}{%
  \ifstrequal{#1}{O}{Ontology}{%
  \ifstrequal{#1}{R}{Referring Expression Generation}{%
  \ifstrequal{#1}{Z}{Controllable and Uncontrollable Agents Task Planning}{}}}}}}%
]}

\makenomenclature



% My pdf code

\usepackage{ifpdf}

\ifpdf
  \usepackage[pdftex]{graphicx}
  \DeclareGraphicsExtensions{.jpg}
  \usepackage[pagebackref,hyperindex=true]{hyperref}
  \usepackage{tikz}
  \usetikzlibrary{arrows,shapes,calc}
\else
  \usepackage{graphicx}
  \DeclareGraphicsExtensions{.ps,.eps}
  \usepackage[dvipdfm,pagebackref,hyperindex=true]{hyperref}
\fi

\graphicspath{{.}{images/}}

%% nicer backref links. NOTE: The flag ThesisInEnglish is used to define the
% language in the back references. Read more about it in These.tex

\iftoggle{ThesisInEnglish}{%
\renewcommand*{\backref}[1]{}
\renewcommand*{\backrefalt}[4]{%
\ifcase #1 %
(Not cited.)%
\or
(Cited in page~#2.)%
\else
(Cited in pages~#2.)%
\fi}
\renewcommand*{\backrefsep}{, }
\renewcommand*{\backreftwosep}{ and~}
\renewcommand*{\backreflastsep}{ and~}
}{%
\renewcommand*{\backref}[1]{}
\renewcommand*{\backrefalt}[4]{%
\ifcase #1 %
(Non cité.)%
\or
(Cité en page~#2.)%
\else
(Cité en pages~#2.)%
\fi}
\renewcommand*{\backrefsep}{, }
\renewcommand*{\backreftwosep}{ et~}
\renewcommand*{\backreflastsep}{ et~}
}

% Links in pdf
\usepackage{color}
\definecolor{linkcol}{rgb}{0,0,0.4} 
\definecolor{citecol}{rgb}{0.5,0,0} 
\definecolor{linkcol}{rgb}{0,0,0} 
\definecolor{citecol}{rgb}{0,0,0}
% Change this to change the informations included in the pdf file

\hypersetup
{
bookmarksopen=true,
pdftitle="Planning For Both Robot and Human: Anticipating and Accompanying Human Decisions",
pdfauthor="Guilhem BUISAN", %auteur du document
pdfsubject="Thèse", %sujet du document
%pdftoolbar=false, %barre d'outils non visible
pdfmenubar=true, %barre de menu visible
pdfhighlight=/O, %effet d'un clic sur un lien hypertexte
colorlinks=true, %couleurs sur les liens hypertextes
pdfpagemode=None, %aucun mode de page
pdfpagelayout=SinglePage, %ouverture en simple page
pdffitwindow=true, %pages ouvertes entierement dans toute la fenetre
linkcolor=linkcol, %couleur des liens hypertextes internes
citecolor=citecol, %couleur des liens pour les citations
urlcolor=linkcol %couleur des liens pour les url
}

% definitions.
% -------------------

\setcounter{secnumdepth}{3}
\setcounter{tocdepth}{2}

% Some useful commands and shortcut for maths:  partial derivative and stuff

\newcommand{\pd}[2]{\frac{\partial #1}{\partial #2}}
\def\abs{\operatorname{abs}}
\def\argmax{\operatornamewithlimits{arg\,max}}
\def\argmin{\operatornamewithlimits{arg\,min}}
\def\diag{\operatorname{Diag}}
\newcommand{\eqRef}[1]{(\ref{#1})}

\usepackage{rotating}                    % Sideways of figures & tables
%\usepackage{bibunits}
%\usepackage[sectionbib]{chapterbib}          % Cross-reference package (Natural BiB)
%\usepackage{natbib}                  % Put References at the end of each chapter
                                         % Do not put 'sectionbib' option here.
                                         % Sectionbib option in 'natbib' will do.
\usepackage{fancyhdr}                    % Fancy Header and Footer

% \usepackage{txfonts}                     % Public Times New Roman text & math font
  
%%% Fancy Header %%%%%%%%%%%%%%%%%%%%%%%%%%%%%%%%%%%%%%%%%%%%%%%%%%%%%%%%%%%%%%%%%%
% Fancy Header Style Options

\pagestyle{fancy}                       % Sets fancy header and footer
\fancyfoot{}                            % Delete current footer settings

%\renewcommand{\chaptermark}[1]{         % Lower Case Chapter marker style
%  \markboth{\chaptername\ \thechapter.\ #1}}{}} %

%\renewcommand{\sectionmark}[1]{         % Lower case Section marker style
%  \markright{\thesection.\ #1}}         %

\fancyhead[LE,RO]{\bfseries\thepage}    % Page number (boldface) in left on even
% pages and right on odd pages
\fancyhead[RE]{\bfseries\nouppercase{\leftmark}}      % Chapter in the right on even pages
\fancyhead[LO]{\bfseries\nouppercase{\rightmark}}     % Section in the left on odd pages

\let\headruleORIG\headrule
\renewcommand{\headrule}{\color{black} \headruleORIG}
\renewcommand{\headrulewidth}{1.0pt}
\usepackage{colortbl}
\arrayrulecolor{black}

\fancypagestyle{plain}{
  \fancyhead{}
  \fancyfoot{}
  \renewcommand{\headrulewidth}{0pt}
}

%\usepackage{MyAlgorithm}
%\usepackage[noend]{MyAlgorithmic}
\usepackage{algorithm}
\usepackage[noend]{algpseudocode}
\usepackage{comment}
\usepackage[ED=EDSYS-Robo, Ets=INSA]{tlsflyleaf}
%%% Clear Header %%%%%%%%%%%%%%%%%%%%%%%%%%%%%%%%%%%%%%%%%%%%%%%%%%%%%%%%%%%%%%%%%%
% Clear Header Style on the Last Empty Odd pages
\makeatletter

\def\cleardoublepage{\clearpage\if@twoside \ifodd\c@page\else%
  \hbox{}%
  \thispagestyle{empty}%              % Empty header styles
  \newpage%
  \if@twocolumn\hbox{}\newpage\fi\fi\fi}

\makeatother
 
%%%%%%%%%%%%%%%%%%%%%%%%%%%%%%%%%%%%%%%%%%%%%%%%%%%%%%%%%%%%%%%%%%%%%%%%%%%%%%% 
% Prints your review date and 'Draft Version' (From Josullvn, CS, CMU)
\newcommand{\reviewtimetoday}[2]{\special{!userdict begin
    /bop-hook{gsave 20 710 translate 45 rotate 0.8 setgray
      /Times-Roman findfont 12 scalefont setfont 0 0   moveto (#1) show
      0 -12 moveto (#2) show grestore}def end}}
% You can turn on or off this option.
% \reviewtimetoday{\today}{Draft Version}
%%%%%%%%%%%%%%%%%%%%%%%%%%%%%%%%%%%%%%%%%%%%%%%%%%%%%%%%%%%%%%%%%%%%%%%%%%%%%%% 

\newenvironment{maxime}[1]
{
\vspace*{0cm}
\hfill
\begin{minipage}{0.5\textwidth}%
%\rule[0.5ex]{\textwidth}{0.1mm}\\%
\hrulefill $\:$ {\bf #1}\\
%\vspace*{-0.25cm}
\it 
}%
{%

\hrulefill
\vspace*{0.5cm}%
\end{minipage}
}

\let\minitocORIG\minitoc
\renewcommand{\minitoc}{\minitocORIG \vspace{1.5em}}

\usepackage{multirow}
%\usepackage{slashbox}

\newenvironment{bulletList}%
{ \begin{list}%
	{$\bullet$}%
	{\setlength{\labelwidth}{25pt}%
	 \setlength{\leftmargin}{30pt}%
	 \setlength{\itemsep}{\parsep}}}%
{ \end{list} }

\theoremstyle{definition}
\newtheorem{definition}{Definition}
\renewcommand{\epsilon}{\varepsilon}

% centered page environment

\newenvironment{vcenterpage}
{\newpage\vspace*{\fill}\thispagestyle{empty}\renewcommand{\headrulewidth}{0pt}}
{\vspace*{\fill}}

\usepackage{tablefootnote}

\theoremstyle{plain}
\newtheorem{constraint}{Constraint}[section]

\algnewcommand\algorithmicforeach{\textbf{for each}}
\algnewcommand\algorithmicin{\textbf{in}}
\algdef{S}[FOR]{ForEach}[2]{\algorithmicforeach\ #1\ \algorithmicin\ #2\ \algorithmicdo}

\usepackage{listings}
\lstdefinestyle{customPlan}{
  language=C,
  commentstyle=\itshape\color{green!25!black},
}
\usepackage{pdfpages}

\sloppy
\begin{document}
\fi


\chapter*{Introduction}
\addstarredchapter{Introduction} %Sinon cela n'apparait pas dans la table des matières
Humans and machines are now collaborating for a long time. On one hand, what started as simple tools quickly gained in complexity and are now robots able to autonomously act on the world with little to no human supervision. On the other hand, humans are more and more dependent on these machines, both for everyday and more specific tasks.
However, both acting on the environment and having some autonomy can lead to incident and injuries if a robotic system is not well made or if a human has not received a specific formation. This is why currently in the industry we see a strong physical separation between human and robots, or some annoying light and sound systems when robots share human environments. Indeed, while the Fitts's HABA MABA \cite{fitts_human_1951} distinction is getting blurrier, some major interaction challenges have not been tackled yet.

In this thesis we propose to explore a way to bring the human and robot closer in order to make them perform tasks in shared environments in a safer and more usable manner than existing systems. To do so, we claim that robots must be able to make their decision based not only on their own perception of the environment but also on their estimation of the beliefs of their human partner. Moreover, the robot must be able to plan taking into account that the other agent will also plan, act and react to the actions of the robot.

\section*{Human Robot Interaction}
%%% AI less error posible vs. mimicry the human
%%% HRI: mimicry the human, trying cognitive architecture on robot to check models, or less error possible-> in two ways, mimicrying the human (that is the thing we know working the best currently), or trying other stuff... We try other stuff
The literature on human robot interaction regroups a wide range of approaches and visions. As in artificial intelligence, a distinction can be made between systems trying to act like humans and systems trying to act rationally. Moreover, some approaches also tries to implement human cognitive models on robot to validate them \cite{act-R}.

In this manuscript, the approach chosen is to make a system acting rationally. Besides, we define the human robot interaction as Goodrich and Schultz as being \textit{the field of study dedicated to understanding, designing, and evaluating robotic systems for use by or with humans}. Thus, the goal is to make a robotic system allowing to, when used by or with humans, to perform a task in the most effective, efficient and satisfactory way. To put it otherwise, we aim at making the most usable (as defined in ISO 9241-11) robotic system.

It worth mentioning that some work focusing on the goal described above still mimicry some human behaviors. Indeed, the best working interaction example we currently have access to is humans collaborating with human. In this thesis, even if the human behavior may be taken as an inspiration, the argument that the robot should act in a certain way because the human does so will not be used.



\section*{Motion and Task Planning}
We consider both humans and robots as agents. An agent is an entity able to modify its environment (and its own state) by performing actions. Now if an agent is given an objective (a specific environment state) and has an estimation of what its actions will change in this environment, it can try to figure out a succession of actions leading to that objective. This process is called planning.



 

\ifdefined\included
\else
\bibliographystyle{acm}
\bibliography{These}
\end{document}
\fi