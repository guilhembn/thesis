\ifdefined\included
\else
\documentclass[a4paper,11pt,twoside]{StyleThese}
\include{formatAndDefs}
\sloppy
\begin{document}
\fi


\chapter*{Introduction}
\addstarredchapter{Introduction} %Sinon cela n'apparait pas dans la table des matières
Humans use machines for a long time. On one hand, what started as simple tools quickly gained in complexity and are now robots able to autonomously act on the world with little to no human supervision. On the other hand, humans are more and more dependent on these machines, both for everyday and more specific tasks.
However, both acting on the environment and having some autonomy can lead to incident and injuries if a robotic system is not well made or if a human has not received a specific training. This is why currently in the industry we see a complete physical separation between human and robots, or some annoying light and sound systems when robots share human environments. Indeed, while the Fitts's HABA MABA \cite{fitts_human_1951} distinction is getting blurrier, some major interaction challenges have not been tackled yet.

In this thesis we propose to explore a way to bring the human and robot closer in order to make them perform tasks in shared environments in a safer and more usable manner than existing systems. To do so, we claim that robots must be able to make their decision based not only on their own perception of the environment but also on their estimation of the beliefs of their human partner. Moreover, the robot must be able to plan taking into account that the other agent will also plan, act and react to the actions of the robot.

\section*{Human Robot Interaction}
%%% AI less error posible vs. mimicry the human
%%% HRI: mimicry the human, trying cognitive architecture on robot to check models, or less error possible-> in two ways, mimicrying the human (that is the thing we know working the best currently), or trying other stuff... We try other stuff
The literature on human robot interaction covers a wide range of approaches and visions. As in artificial intelligence, a distinction can be made between systems trying to act like humans and systems trying to act rationally. Moreover, some approaches also try to implement human cognitive models on robot to validate them. %\cite{act-R}.

In this manuscript, the approach chosen is to make a system acting rationally. Besides, we define the \acrfull{hri} as Goodrich and Schultz as being \textit{the field of study dedicated to understanding, designing, and evaluating robotic systems for use by or with humans}. Thus, the goal is to make a robotic system allowing to, when used by or with humans, to perform a task in the most effective, efficient and satisfactory way. To put it otherwise, we aim at making the most usable (as defined in ISO 9241-11) robotic system.

It is worth mentioning that some work focusing on the goal described above still mimicry some human behaviors. Indeed, the best working interaction example we currently have access to is humans collaborating with human. In this thesis, even if the human behavior may be taken as an inspiration, the argument that the robot should act in a certain way because the human does so will not be used. However, we present ways of using human task and action modeling to improve robot decisions and planning.



\section*{Motion and Task Planning}
Both humans and robots are considered as agents. An agent is an entity able to modify its environment (and its own state) by performing actions. Now if an agent is given an objective (a specific environment state) and has an estimation of what its actions will change in this environment, it can try to figure out a succession of actions leading to that objective. This process is called planning. We can identify two main types of planning. 

First, task (or symbolic) planning models the world into facts and agents' actions as changes over these facts. In this thesis, we explore how human robot task planning can benefit from communication action feasibility and cost estimation (Chapter~\ref{chapter:comm}). Moreover, we propose a task planning scheme aiming at emulating human planning process to generate human robot collaborative conditional plans (Chapter~\ref{chapter:doublehtn}). Then, motion (or geometrical) planning represents the world in a more refined way, accounting for geometrical models of agents and the environment. Actions are represented as trajectories, moving objects or (parts of) agents across space. In this thesis, we will especially deal with navigation, a sub part of motion planning, and study how planning for both the human and robot trajectories allows to elaborate more efficient trajectories when the robot is in proximity to the human.

\section*{Summary of the Thesis}
We are interested in how planning for both the human and the robot can improve the interaction while doing collaborative tasks. We start in Chapter~\ref{chapter:sota} by giving context to this thesis by presenting task and navigation planning, what are the current challenges in human robot interaction and how can we model human actions and provide shared plans. Then, in Chapter~\ref{chapter:navigation} we explore an approach to robot navigation planning where both the robot and the human trajectories are computed at position control rate. This approach uses an optimization scheme allowing to define constraints representing the interaction between the trajectories. We use it to enhance the mutual manifestness of the robot and show through a user study how it improves the efficiency of the robot navigation in narrow crossing scenarios. We also present how we used this approach on other robots and on a complete robotic system deployed in the wild during the \acrfull{mummer} project. Alleviating from inherent ephemeral and implicit nature of navigation interactions, we continue exploring human and robot planning in task planning.

In Chapter~\ref{chapter:comm}, our objective is to use the observations made in navigation and to apply them in the symbolic domain, more explicitly. More precisely, we want to consider communication as actions the robot must plan in order to have the best interaction possible. Planning communications requires planning for both agents. This lead to two contributions. First, we present an efficient algorithm running over an ontology resolving the content of communications aiming at referring to an object of the environment to the hearer. This problem is called referring expression generation and, while it has been studied for over thirty years, we show that our approach is not only the fastest one to date but also the most suitable for human robot interaction scenarios. Then, we used this efficient method for resolving communication content to estimate the feasibility and cost of such communication action during task planning. Using this approach allows to avoid plans that would have been unrecoverable during the execution and to find more efficient ones.

In Chapter~\ref{chapter:doublehtn} we propose a hierarchical task planning scheme that is not only able to update human and robot beliefs separately throughout the planning process, but also reason on distinct human and robot action models. While the robot action model is close to ones used in hierarchical task network planning, the human one is thought to be made through a task modeling approach as done in human computer interaction. Actions are represented as functions over the beliefs and we specify rules on which actions may update or reason on which beliefs. We implemented this scheme in a prototype planner which we named \acrfull{hatpehda} and showed that it allows to represent and to plan for intricate human robot collaborative scenarios. 

Finally, in Chapter~\ref{chapter:integration}, we present some interesting details about how \acrshort{hatpehda} can be integrated in a robotic architecture. We end by presenting a novel human robot collaborative task: the director task. Inspired from psychology studies, it induces several challenges for human robot interaction. We introduce an full robotic architecture able to cope with the nominal cases and in which \acrshort{hatpehda} has been integrated.

\subsection*{List of Publications}
\subsubsection*{Published}
\begin{itemize}
\item Buisan, G., Sarthou, G., Bit-Monnot, A., Clodic, A., \& Alami, R. (2020, August). Efficient, situated and ontology based referring expression generation for human-robot collaboration. In \textit{2020 29th IEEE International Conference on Robot and Human Interactive Communication (RO-MAN)} (pp. 349-356). IEEE.

\item Buisan, G., Sarthou, G., \& Alami, R. (2020, November). Human aware task planning using verbal communication feasibility and costs. In \textit{International Conference on Social Robotics} (pp. 554-565). Springer, Cham.

\item Buisan, G., \& Alami, R. (2021, March). A Human-Aware Task Planner Explicitly Reasoning About Human and Robot Decision, Action and Reaction. In \textit{Companion of the 2021 ACM/IEEE International Conference on Human-Robot Interaction} (pp. 544-548).
\end{itemize}

\subsubsection*{Accepted}
\begin{itemize}
\item Belhassein, K.*, Buisan, G.*, Clodic, A., \& Alami, R. Towards Methodological Principles for User Studies in Human-Robot Interaction. To be published in \textit{ACM Transactions on Human-Robot Interaction Journal}.
\end{itemize}

\subsubsection*{Submitted}

\begin{itemize}
\item Buisan, G., Compan, N., Caroux, L., Clodic, A., Carreras, O., Vrignaud, C., \& Alami, R. Evaluating the Impact of Time to Collision Constraint and Head Gaze on Usability for Robot Navigation in a Corridor. Submitted to \textit{IEEE Transactions on Human-Machine Systems Journal}.
\end{itemize}





 

\ifdefined\included
\else
\bibliographystyle{acm}
\bibliography{These}
\end{document}
\fi