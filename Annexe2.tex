\chapter{Résumé Français : }
\label{annex:frenchversion}
Nous fournissons ici un résumé en langue française des travaux présentés dans ce manuscrit de thèse.

\section{Introduction}
Les outils utilisés par les humains, autrefois simples, ont rapidement gagnés en complexité. Ces outils aujourd'hui devenus robots sont capables d'agir de manière autonomes et nécessitent de moins en moins de supervision humaine. Dans le même temps, les humains sont devenus de plus en plus dépendants de ces machines à la fois pour la vie quotidienne et pour des tâches plus spécifiques. Cependant, dans l'industrie les robots et les humains ont très souvent leurs espaces de travail séparés, lorsque ce n'est pas le cas, les machines voient leurs capacités limités et les humains requièrent une formation poussée pour les utiliser.

Dans cette thèse, nous explorons des méthodes pour rapprocher les humains et les robots afin de leur faire effectuer des tâches collaboratives au sein d’environnements partagés. Pour ce faire, nous soutenons que les robots doivent être capables de prendre des décisions non seulement sur leurs propres connaissances et perception de l’environnement, mais aussi sur leurs estimations des croyances ((



