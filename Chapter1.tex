\ifdefined\included
\else
\documentclass[a4paper,11pt,twoside]{StyleThese}
\usepackage{amsmath,amssymb, amsthm}             % AMS Math
\usepackage[T1]{fontenc}
\usepackage[utf8x]{inputenc}
\usepackage{babel}
\usepackage{datetime}

\usepackage{silence}

\WarningFilter{minitoc(hints)}{W0023}
\WarningFilter{minitoc(hints)}{W0028}
\WarningFilter{minitoc(hints)}{W0030}

\usepackage{lmodern}
\usepackage{tabularx}
%\usepackage{tabular}
\usepackage{multirow}
\usepackage{xspace}

\usepackage{hhline}
\usepackage[left=1.5in,right=1.3in,top=1.1in,bottom=1.1in,includefoot,includehead,headheight=13.6pt]{geometry}
\renewcommand{\baselinestretch}{1.05}

% Table of contents for each chapter

\usepackage[nottoc, notlof, notlot]{tocbibind}
\usepackage{minitoc}
\setcounter{minitocdepth}{2}
\mtcindent=15pt
% Use \minitoc where to put a table of contents

\usepackage{aecompl}

% Glossary / list of abbreviations

\usepackage[intoc]{nomencl}
\iftoggle{ThesisInEnglish}{%
\renewcommand{\nomname}{Glossary}
}{ %
\renewcommand{\nomname}{Liste des Abréviations}
}

\usepackage{etoolbox}
\renewcommand\nomgroup[1]{%
  \item[\bfseries
  \ifstrequal{#1}{A}{Number Sets}{%
  \ifstrequal{#1}{G}{Agents Beliefs and Action Models}{%
  \ifstrequal{#1}{N}{Navigation}{%
  \ifstrequal{#1}{O}{Ontology}{%
  \ifstrequal{#1}{R}{Referring Expression Generation}{%
  \ifstrequal{#1}{Z}{Controllable and Uncontrollable Agents Task Planning}{}}}}}}%
]}

\makenomenclature



% My pdf code

\usepackage{ifpdf}

\ifpdf
  \usepackage[pdftex]{graphicx}
  \DeclareGraphicsExtensions{.jpg}
  \usepackage[pagebackref,hyperindex=true]{hyperref}
  \usepackage{tikz}
  \usetikzlibrary{arrows,shapes,calc}
\else
  \usepackage{graphicx}
  \DeclareGraphicsExtensions{.ps,.eps}
  \usepackage[dvipdfm,pagebackref,hyperindex=true]{hyperref}
\fi

\graphicspath{{.}{images/}}

%% nicer backref links. NOTE: The flag ThesisInEnglish is used to define the
% language in the back references. Read more about it in These.tex

\iftoggle{ThesisInEnglish}{%
\renewcommand*{\backref}[1]{}
\renewcommand*{\backrefalt}[4]{%
\ifcase #1 %
(Not cited.)%
\or
(Cited in page~#2.)%
\else
(Cited in pages~#2.)%
\fi}
\renewcommand*{\backrefsep}{, }
\renewcommand*{\backreftwosep}{ and~}
\renewcommand*{\backreflastsep}{ and~}
}{%
\renewcommand*{\backref}[1]{}
\renewcommand*{\backrefalt}[4]{%
\ifcase #1 %
(Non cité.)%
\or
(Cité en page~#2.)%
\else
(Cité en pages~#2.)%
\fi}
\renewcommand*{\backrefsep}{, }
\renewcommand*{\backreftwosep}{ et~}
\renewcommand*{\backreflastsep}{ et~}
}

% Links in pdf
\usepackage{color}
\definecolor{linkcol}{rgb}{0,0,0.4} 
\definecolor{citecol}{rgb}{0.5,0,0} 
\definecolor{linkcol}{rgb}{0,0,0} 
\definecolor{citecol}{rgb}{0,0,0}
% Change this to change the informations included in the pdf file

\hypersetup
{
bookmarksopen=true,
pdftitle="Planning For Both Robot and Human: Anticipating and Accompanying Human Decisions",
pdfauthor="Guilhem BUISAN", %auteur du document
pdfsubject="Thèse", %sujet du document
%pdftoolbar=false, %barre d'outils non visible
pdfmenubar=true, %barre de menu visible
pdfhighlight=/O, %effet d'un clic sur un lien hypertexte
colorlinks=true, %couleurs sur les liens hypertextes
pdfpagemode=None, %aucun mode de page
pdfpagelayout=SinglePage, %ouverture en simple page
pdffitwindow=true, %pages ouvertes entierement dans toute la fenetre
linkcolor=linkcol, %couleur des liens hypertextes internes
citecolor=citecol, %couleur des liens pour les citations
urlcolor=linkcol %couleur des liens pour les url
}

% definitions.
% -------------------

\setcounter{secnumdepth}{3}
\setcounter{tocdepth}{2}

% Some useful commands and shortcut for maths:  partial derivative and stuff

\newcommand{\pd}[2]{\frac{\partial #1}{\partial #2}}
\def\abs{\operatorname{abs}}
\def\argmax{\operatornamewithlimits{arg\,max}}
\def\argmin{\operatornamewithlimits{arg\,min}}
\def\diag{\operatorname{Diag}}
\newcommand{\eqRef}[1]{(\ref{#1})}

\usepackage{rotating}                    % Sideways of figures & tables
%\usepackage{bibunits}
%\usepackage[sectionbib]{chapterbib}          % Cross-reference package (Natural BiB)
%\usepackage{natbib}                  % Put References at the end of each chapter
                                         % Do not put 'sectionbib' option here.
                                         % Sectionbib option in 'natbib' will do.
\usepackage{fancyhdr}                    % Fancy Header and Footer

% \usepackage{txfonts}                     % Public Times New Roman text & math font
  
%%% Fancy Header %%%%%%%%%%%%%%%%%%%%%%%%%%%%%%%%%%%%%%%%%%%%%%%%%%%%%%%%%%%%%%%%%%
% Fancy Header Style Options

\pagestyle{fancy}                       % Sets fancy header and footer
\fancyfoot{}                            % Delete current footer settings

%\renewcommand{\chaptermark}[1]{         % Lower Case Chapter marker style
%  \markboth{\chaptername\ \thechapter.\ #1}}{}} %

%\renewcommand{\sectionmark}[1]{         % Lower case Section marker style
%  \markright{\thesection.\ #1}}         %

\fancyhead[LE,RO]{\bfseries\thepage}    % Page number (boldface) in left on even
% pages and right on odd pages
\fancyhead[RE]{\bfseries\nouppercase{\leftmark}}      % Chapter in the right on even pages
\fancyhead[LO]{\bfseries\nouppercase{\rightmark}}     % Section in the left on odd pages

\let\headruleORIG\headrule
\renewcommand{\headrule}{\color{black} \headruleORIG}
\renewcommand{\headrulewidth}{1.0pt}
\usepackage{colortbl}
\arrayrulecolor{black}

\fancypagestyle{plain}{
  \fancyhead{}
  \fancyfoot{}
  \renewcommand{\headrulewidth}{0pt}
}

%\usepackage{MyAlgorithm}
%\usepackage[noend]{MyAlgorithmic}
\usepackage{algorithm}
\usepackage[noend]{algpseudocode}
\usepackage{comment}
\usepackage[ED=EDSYS-Robo, Ets=INSA]{tlsflyleaf}
%%% Clear Header %%%%%%%%%%%%%%%%%%%%%%%%%%%%%%%%%%%%%%%%%%%%%%%%%%%%%%%%%%%%%%%%%%
% Clear Header Style on the Last Empty Odd pages
\makeatletter

\def\cleardoublepage{\clearpage\if@twoside \ifodd\c@page\else%
  \hbox{}%
  \thispagestyle{empty}%              % Empty header styles
  \newpage%
  \if@twocolumn\hbox{}\newpage\fi\fi\fi}

\makeatother
 
%%%%%%%%%%%%%%%%%%%%%%%%%%%%%%%%%%%%%%%%%%%%%%%%%%%%%%%%%%%%%%%%%%%%%%%%%%%%%%% 
% Prints your review date and 'Draft Version' (From Josullvn, CS, CMU)
\newcommand{\reviewtimetoday}[2]{\special{!userdict begin
    /bop-hook{gsave 20 710 translate 45 rotate 0.8 setgray
      /Times-Roman findfont 12 scalefont setfont 0 0   moveto (#1) show
      0 -12 moveto (#2) show grestore}def end}}
% You can turn on or off this option.
% \reviewtimetoday{\today}{Draft Version}
%%%%%%%%%%%%%%%%%%%%%%%%%%%%%%%%%%%%%%%%%%%%%%%%%%%%%%%%%%%%%%%%%%%%%%%%%%%%%%% 

\newenvironment{maxime}[1]
{
\vspace*{0cm}
\hfill
\begin{minipage}{0.5\textwidth}%
%\rule[0.5ex]{\textwidth}{0.1mm}\\%
\hrulefill $\:$ {\bf #1}\\
%\vspace*{-0.25cm}
\it 
}%
{%

\hrulefill
\vspace*{0.5cm}%
\end{minipage}
}

\let\minitocORIG\minitoc
\renewcommand{\minitoc}{\minitocORIG \vspace{1.5em}}

\usepackage{multirow}
%\usepackage{slashbox}

\newenvironment{bulletList}%
{ \begin{list}%
	{$\bullet$}%
	{\setlength{\labelwidth}{25pt}%
	 \setlength{\leftmargin}{30pt}%
	 \setlength{\itemsep}{\parsep}}}%
{ \end{list} }

\theoremstyle{definition}
\newtheorem{definition}{Definition}
\renewcommand{\epsilon}{\varepsilon}

% centered page environment

\newenvironment{vcenterpage}
{\newpage\vspace*{\fill}\thispagestyle{empty}\renewcommand{\headrulewidth}{0pt}}
{\vspace*{\fill}}

\usepackage{tablefootnote}

\theoremstyle{plain}
\newtheorem{constraint}{Constraint}[section]

\algnewcommand\algorithmicforeach{\textbf{for each}}
\algnewcommand\algorithmicin{\textbf{in}}
\algdef{S}[FOR]{ForEach}[2]{\algorithmicforeach\ #1\ \algorithmicin\ #2\ \algorithmicdo}

\usepackage{listings}
\lstdefinestyle{customPlan}{
  language=C,
  commentstyle=\itshape\color{green!25!black},
}
\usepackage{pdfpages}

\sloppy
\begin{document}
\setcounter{chapter}{0} %% Numéro du chapitre précédent ;)
\dominitoc
\faketableofcontents
\fi

\chapter{Planning for Human Robot Interaction Background}
\minitoc

% Planifier pour l'autre c'est important : SAvoir qu'il va agir/réagir (éviter le blocage dans le couloir, savoir qu'on va atteindre le but même si le robot ne peut pas faire certaines actions), lui indiquer clairement nos intentions pour qu'il puisse prendre sa décision dans les meilleures conditions (mutual manifestness, legibility, predictability, coordination smoothers, ...)
% Motion planning (+ Human Aware)
% Task Planning (+ Human Aware)
% Usability and action model and automation

This first chapter aims at setting the context for this thesis. While not providing a exhaustive state of the art, it presents challenges of planning for human robot interaction. Exhaustive related works will be reviewed in the beginning of each chapter. In what follows, we first define robot motion and task planning, then we describe the challenges arising for the specific context of human robot interaction and finally we present general approaches trying to cope with these challenges by modeling and planning for the human.

\section{Task and Navigation Planning}
As put by Ghallab, Nau and Traverso, "the purpose of planning is to synthesize an organized set of actions to carry out some activity" \cite{ghallab_nau_traverso_2016}. Planning can be \textit{domain-specific} if the planning method (and set of actions) is precisely made to solve a specific type of activity. Domain-specific planning includes navigation aiming at planning a trajectory for moving the robot base from a place to another while respecting its kinodynamic constraints and avoiding obstacles. On the other hand, \textit{domain independent} planning uses methods which can be applied to a wide varieties of problems using abstraction. Actions are then represented as functions modeling the changes it has on a generic world state to produce a new world state. In any case planning require to model the environment and the actions allowing to predict how an agent actions would impact this environment.

Besides, the robot not only needs to plan, but also to act. Acting refers to the process by which the robot decides "how to perform the chosen actions while reacting to the context in which the activity takes place". Indeed, a planning process can usually only rely on the world state estimated at the beginning of the process and the models of how it evolves (caused or not by an agent action). However, this estimation can be coarse and can contain a lot of unknown information, moreover the models used are always imperfect and may not represent exactly how the world state evolves over time. For example in navigation planning, the map on which the planning is done may be incomplete as some obstacles may not be detectable at the robot starting position. Besides the robot controllers might not be able to follow exactly the planned trajectory, and thus the robot would fall outside the plan. This is why Ghallab, Nau and Traverso argue for an "interplay" between planning and acting.

In navigation this is usually done via a global/local planner approach \cite{choset2005principles} \improvement{add cite or this enough ?}. First the global planning process finds an obstacle-free general trajectory from the start to the end point over a known map of the environment. Then, a local planner tries to find a more precise and shorter trajectory from the current estimated position of the robot to a point on the global plan while also dealing with newly detected obstacles. This local trajectory be recomputed at position control speed (around 20Hz for speeds around 2 meters per seconds) and can be as short as only a speed command sent to the controller \improvement{cite, DWA ?} but can also predict a trajectory for several second in the future. \improvement{cite... Principle of robot motion, optimal control ? TEB ?} For more abstract task planning this often translate as having a "descriptive model" for planning, where tasks are represented as high level symbols and an "operational model" for acting, where tasks can be refined into low level commands depending on current world state. \unsure{re referencer Malik et al. ? Si oui ce serait presque sur tout le paragraphe...}
% interplay between Planning and acting, in navigation and in general

% cost based planning ? Maintenant ou dans la partie HRI ?

\section{Human-Robot Interaction}
\subsection{Usability and Automation}

\subsection{Joint Action in Human-Robot Interaction}

\section{Modeling Human Actions and Shared Plans}

\ifdefined\included
\else
\bibliographystyle{acm}
\bibliography{These}
\end{document}
\fi
