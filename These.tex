
% Choose the language of your thesis passing 'french' or 'english' as
% \documentclass option.
% Note1: The 'page de garde' will always be written in French.
% Note2: You will have an error if you change the language of the document and
%        compile it without cleaning the auxiliary files. Compiling it again
%        should solve the problem.
\documentclass[english,a4paper,11pt,twoside]{StyleThese}
\newcommand{\included}{}




\include{formatAndDefs}

\usepackage{xargs}                      % Use more than one optional parameter in a new commands
\usepackage{xcolor}  % Coloured text etc.
% 
\usepackage[colorinlistoftodos,prependcaption,textsize=tiny]{todonotes}
\newcommandx{\unsure}[2][1=]{\todo[linecolor=red,backgroundcolor=red!25,bordercolor=red,#1]{#2}}
\newcommandx{\change}[2][1=]{\todo[linecolor=blue,backgroundcolor=blue!25,bordercolor=blue,#1]{#2}}
\newcommandx{\info}[2][1=]{\todo[linecolor=olive,backgroundcolor=olive!25,bordercolor=olive,#1]{#2}}
\newcommandx{\improvement}[2][1=]{\todo[linecolor=violet,backgroundcolor=violet!25,bordercolor=violet,#1]{#2}}
\newcommandx{\thiswillnotshow}[2][1=]{\todo[disable,#1]{#2}}

%%%%% Chapter 1
\newcommand{\robotmodel}{\mathcal{M}^R}
\newcommand{\humanmodel}{\mathcal{M}^H_r}
\newcommand{\robotinhumanmodel}{\mathcal{M}^R_h}


%%%%% Chapter 2
\newcommand{\realset}{\mathbb{R}}
\newcommand{\intset}{\mathbb{N}}
\newcommand{\robotband}{B_\mathcal{R}}
\newcommand{\humanband}{B_\mathcal{H}}

%%%%% Chapter 3
\newcommand{\knowledgebase}{K}
\newcommand{\abox}{\mathcal{A}}
\newcommand{\tbox}{\mathcal{T}}
\newcommand{\rbox}{\mathcal{R}}
\newcommand{\variableset}{X}
\newcommand{\labelfunc}{\mathcal{L}}
\newcommand{\costcompfunc}{\mathcal{C}}
\newcommand{\indivset}{A}
\newcommand{\classset}{T}
\newcommand{\indiv}{a}
\newcommand{\class}{t}
\newcommand{\relationset}{R}
\newcommand{\goalindiv}{a_t}
\newcommand{\usablepropset}{U}
\newcommand{\regproblem}{REG}
\newcommand{\node}{\mathfrak{n}}
\newcommand{\transition}{\mathfrak{t}}
\newcommand{\sparql}{\textsc{SparQL}}
\newcommand{\softdiff}{\,\delta\,}
\newcommand{\harddiff}{\,\Delta\,}

%%%%% Chapter 4
\newcommand{\statespace}{S}
\newcommand{\worldstate}{s}
\newcommand{\agent}{a} %%% TODO change agent or indiv
\newcommand{\agents}{Ag}
\newcommand{\agentstate}{\sigma}
\newcommand{\ctrlagents}{\widehat{Ag}}
\newcommand{\unctrlagents}{\widetilde{Ag}}
\newcommand{\agentsstates}{\sigma}
\newcommand{\ctrlagentsstates}{\widehat{\sigma}}
\newcommand{\unctrlagentsstates}{\widetilde{\sigma}}
\newcommand{\agentsstatesset}{\Sigma}
\newcommand{\actionmodel}{\Lambda}
\newcommand{\operators}{Op}
\newcommand{\abstracttasks}{A}  %% TODO : Change
\newcommand{\methods}{Me}
\newcommand{\agenda}{d}
\newcommand{\plan}{\pi}
\newcommand{\triggerset}{T}  %% TODO : Change
\newcommand{\policy}{\Pi}

%%%%%
% À mettre dans le préambule (avant \begin{document})
%%%%%
%% Titre, auteur, date, laboratoire, cotutelle
\title{Planning for both robot and human: anticipating and accompanying human decisions}
\author{Guilhem BUISAN}
\defencedate{Date de défense (08/07/2021)}
\lab{Laboratoire d'Analyse et d'Architecture des Systèmes (LAAS-CNRS)}
%\cotutelle{Institut de cotutelle}

%% Directeur(s) de thèse
\nboss{1}                                    % Nombre de directeur(s) de thèse
\makesomeone{boss}{2}{Second DIRECTEUR}{}{}  % Sera affiché en second
\makesomeone{boss}{1}{Thierry SIMEON}{}{} % Sera affiché en premier
%% Referee
\nreferee{2}
\makesomeone{referee}{1}{Olivier SIMONIN}{}{}
\makesomeone{referee}{2}{Daniele NARDI}{}{}
%% Jury
\njudge{3}
\makesomeone{judge}{1}{Premier MEMBRE}{Professeur d'Université}{Président du Jury}
\makesomeone{judge}{2}{Second MEMBRE}{Astronome Adjoint}{Membre du Jury}
\makesomeone{judge}{3}{Troisième MEMBRE}{Chargé de Recherche}{Membre du Jury}

\sloppy
\begin{document}
\makeflyleaf

\cleardoublepage

\dominitoc

\pagenumbering{roman}

 \cleardoublepage

% Here you can see an example of how to create text conditioned by the language
% variable. The \iftoggle command:
%
%   \iftoggle{ThesisInEnglish}{%
%   <your-text-in-english>
%   }{%
%   <your-text-in-french>
%   }
%
% will compile only one of the two blocks, depending on the variable you set at
% the beginning of this document. Language selection is managed this way in the
% formatAndDefs.tex file. You too can create sections of your thesis that is
% language dependend this way, although you probably won't need it. Another use
% of \iftoggle can be found at the end of this file.
\iftoggle{ThesisInEnglish}{%
\section*{Acknowledgments}
}{%
\section*{Remerciements}
}

A faire en dernier :-) 

\tableofcontents

\printnomenclature
% Use \mtcfixnomenclature below if you have a glossary (added with
% \printnomenclature above) and you're see a shift in the mini-table of
% contents at the begining of each chapter (example: no mini-toc in chapter 1;
% mini-toc of chapter 1 appearing in chapter 2; and so on).
%
% You should not use \mtcfixnomenclature if you have no glossary (that means,
% if you don't use \printnomenclature or if your glossary is empty).
\mtcfixnomenclature

\mainmatter

\ifdefined\included
\else
\documentclass[a4paper,11pt,twoside]{StyleThese}
\include{formatAndDefs}
\sloppy
\begin{document}
\fi


\chapter*{Introduction}
\addstarredchapter{Introduction} %Sinon cela n'apparait pas dans la table des matières
Humans use machines for a long time. On one hand, what started as simple tools quickly gained in complexity and are now robots able to autonomously act on the world with little to no human supervision. On the other hand, humans are more and more dependent on these machines, both for everyday and more specific tasks.
However, both acting on the environment and having some autonomy can lead to incident and injuries if a robotic system is not well made or if a human has not received a specific training. This is why currently in the industry we see a complete physical separation between human and robots, or some annoying light and sound systems when robots share human environments. Indeed, while the Fitts's HABA MABA \cite{fitts_human_1951} distinction is getting blurrier, some major interaction challenges have not been tackled yet.

In this thesis we propose to explore a way to bring the human and robot closer in order to make them perform tasks in shared environments in a safer and more usable manner than existing systems. To do so, we claim that robots must be able to make their decision based not only on their own perception of the environment but also on their estimation of the beliefs of their human partner. Moreover, the robot must be able to plan taking into account that the other agent will also plan, act and react to the actions of the robot.

\section*{Human Robot Interaction}
%%% AI less error posible vs. mimicry the human
%%% HRI: mimicry the human, trying cognitive architecture on robot to check models, or less error possible-> in two ways, mimicrying the human (that is the thing we know working the best currently), or trying other stuff... We try other stuff
The literature on human robot interaction covers a wide range of approaches and visions. As in artificial intelligence, a distinction can be made between systems trying to act like humans and systems trying to act rationally. Moreover, some approaches also try to implement human cognitive models on robot to validate them. %\cite{act-R}.

In this manuscript, the approach chosen is to make a system acting rationally. Besides, we define the \acrfull{hri} as Goodrich and Schultz as being \textit{the field of study dedicated to understanding, designing, and evaluating robotic systems for use by or with humans}. Thus, the goal is to make a robotic system allowing to, when used by or with humans, to perform a task in the most effective, efficient and satisfactory way. To put it otherwise, we aim at making the most usable (as defined in ISO 9241-11) robotic system.

It is worth mentioning that some work focusing on the goal described above still mimicry some human behaviors. Indeed, the best working interaction example we currently have access to is humans collaborating with human. In this thesis, even if the human behavior may be taken as an inspiration, the argument that the robot should act in a certain way because the human does so will not be used. However, we present ways of using human task and action modeling to improve robot decisions and planning.



\section*{Motion and Task Planning}
Both humans and robots are considered as agents. An agent is an entity able to modify its environment (and its own state) by performing actions. Now if an agent is given an objective (a specific environment state) and has an estimation of what its actions will change in this environment, it can try to figure out a succession of actions leading to that objective. This process is called planning. We can identify two main types of planning. 

First, task (or symbolic) planning models the world into facts and agents' actions as changes over these facts. In this thesis, we explore how human robot task planning can benefit from communication action feasibility and cost estimation (Chapter~\ref{chapter:comm}). Moreover, we propose a task planning scheme aiming at emulating human planning process to generate human robot collaborative conditional plans (Chapter~\ref{chapter:doublehtn}). Then, motion (or geometrical) planning represents the world in a more refined way, accounting for geometrical models of agents and the environment. Actions are represented as trajectories, moving objects or (parts of) agents across space. In this thesis, we will especially deal with navigation, a sub part of motion planning, and study how planning for both the human and robot trajectories allows to elaborate more efficient trajectories when the robot is in proximity to the human.

\section*{Summary of the Thesis}
We are interested in how planning for both the human and the robot can improve the interaction while doing collaborative tasks. We start in Chapter~\ref{chapter:sota} by giving context to this thesis by presenting task and navigation planning, what are the current challenges in human robot interaction and how can we model human actions and provide shared plans. Then, in Chapter~\ref{chapter:navigation} we explore an approach to robot navigation planning where both the robot and the human trajectories are computed at position control rate. This approach uses an optimization scheme allowing to define constraints representing the interaction between the trajectories. We use it to enhance the mutual manifestness of the robot and show through a user study how it improves the efficiency of the robot navigation in narrow crossing scenarios. We also present how we used this approach on other robots and on a complete robotic system deployed in the wild during the \acrfull{mummer} project. Alleviating from inherent ephemeral and implicit nature of navigation interactions, we continue exploring human and robot planning in task planning.

In Chapter~\ref{chapter:comm}, our objective is to use the observations made in navigation and to apply them in the symbolic domain, more explicitly. More precisely, we want to consider communication as actions the robot must plan in order to have the best interaction possible. Planning communications requires planning for both agents. This lead to two contributions. First, we present an efficient algorithm running over an ontology resolving the content of communications aiming at referring to an object of the environment to the hearer. This problem is called referring expression generation and, while it has been studied for over thirty years, we show that our approach is not only the fastest one to date but also the most suitable for human robot interaction scenarios. Then, we used this efficient method for resolving communication content to estimate the feasibility and cost of such communication action during task planning. Using this approach allows to avoid plans that would have been unrecoverable during the execution and to find more efficient ones.

In Chapter~\ref{chapter:doublehtn} we propose a hierarchical task planning scheme that is not only able to update human and robot beliefs separately throughout the planning process, but also reason on distinct human and robot action models. While the robot action model is close to ones used in hierarchical task network planning, the human one is thought to be made through a task modeling approach as done in human computer interaction. Actions are represented as functions over the beliefs and we specify rules on which actions may update or reason on which beliefs. We implemented this scheme in a prototype planner which we named \acrfull{hatpehda} and showed that it allows to represent and to plan for intricate human robot collaborative scenarios. 

Finally, in Chapter~\ref{chapter:integration}, we present some interesting details about how \acrshort{hatpehda} can be integrated in a robotic architecture. We end by presenting a novel human robot collaborative task: the director task. Inspired from psychology studies, it induces several challenges for human robot interaction. We introduce an full robotic architecture able to cope with the nominal cases and in which \acrshort{hatpehda} has been integrated.

\subsection*{List of Publications}
\subsubsection*{Published}
\begin{itemize}
\item Buisan, G., Sarthou, G., Bit-Monnot, A., Clodic, A., \& Alami, R. (2020, August). Efficient, situated and ontology based referring expression generation for human-robot collaboration. In \textit{2020 29th IEEE International Conference on Robot and Human Interactive Communication (RO-MAN)} (pp. 349-356). IEEE.

\item Buisan, G., Sarthou, G., \& Alami, R. (2020, November). Human aware task planning using verbal communication feasibility and costs. In \textit{International Conference on Social Robotics} (pp. 554-565). Springer, Cham.

\item Buisan, G., \& Alami, R. (2021, March). A Human-Aware Task Planner Explicitly Reasoning About Human and Robot Decision, Action and Reaction. In \textit{Companion of the 2021 ACM/IEEE International Conference on Human-Robot Interaction} (pp. 544-548).
\end{itemize}

\subsubsection*{Accepted}
\begin{itemize}
\item Belhassein, K.*, Buisan, G.*, Clodic, A., \& Alami, R. Towards Methodological Principles for User Studies in Human-Robot Interaction. To be published in \textit{ACM Transactions on Human-Robot Interaction Journal}.
\end{itemize}

\subsubsection*{Submitted}

\begin{itemize}
\item Buisan, G., Compan, N., Caroux, L., Clodic, A., Carreras, O., Vrignaud, C., \& Alami, R. Evaluating the Impact of Time to Collision Constraint and Head Gaze on Usability for Robot Navigation in a Corridor. Submitted to \textit{IEEE Transactions on Human-Machine Systems Journal}.
\end{itemize}





 

\ifdefined\included
\else
\bibliographystyle{acm}
\bibliography{These}
\end{document}
\fi
\ifdefined\included
\else
\documentclass[a4paper,11pt,twoside]{StyleThese}
\include{formatAndDefs}
\sloppy
\begin{document}
\setcounter{chapter}{0} %% Numéro du chapitre précédent ;)
\dominitoc
\faketableofcontents
\fi

\chapter{Planning for Human Robot Interaction Background}
\minitoc

% Planifier pour l'autre c'est important : SAvoir qu'il va agir/réagir (éviter le blocage dans le couloir, savoir qu'on va atteindre le but même si le robot ne peut pas faire certaines actions), lui indiquer clairement nos intentions pour qu'il puisse prendre sa décision dans les meilleures conditions (mutual manifestness, legibility, predictability, coordination smoothers, ...)
% Motion planning (+ Human Aware)
% Task Planning (+ Human Aware)
% Usability and action model and automation

This first chapter aims at setting the context for this thesis. While not providing a exhaustive state of the art, it presents challenges of planning for human robot interaction. Exhaustive related works will be reviewed in the beginning of each chapter. In what follows, we first define robot motion and task planning, then we describe the challenges arising for the specific context of human robot interaction and finally we present general approaches trying to cope with these challenges by modeling and planning for the human.

\section{Task and Navigation Planning}
As put by Ghallab, Nau and Traverso, "the purpose of planning is to synthesize an organized set of actions to carry out some activity" \cite{ghallab_nau_traverso_2016}. Planning can be \textit{domain-specific} if the planning method (and set of actions) is precisely made to solve a specific type of activity. Domain-specific planning includes navigation aiming at planning a trajectory for moving the robot base from a place to another while respecting its kinodynamic constraints and avoiding obstacles. On the other hand, \textit{domain independent} planning uses methods which can be applied to a wide varieties of problems using abstraction. Actions are then represented as functions modeling the changes it has on a generic world state to produce a new world state. In any case planning require to model the environment and the actions allowing to predict how an agent actions would impact this environment.

Besides, the robot not only needs to plan, but also to act. Acting refers to the process by which the robot decides "how to perform the chosen actions while reacting to the context in which the activity takes place". Indeed, a planning process can usually only rely on the world state estimated at the beginning of the process and the models of how it evolves (caused or not by an agent action). However, this estimation can be coarse and can contain a lot of unknown information, moreover the models used are always imperfect and may not represent exactly how the world state evolves over time. For example in navigation planning, the map on which the planning is done may be incomplete as some obstacles may not be detectable at the robot starting position. Besides the robot controllers might not be able to follow exactly the planned trajectory, and thus the robot would fall outside the plan. This is why Ghallab, Nau and Traverso argue for an "interplay" between planning and acting.

In navigation this is usually done via a global/local planner approach \cite{choset2005principles} \improvement{add cite or this enough ?}. First the global planning process finds an obstacle-free general trajectory from the start to the end point over a known map of the environment. Then, a local planner tries to find a more precise and shorter trajectory from the current estimated position of the robot to a point on the global plan while also dealing with newly detected obstacles. This local trajectory be recomputed at position control speed (around 20Hz for speeds around 2 meters per seconds) and can be as short as only a speed command sent to the controller \improvement{cite, DWA ?} but can also predict a trajectory for several second in the future. \improvement{cite... Principle of robot motion, optimal control ? TEB ?} For more abstract task planning this often translate as having a "descriptive model" for planning, where tasks are represented as high level symbols and an "operational model" for acting, where tasks can be refined into low level commands depending on current world state. \unsure{re referencer Malik et al. ? Si oui ce serait presque sur tout le paragraphe...}
% interplay between Planning and acting, in navigation and in general

% cost based planning ? Maintenant ou dans la partie HRI ?

\section{Human-Robot Interaction}
\subsection{Usability and Automation}

\subsection{Joint Action in Human-Robot Interaction}

\section{Modeling Human Actions and Shared Plans}

\ifdefined\included
\else
\bibliographystyle{acm}
\bibliography{These}
\end{document}
\fi

\ifdefined\included
\else
\documentclass[a4paper,11pt,twoside]{StyleThese}
\include{formatAndDefs}
\sloppy
\begin{document}
\setcounter{chapter}{1} %% Numéro du chapitre précédent ;)
\dominitoc
\faketableofcontents
\fi

\chapter{Coplanning for navigation}
\minitoc

\section{Introduction}
In a lot of human robot interaction scenarios, the robot has to move in the environment to accomplish its task. It can either be that the task cannot be done in the direct vicinity of the robot or that the task itself is to move elsewhere. For example in the MuMMER project, a Pepper robot in a mall has to give direction instructions to guide a human to their wanted location. The robot is also able to point to visible landmarks to locate the beginning of the route (\textit{e.g.} saying \textit{"Take these stairs, then take the corridor on your right and the shop will be on your left" while pointing to the stairs}). However, some obstacles in the proximity of the robot and the guided human can prevent them to see the pointed landmarks, or a corridor crossing can be hidden, making the route description one step longer than it should be. Thus, to perform the task of route guiding more efficiently, the robot might decide to move.
In the Spencer project, another robot has to guide people to their gate in the Schipol airport. Here, the robot will navigate all the way from the starting point to the final destination while ensuring the human is actually following it, but also has to avoid other pedestrians. In this example, the navigation of the robot is a main part of the task.
In both example, the robot has to make plan its motion such as the physical and psychological safety of surrounding humans are ensured. However, not taking into account the motion of these humans during the planning process may lead to suboptimal trajectories or even deadlock.
We propose in this chapter to, after a survey of related works, present a navigation planner algorithm taking into account both the robot and the human, then to show how this approach can be used to enhance mutual manifestness and improve efficiency in a narrow corridor crossing scenario through a user study, and finally report some extension made to the approach to include humanoid robots, flying drone and to estimate the progression of the navigation task. 

\section{Related Work}
\subsection{Human-Aware Robot Navigation}
The aim of robot navigation is to make the robot base move from one place to another while avoiding static and moving obstacles. However, when the robot has to move in an environment where humans are evolving other constraints must be added. The robot must not only avoid the humans, as any other moving obstacle, to ensure their physical safety (not harming them), but also take into account their psychological safety (not stressing or frightening them) \cite{sisbot_human_2007}, \cite{kruse_human-aware_2013}. In order to respect these constraints several methods have been used. The first largely used is based on costmap exploration. Based on the robot known humans and obstacles in the environment a grid is built, where each cell has a cost representing places the robot should avoid to pass through. Then, given a start and an end points, a planner can explore this grid and try to minimize the cost along the trajectory (\cite{sisbot_human_2007}, \cite{lu_towards_2013}). These approaches are usually pretty efficient but since a whole grid can take time to compute, they can perform poorly in dynamic environments.

Another approach is to use the social force model \cite{helbing_social_1995}. A robot trajectory is computed based on repulsive or attractive force fields set on humans, obstacles and goal \cite{ferrer_robot_2013}. This gives good results in open environments but the trajectories can be erratic in confined environment with a lot of obstacles and humans because of the diverging "forces" applied. Moreover, by only considering the robot plan, these planners return no solution if the robot and the human must cross each other in a narrow corridor where the human is centered leaving no place for the robot to go. This is why we need a planner able to \textit{infer} that the human can move to one side of the corridor allowing the robot to cross on the other side.

In their work, Kuderer et al. use social force model to both compute the robot trajectories and predict the nearby human ones \cite{kuderer_feature-based_2012}. However, the resulting human trajectories are more reactions to robot motion than coplanning solutions.

To overcome this limitation, Khambhaita \& Alami proposed a navigation planner based on an optimization scheme. In this approach the trajectories of the robot and of the nearby humans are optimized together at real time to create, at position control rate, a conavigation solution \cite{khambhaita_viewing_2017}. This ensures that at all time it exists for the humans a solution to go to their known goal, and that this solution is optimal regarding a different set of constraint based on human models.

Although, even if the robot computes an optimal solution for the human and itself, it is pointless if it cannot communicate or show this solution to the human (\textit{e.g.} either it plans to go to the left or the right of the corridor, so the human can either accept or decline this plan). Thus, the robot must also try to make its intention clear \cite{pacchierotti_evaluation_2006}. This ability of a robot to exhibit its future actions is called legibility. A legible robot will have its future actions and goals inferred quicker \cite{dragan_legibility_2013}, which is crucial in entangled tasks such as crossing in a narrow corridor. For navigation, legibility can be increased either by changing the robot speed along the trajectory \cite{kruse_legible_2012} or by modifying its trajectory \cite{khambhaita_viewing_2017}.

In a broader sense, the changes in an agent's own behavior in order to make easier the interaction with another agent are called coordination smoothers \cite{vesper_minimal_2010}. We claim that a robot should exhibit some coordination smoothers when interacting with a human to increase its usability. Moreover, all the coordination smoothers are not equal, as some can bring more information thant other. A simple blinking light and beeping sound when the robot is moving are conveying less information than turn signals for example. In our case, since we deal with anthropomorphic robots, we can try to make even more efficient coordination smoothers by using what can be identified as the head of the robot.

\subsection{Communicating Intents Via the Robot Gaze} \unsure{Cette subsection n'a peut-être rien à faire ici...}
Some robots have a movable part that can be identified as an head, and often contains camera or similar devices that can be recognized eyes. the resulting robot \textit{gaze} has already been used to effectively increase the user attention and engagement \cite{mutlu_storytelling_2006}, \cite{zaraki_designing_2014}. Moreover, the robot gaze has also been shown useful in navigation to indicate turning intentions \cite{lu_towards_2013}, \cite{may_show_2015}, and thus increase legibility.

Besides, in intricate collaborative activities, each agent must show to the other one that they are aware of their presence and actions. Pacherie defines it as the mutual manifestness: \textit{each subject must be aware, in some sense, of the event as an event that is present to both; in other words the fact that both are attending to the same object or event should be open or mutually manifest} \cite{pacherie_phenomenology_2011}. Thus, it is interesting to know if in intricate human robot navigation tasks, making the robot show mutual manifestness increase the efficiency of the task.



\section{The Human Aware Timed Elastic Band}
The only work to our knowledge being able to, in real time, plan trajectories for the robot and the humans surrounding it, is the \textit{Human Aware Timed Elastic Band} \cite{khambhaita_viewing_2017}. Thus, we used it as the backbone of our work, and made several contributions around it.

\subsection{General scheme}
The human aware timed elastic band algorithm is based on the timed elastic band (TEB) approach from Rosmann et al. \cite{rosmann_efficient_2013}. This approach is a local optimization problem where the successive positions $(x_i, y_i) \in \realset$ and orientations $\theta_i \in S^1$ of the robot along with the time steps $\Delta T_i \in \realset$ between each consecutive poses are optimized to minimize a multi criteria cost function up to a fixed length horizon $n \in \intset$. To put it otherwise, the elastic band trajectory of the robot is represented by its poses: 
\[Q = \{\textbf{s}_i\}_{i=0..n} with \textbf{s}_i = [x_i, y_i, \theta_i]^T\] to which are added the time intervals between two consecutive poses: \[\tau = {\Delta T_i}_{i=0..n-1}\] Resulting in the \textit{time elastic band} \[B := (Q, \tau)\] having to be optimized to minimize the cost function $f(B)$ to get the optimal trajectory \[B^* = \mathop{\mathrm{argmin}}_B\,f(B)\]
This function takes the form of a multi criteria weighted sum cost function which can be rewritten as: \[ f(B) = \sum_{k} \gamma_k f_k(B) \] where $\gamma_k \in \realset$ are weights allowing the designer to balance the importance between cost functions $f_k$.

This planner has been integrated has a local planner in the ROS architecture. Provided with a global plan (often generated with an A* algorithm) of the long trajectory, the local planner generates short term plans (up to several meters), avoiding static and dynamic obstacles (both known by the global planner and discovered with the robot sensor during the navigation) and minimizing the trajectory duration. In addition, the local planner is responsible for generating the speed command at position control rate (around 10 Hz usually). TEB does it by optimizing the local trajectory and computing the wanted robot speed from the first two poses and the time interval between them. Moreover, if the optimization process takes too long, the length horizon of the global trajectory on which the local optimization is performed is reduced, and increased if the optimization time is satisfactory.

In the human aware timed elastic band approach, multiple timed band are considered. In addition to the robot band $\robotband$ represents the robot trajectory, it also considers multiple human bands $B_{\mathcal{H}_k}$ with $k in \intset$ the number of humans in vicinity of the robot. For simplicity purpose, in this thesis we will only consider one human in the vicinity of the robot, and thus one human band $\humanband$. However, the approach has been shown working successfully up to three humans.
Moreover, the weighted-sum cost function becomes:
\begin{equation} \label{eq:hateb_obj_function}
f(\robotband, \humanband) = \sum_a \gamma_a f_a(\robotband) + \sum_b \gamma_b f_b(\humanband) + \sum_c \gamma_c f_c(\robotband, \humanband)
\end{equation}   

where $fa$, $fb$ and $fc$ represent respectively cost functions associated with robot trajectory constraints, human trajectory constraints and human-robot social constraints. Then, the optimization process consist in finding the optimal robot and human trajectories $\robotband, \humanband$ such as:
\[\{\robotband^*, \humanband^*\} = \mathop{\mathrm{argmin}}_{\{\robotband, \humanband\}}\,f(\robotband, \humanband)\]





\subsection{Constraints}
In this optimization scheme, all the constraints are represented as cost in the function. Thus, there is no \textit{hard constraints}, but using the weight of each one, we are able to prioritize some over the others. Moreover, when a trajectory has been optimized, before being executed, the local planner checks that it respects all the defined hard constraints (kinodynamic constraints and obstacles separation).

The new formulation of Khambhaita et al. allows to separate the constaints into three categories:
\begin{itemize}
\item Robot trajectory constraints: these constraints represent the robot kinodynamic constraints (non holonomic, maximum speed, maximum acceleration) as well as preventing the robot trajectory to differ too much from the global plan. They are presented in \cite{rosmann_efficient_2013}.
\item Human trajectory constraints: these constraints represent the human kinodynamic constraints and prevent them to differ too much from the global trajectory. They are the same as the robot ones, but their parameters (\textit{e.g.} maximum speed threshold) must not be set by the designer but evaluated by the robot during the evaluation.
\item Human-robot social constraints: these constraints represent how the human and robot trajectory must interact with each other. Khambhaita et al. presneted the \textit{safety} constraint, ensuring a sufficient distance between the robot and the human; the \textit{directional constraint} discouraging trajectories where the robot and the human move straight to each other; and the \textit{time-to-collision} (TTC) constraint, preventing the robot and the human to adopt speed which, if maintained, would lead to a collision. The latter will be detailed in what follows as it was studied more in depth through a user study.
\end{itemize}

It worth noting that different weights can be set for each constraint, and that they can be adjusted dynamically during the navigation. Moreover, by setting different weight between the robot and the human for the constraint preventing to move away from the global plan, we can adjust the \textit{stiffness} of the trajectories, thus allowing one agent or the other to elongate their trajectory, taking more or less effort into the collaborative navigation.

\section{Evaluating enhanced mutual manifestness in a crossing scenario}
In this section we present an user study aiming at assessing the pertinence of using a conavigation planner in a situation where a human and a robot must cross each other in a narrow corridor. This task of crossing in narrow corridor is challenging as both agents start in the center of either end of the corridor, and there is no way for one agent to find a way if the other agent does not move to the other side. Thus, we state that not only coplanning is required to find a plan reaching the other end of the corridor (by planning that the other agent will also cooperate and move on one side), but showing intentions and awareness of the other agent is crucial for the interaction to unfold without trouble.

\subsection{Robot behavior design}
For this user study we were particularly interested in finding if and how navigation coplanning would lead to higher mutual manifestness and to higher efficiency in crossing.
To do so, we designed a robot behavior using the HATEB navigation planner.
In their work Kambhaita et al. showed that during a narrow crossing the robot is able to plan that the human and the robot will choose opposite sides of the corridor. But if the robot shows its plan when it faces the human, they would have little time to react, and might also move to the same side as the robot, needing negotiation and replanning, reducing the overall efficiency of the crossing. The robot must thus, show the plan (\textit{i.e.} the plan trajectory, or here, the side of the corridor it plans to take) early in the crossing.
By reducing the TTC constraint function threshold and increasing its weight, we discourage trajectories where the robot and the human are facing each other. Thus, if the robot trajectory stiffness is lower than the human one, the robot will move to the chosen side of the corridor early in the trajectory as shown in Fig.~\ref{ttc_explanations}.


\begin{figure}[hbtp]
\centering
\includegraphics[scale=0.4]{figures/chapter2/condition_1_proactivity_shrink.png}
\caption{Influence of the modification of the TTC constraint cost weight on the trajectory. On the left, the weight is low, the robot will show the side and avoid the human at the last moment. On the right, the weight is high, the robot will show the chosen side and avoid the human much earlier.}
\label{ttc_explanations}
\end{figure}

Moreover, as stated before \unsure{maybe move the gaze stuff from related work to here?} several papers show that using the \textit{head} of a robot can significantly improve legibility and mutual manifestness. Thus, we also chose to make the robot look at its future planned trajectory as shown in Fig.~\ref{head_gaze_behavior}. This is possible thanks to the HATEB algorithm which, unlike many other local planner only publishing  speed commands \improvement{Adds ref to DWA for example ?}, also publishes a precise short-term trajectory. Finally, to show the robot awareness of the human presence, we made it \textit{glance} at the human twice when they enter a large and a small radius circle.

\begin{figure}[hbtp]
\centering
\includegraphics[scale=0.4]{figures/chapter2/expe_human-min.png}
\caption{Behavior implemented for the robot head. The robot \textit{looks} at a point placed at its planned position X seconds in the future and h meters above the ground.}
\label{experiment_adream}
\end{figure}

\subsection{User study protocol}

\begin{figure}[hbtp]
\centering
\includegraphics[scale=0.4]{figures/chapter2/expe_human-min.png}
\caption{The  study  environment.  The  participant  had  to  go  from  the  yellow  cross  marked  on  the  ground  to  the  green  square  also  marked  on  the  ground,which was the robot starting point. Crossing occurred roughly in the area where the robot and participant are on the picture. Trajectories are displayed on the picture for example only and where not marked on the ground or suggested by the experimenters at any time.}
\label{head_gaze_behavior}
\end{figure}

\subsubsection{Objective}
The aim of this study is to evaluate the impact of the TTC cost constraint and head behavior on usability. We designed an user study where actual individuals have to walk through a corridor facing a fully autonomous navigating robot. The afore-explained robot behavior was used. We measured the quality of the interaction between the robot and the human with both objective (visual behavior) and subjective data. The subjective evaluation was based on three dimensions: (1) perceived efficiency of the robot navigation, (2) user satisfaction and (3) situation awareness.

\subsubsection{Participants}
We recruited a total of 28 participants (12 males and 18 females) aging from 21 to 41 (mean: 27.32, SD: 4.13). All 28 participants had never used or interacted with a PR2 for navigation tasks, and had a neutral or good vision of robotics (mean: 5.96 over a 7 points Likert scale, SD: 1.07). This research complied with the tenets of Declaration of Helsinki. Informed consent was obtained from each participant.

\subsubsection{Material}
A Willow Garage PR2 robot, at its lower spine position was used in this experiment. The robot measured 1.33 meters from ground to top. The entire robot can be considered as anthropomorphic and possesses a two degrees of freedom head integrating cameras resembling eyes.

The participant position was tracked using an Optitrack motion capture system, tracking a worn solid headband. This system allowed the robot to track the human anywhere in the room, without looking at them.

The experiment was conducted in a L-shaped corridor (Fig.~\ref{experiment_adream}). The participant and the robot started from opposite side of the corridor. The participant had to walk 6 meters before entering the long straight corridor part and seeing the robot, then walk 13 meters.

We used a \textit{ETG 2w} eyetracker from SMI to collect the eye movement data of the participant. It is a portable device, allowing, after a short calibration process, to track the user gaze, and measuring where the user looks at. The data were analyzed using the \textit{BeGaze 3.6} software from SMI.

Three questionnaires and an interview were used to collect the subjective measures. \improvement{Add questionnaires in appendix}
\begin{itemize}
\item Pertinence of robot decision: The PeRDITA questionnaire \cite{devin_evaluating_2018}, jointly developed between the LAAS-CNRS and the CLLE in Toulouse, France, aims at evaluating the participant perceived pertinence of robot decision during a human robot collaborative task. In its complete form, it measures 5 dimensions: interaction, competence perception, verbal, acting and collaboration. However, in this study the robot is mute, and as the dimensions are independent we chose to remove the verbal dimension.
\item Situation Awareness: Several techniques exists to measure the situation awareness during a task \cite{endsley_design_1988}. However, they require to freeze and hide the situation to the user, and probe their working memory by questioning them about its near future. In our setup, we can't stop the robot and make it disappear while it is navigating. Thus, we have developed a series of 6 questions for measuring the user situation awareness. These questions are presented to the user just after the navigation, and ask them to rank on a 6 points Likert scale each 3 stages (2 questions per stage) of the Endsley's model: perception, comprehension and projection.
\item User satisfaction: For measuring the user satisfaction we used the AttrakDiff questionnaire. It is a standardized UX (User Experience) questionnaire measuring both hedonic qualities and global attractiveness. We used the french translation of this questionnaire \cite{lallemand_creation_2015}.
\item Interview: The interview was constituted of 8 semi directed questions always asked in the same order. These questions aimed at qualitatively evaluate the user experience, behavior and perception of the user during the navigation. The interviewer was only allowed to read the questions and to make the participant elaborate by asking neutral questions like \"why?\" or \"can you tell me more?\".
\end{itemize}


\subsubsection{Experimental design}
The user study was a 2 $\times$ 2 within-participants user study to evaluate how the time-to-collision constraint and the head behavior impact the robot navigation effectiveness efficiency and satisfaction. The independent variables were the HATEB time-to-collision cost parameters (both weight and threshold) and the head behavior. The conditions for the time-to-collision variable were $\gamma_ttc = 0.01$ (in Eq.~\ref{eq:hateb_obj_function}) with $\tau = 1s$ for the \textit{low TTC} condition and $\gamme_ttc = 15$ (in Eq.~\ref{eq:hateb_obj_function}) with $\tau = 4s$ for the \textit{high TTC} condition. For the both \textit{continuous} and \textit{alternated} head behavior conditions the robot head was pointing towards a the robot planned position in 1.5s in the future at 1m above the ground. In addition, in the \textit{alternated} head behavior, the robot pointed its head towards the human when they entered the long part of the corridor during 1.5s and again during 1.2s when the robot and human were 3.5m apart.
The participant goal position was marked with a square on the ground, and was the starting point of the robot. The robot final position was 10m straight ahead of its starting position. So, the participant was able to reach their natural walking speed before turning at the corner of the L shaped corridor. The robot was only started when the participant was about to turn (2m before the turn), giving the impression that the robot was coming from further away while ensuring that the crossing happened around the same place independently of the participant walking speed.

\subsubsection{Study procedure}
The evaluation was cut into 4 blocks. A block consisted in two same condition crossing followed by questionnaires filling. A crossing was composed by the placement of the participant and the robot on their respective starting positions, then the participant was free to go to their previously indicated goal location while crossing the robot. The three questionnaires were filled next to the participant starting location and were concerning only the two crossings made in the current block. The 4 conditions order were randomized between participants and the condition change was made between two block but never between the two crossings inside the same block.

Before starting the experiment trials, a training trial was made with the robot starting shifted to one side of the corridor and going in straight line with its head fixed looking straight. Just after this training trial, the participant was brought close to the robot and invited to inspect it. A specific head behavior was triggered making the robot head to follow the human allowing the participant to notice without being told that the robot was able to know their position and that its head could move. Moreover, the experimenter showed that robot arms were locked in place in a tucked position, and that they kept the emergency stop remote and was able to stop the robot at any time.

After the 4 blocks have been passed by the participant, the experimenter interviewed them. The audio was recorded and the answer written down.

The whole study lasted around 45 minutes per participant.

\subsubsection{Measures}
The analysis of the data was made on 27 participants because one did not fill all the questionnaires and their data were thus removed from the study. The quantitative data (questionnaires and oculometry) were analyzed using a non parametric two-way repeated measures Friedman ANOVA test.
\subsubsection{Questionnaires}
The three questionnaires have been passed 5 times each (one trial + four blocks). The results were codified from 1 to 7 for the PeRDITA, from 0 to 6 for the AttrakDiff and from 1 to 6 for the situation awareness questionnaire while taking care of reordering inverted items.

The PeRDITA Cronbach's alphas were for each dimension: $\alpha = 0.89$ for the interaction, $\alpha = 0.87$ for the competence, $\alpha = 0.85$ for the acting and $\alpha = 0.86$ for the collaboration.

For the situation assessment questionnaire, the Cronbach's alphas were: $\alpha = 0.93$ for the perception, $\alpha = 0.88$ for the comprehension and $\alpha = 0.87$ for the projection.

\subsubsection{Oculometry}



\subsection{Results}

\subsection{Discussion}

\section{Extending HATEB}
\subsection{Adapting HATEB to other robots}

\subsection{Using the estimated time to goal to measure the execution of the planned trajectory}

\section{Conclusion}

\ifdefined\included
\else
\bibliographystyle{acm}
\bibliography{These}
\end{document}
\fi

\ifdefined\included
\else
\documentclass[a4paper,11pt,twoside]{StyleThese}
\include{formatAndDefs}
\sloppy
\begin{document}
\setcounter{chapter}{2} %% Numéro du chapitre précédent ;)
\dominitoc
\faketableofcontents
\fi

\chapter{Evaluating communication needs at task planning level}
\minitoc
\printnomenclature

\section{Introduction and Example}
In the previous chapter, we showed interactions are more efficient and satisfactory if the robot consider the plan of the human in its own path of action. Not only it allows at least to ensure that the task is feasible for both agents (provided the models are correct enough) but also to perform coordination smoothers or other communication actions.

In this chapter, we alleviate from the inherent complexity of geometrical navigation planning to further study at symbolic level the planning of communication actions in plans involving multiple agents. We will especially focus on one type of explicit communication, being verbally designating an object, a problem called referring expression generation.

To entrench the problem and illustrate this chapter contents, let us take the situation depicted in Figure~\ref{fig:chap3illustrate} \improvement{Add figure}.  

First, we review the literature concerning referring expression generation and communication actions in task planning. Then we present a novel approach for referring expression generation, which runs on ontologies and is both efficient and suitable for human robot interaction scenarios. We then show how such a communication planner can be included in task planning allowing for precise estimation of communication feasibility and cost at task planning level. Finally, an extension of the referring expression generation algorithm is presented using past actions and tasks to refer to objects.

\section{Related Work}
We claim that estimating the content of some communication at task planning level is needed to generate useful plans. Indeed, some communication actions are known to be necessary already while elaborating a plan, but might not be feasible.
In this section we will firstly review how a robot can autonomously verbally designate an object to an hearer. This problem is called the \textbf{referring expression generation} (REG) problem.
Then, we will review several task planning approaches allowing to account for communication actions.

\subsection{Referring Expression Generation}
As defined by Reiter and Dale, Referring Expression Generation (REG) "\textit{is concerned with how we produce a description of an entity that enables the hearer to identify that entity in a given context} \cite{reiter1997building}. An intuition about what a well constructed referring expression (RE) is, is given by the Grice's maxims \cite{grice1975logic}. These maxims aim at defining principles for smooth cooperative activities (including verbal communication). They fall into four categories:
\begin{itemize}
\item \textit{Quantity}: The communication should be as informative as required but not more.
\item \textit{Quality}: The communication should be as true as possible. The sender should not communicate information that they consider false or unsure.
\item \textit{Relation}: The communication should be relevant in the current context. This is especially important when performing a collaborative task, where the world state is constantly changing and the relevance of a communication can quickly change.
\item \textit{Manner}: The communication should be unambiguous and brief.
\end{itemize}

The REG problem is actually composed of two parts: the content determination --- aiming at deciding which attributes (and relations) to use --- and the linguistic realization --- refining the attributes of the content into verbalizable/writable words \cite{krahmer2012computational}. In this thesis, we will only consider the content determination, as we assume that the linguistic realization will not have any impact on the plan once the content of the RE has been decided.

To our knowledge the first REG formulation and algorithm was coined by Dale and used a depth-first search over a knowledge base being a key-value tree representing attribute of objects \cite{dale1989cooking}. However, this approach lead to over specified referring expressions, containing redundant information and thus violating the maxim of quantity. This defect was corrected in a subsequent work with the \textit{Full Brevity} algorithm \cite{dale1992generating}, always generating the shortest referring expression, but at the cost of an exhaustive search. Besides, to be as relevant as possible, the attribute of the referred object to be included in the RE should be chosen carefully. Indeed, not all the attributes are equally understandable by the hearer, the color or the shape for example will often be quicker to understand than spatial relation. The Incremental Algorithm is the first approach tackling this issue \cite{dale1995computational}. By taking as input a preference list of ordered attributes, it is able to generate the smallest RE while prioritizing the attribute used.

However, all the presented approaches are running on dedicated key-value knowledge bases representing only the attribute of the entities and are thus unable to use relations between them to generate REs. For example, an object having the same attributes (color, size, shape, ...) as another one will not have any RE generated by the previous approaches, even if one is in a blue box and the other in a green one. By introducing a new knowledge representation, being a labeled directed multi-graph linking entities and attributes, Krahmer \textit{et al.} were able to solve this issue. The graph is dedicated to the problem of REG and is called a \textit{REG graph}. Moreover, a cost can be set on each edge of the graph to represent the complexity of the hearer to understand this relation. By exploring this graph through a branch and bound approach, the Graph-Based Algorithm \cite{krahmer2003graph} is able to generate the smallest and less costly RE for a given entity. This algorithm has then been refined to integrate types of entities in the exploration \cite{krahmer2012computational}, to be more computationnally efficent \cite{li2017automatically} or to over specify the RE \cite{viethen2013graphs}.

Other approaches also include learning for generating REs. Yamakata \textit{et al.} use a beliefs network based method to disambiguate entities based on multiple attributes \cite{yamakata2004belief}. Besides, they state that their algorithm runs on the hearer estimated belief network, we think that it is an important feature to generate relevant REs. However, they indicate that a belief network should be trained for each attribute, which can be really impractical in a real world robotic application.

Every approach presented until then are relying on REG dedicated knowledge bases or data structures. Such structures can be cumbersome to maintain in a dynamic world where relations between entities can change along the task. Moreover, in complete robotic architecture knowledge bases managing relations already exists, but are not dedicated to REG. The DIST-PIA method tries to mitigate this issue by having a domain-independent Incremental Algorithm querying dedicated knowledge base consultants to elaborate the RE \cite{williams2017referring}. This approach has been successfully integrated in a complete robotic architecture \cite{williams2019dempster}. Another work having been integrated into a robotic architecture is made by Ros \textit{et al.} \cite{ros2010one}. The knowledge base used is an ontology, which is more and more used in robotics to store symbolic knowledge. However, it does not support using the relations to generate REs (it only relies on the attributes of the entities). It has been integrated in a robotic architecture allowing the robot to guess the object the human is thinking of in a dynamic environment \cite{lemaignan2012grounding}.

To the best of our knowledge, none of these approaches have been used to determine the feasability and the cost of a referring communication action at task planning level.



\subsection{Task Planning with Communication Actions}

Recently, more and more research is dedicated to human robot verbal communication planning, mainly to answer the \textit{what} and the \textit{when} to communicate \cite{mavridis2015review}. The vast majority of works treats these questions during execution. Indeed, they assume a given plan (multi-agents or not) and insert verbal communication actions when needed.
\textit{Chaski} is a plan execution system allowing to perform collaborative activity with a human \cite{shah2011improved}. The system is able to generate verbal communication when starting or finishing a task allowing agents to coordinate their actions and to update their plans.
With their \textit{inverse semantic} algorithm, Tellex \textit{et al.} provide the robot with a capability to ask a nearby human for help when it fails \cite{tellex2014asking}. Indeed, when following a plan, if the robot detects an unfeasible action, it can plan which human action would help it in the plan and is able, using REG \textit{inter alia}, to verbally demand them to act. 
Sebastiani \textit{et al.} are able, by merging multiple multi-agent HATP plans, to generate conditional plans which then can be verbally negotiated (by asking the human about task allocation) during the execution with the human \cite{sebastiani2017dealing}. 
Devin and Alami proposed a supervision component which is able, when given a multi-agent plan elaborated by HATP, to estimate the beliefs of the human partner \cite{devin2016implemented}. Then, they monitor divergences between the robot and the human's beliefs. If a divergence is detected as not allowing the human to perform their next actions of the plan, a verbal communication aligning the needed belief is done by the robot. 

However, in all the previous work, the need and the content of communication actions are resolved only when executing the plan. This is in some case not enough and more recent works focus on resolving communication needs already at task planning level.
Roncone \textit{et al.} propose a task planner where domains are easily written and visualized thanks to an high-level task tree representation \cite{roncone2017transparent}. This domain is then changed into a POMDP which can be solved to obtain a policy. They define three types of verbal communication: (1) \textit{command} is a robot instruction to the human, which can be accepted or declined; (2) \textit{ask} allows the robot to question the human about the progress of their task; and (3) \textit{inform} makes the robot speaks about its next action intent. These three types of action are coded in the POMDP and may be included in the policy depending on the situation and their cost.
A similar approach has been realized by Unhelkar \textit{et al.} where they add one type of communication: \textit{answer} allowing the robot to answer a human querying about its next intent \cite{unhelkar2020decision}. These verbal communication actions are then integrated into a POMDP. This POMDP is elaborated thanks to a provided task model represented as an multi-agent MDP, a robot communication model (including communication cost model) and a human action selection model represented with an agent Markov model. This human model can be refined throughout the interaction. The POMDP is then solved to generate a robot policy.
It is interesting to note that in the presented works, the communication costs are only based on the time of execution (the \textit{when}) --- to ensure multiple communications are not too close in time --- but not on the content of said communication (the \textit{what}, \textit{e.g.} the length of the communication, the complexity of understanding it). Moreover, by not considering the content of the communication at planning time (communication actions are considered as template with arguments determined at execution time) they do not ensure that it will be feasible when executing. They mitigate this issue by only considering communication about the plan and actions, and not about belief alignment or object referring. This shows the interest of our approach as it tries to tacle, at planning level, two of the five challenges identified by Unhelkar \textit{et al.}: "estimating benefit of communication" and "quantifying cost of communication" \cite{unhelkar2017challenges}.
Finally, it appears clear in the presented works that planning for communication can only be done if the robot plans for both agents.


\improvement{Add a conductor example, the ones with cubes and areas maybe ?}
\section{Ontology based Referring Expression Generation for Human Robot Interaction}
To estimate the feasibility and the cost of communication action during task planning, we need to be able to quickly resolve the content of a communication. Since considering every type of communication would be intractable we focus on a special type of verbal communication: referring expressions.
In this section we present an efficient algorithm that is able to generate referring expression for human-robot interaction based on ontologies. We first introduce the concept of ontology and argue about its use in human-robot interaction scenarios. Then we propose a list a features needed for REG in human-robot interaction. Next, we formally define the problem of ontology-based REG for HRI, and present an efficient algorithm to solve it. Finally, we show the results of this approach both in term of found solutions and time complexity.

This part has been done in close collaboration with Guillaume Sarthou.

\subsection{Using ontologies for human robot interaction}
An ontology is a data representation used in many domains. In robotics it is more and more used as a knowledge base. It allows to represent multiple concepts inheriting from one another and entities as instantiation of these concepts. Moreover, the entities can be linked through properties representing relations. Reasoners can use this structure to deduce and complete the ontology. Recently, ontologies are even standardized for robotic application such as the IEEE-SA P1872.2 Standard for Autonomous Robotics Ontology.

Formally, as coined by ... \cite{ontology_def}, an knowledge base ontology is defined by the tuple $K = \langle \abox, \tbox, \rbox \rangle$. 
The \textit{TBox} $\tbox$ contains the concepts, called \textit{classes} representing the possible types of entities known by the agent. More specifically, it is a finite directed acyclic graph (DAG) $\tbox = \langle T, H \rangle$ with $T$ the set of classes/types and $H$ the directed edges representing the inheritance/inclusion links between them. For simplicity purposes we will refer to them as \textit{isA} links. In an ontology representing the example depicted in Figure~\ref{fig:chapter3_example}, we may have: $\{Cube, Table, Object, Agent, Pickable, Robot, Human\} \subset T$ and $\{(Cube, Pickable), (Pickable, Object), (Table, Object), (Robot, Agent), (Human, Agent)\} \subset H$ (\textit{i.e.} $(Cube, isA, Pickable), (Pickable, isA, Object), (Table, isA, Object), (Robot, isA, Agent), (Human, isA, Agent)$). 
The RBox $\rbox = \langle P, Incl, Inv \rangle$ contains the properties, their inheritances and inverses known by the agent. $P$ is the set of properties, $Incl$ the finite DAG representing inheritances/inclusions between the properties and $Inv = \{(p_i, p_j) \in P^2\}$ representing the inverse properties. In an ontology representing the example depicted in Figure~\ref{fig:chapter3_example} the RBox may include: $\{isIn, hasIn, isOn, hasOn, geometricProperty\} \subset P$, $\{(isIn, geometricProperty), (hasIn, geometricProperty), (isOn, geometricProperty), (hasOn, geometricProperty)\} \subset Incl$ and $\{(isIn, hasIn), (hasIn, isIn), (isOn, hasOn), (hasOn, isOn)\} \subset Inv$. Note that to fully match the definition of ... \cite{ontology_ref} it would require to declare the disjunctive, transitive, reflexive and chain relations in $\rbox$ and the disjunctive classes in $\tbox$. As they will reasoned upon in this thesis, we chose to omit them.
Finally, the ABox $\abox = \langle \indivset, C_0, R \rangle$ contains the entities, their types and relations. $\indivset$ is the set of entities. $C_0 = \{(a, t)|a \in \indivset, t \in T\}$ contains the direct types of each entities (an entity must have at least one direct type, but can have multiple ones). Finally $R = \{(s, p, o)|(s, o) \in \indivset^2, p \in P\}$ is the set of relations between entities. For example, in an ontology representing the example of Figure~\ref{fig:chapter3_example} we would have in the ABox: $\{cube_23, cube_12, table_1, human_3, pr2_robot\} \subset \indivset$ along with $\{(cube_23, Cube), (cube_12, Cube), (table_1, Table), (human_3, Human), (pr2_robot, Robot)\} \subset C_0$ and $(cube_23, isOn, table_1) \in R$.
By using the hierarchy of types we also define $C$ representing the graph of direct and inherited types of entities. $C$ is constructed by adding all the types that can be reached from a direct type of an entity by following a path in $H$. For example $(cube_23, Cube) \in C_0 \implies (cube_23, Cube) \in C \land (cube_23, Pickable) \in C \land (cube_23, Object) \in C$ if we reuse the example $H$ presented before.
We define the "isA" property for simplicity purpose. The "isA" property allows to represent hierarchy of types and entities types (as defined in $C$) while only representing triplet, as typical relation (\textit{e.g.} $(cube_23, isA, Cube)$, $(cube_23, isA, Pickable)$, $(cube_23, isA, Object)$. This definition is only intended to help with the notation.

In all the following work we will consider the TBox and RBox as static. They will be defined before any experiment and will not be modified at runtime. They can be considered as the semantic knowledge of the robot. The ABox on the other hand, will contain both predefined entities and relations but also sensed entities and computed facts. It will contain usual symbolic facts, computed by the situation assessment, found in the knowledge bases of typical robotics architecture. However, thanks to their typing and the hierarchy of both types and properties deduction and reasoning can be done on them.

In this thesis we will not present the different reasoners of the ontology, but rather assume that the ontologies used are all been preprocessed and are consistent (\textit{e.g.} if a relation is in $R$, all the inverse properties of this relation have been added to $R$).


% One ontology per agent
Finally, we want to be able to estimate and reason on the human beliefs. To do so, we will use one knowledge base (\textit{i.e.} ontology) per agent considered by the robot in addition to its own. To follow the notation of Chakraborti \cite{chakraborti2018human} presented earlier in this thesis, we will note $K^R = \langle \abox^R, \tbox^R, \rbox^R \rangle$ the knowledge base of the robot and $K^H_r = \langle \abox^H_r, \tbox^H_r, \rbox^H_r \rangle$ the robot estimated knowledge base of the human it is interacting with. In practice, we will have $\tbox^R = \tbox^H_r$ and $\rbox^R = \rbox^H_r$, and only have differences in the ABoxes.

\subsection{REG feature for communication action estimation during task planning}
We saw previously that REG is an important and interesting problem for human-robot interaction scenarios. However, as its application will be on an environment perceived in real-time, along a collaborative task and to a specific human, additional constraints have to been considered.

First, we want to be able to \textbf{use the relations between entities} to refer to one. 
Then, we want the algorithm to \textbf{run on existing knowledge bases}. Many presented approaches rely on a dedicated knowledge representation. Such representation can be cumbersome to maintain during an interaction in an evolving environment. 
Moreover, we claim that the ontologies used already contain the knowledge needed to perform the REG. We also want to support the \textbf{preference ordering} per agent. Indeed, some relations are understood better and quicker than other, and this preference can change depending on the agent we are interacting with. 
In addition, we want the algorithm to consider the verbalization through \textbf{the use of types}. All the approaches presented before only focus on the content determination of the REG and consider that the linguistic realization (the verbalization) will be perfect. They consider that all the content can be verbalized (it exists a word for every bit of the content and the content can be verbalized unambiguously). We state that the type is the minimal information needed to refer to an entity (\textit{e.g.} the \textit{cube\_23} cannot be verbalized directly as "cube 23", only its type can be verbalized as "the cube").
Likewise, we can imagine that in large robotic ontologies, every type or relation cannot be verbalized (\textit{e.g.} we do not want the robot to say \textit{Pickable} type, the \textit{geometricProperty} or the \textit{hasMesh} property). Thus, our algorithm should be able to \textbf{select only verbalizable types and properties}.
Finally, in an interaction, it is clear for the hearer that some entities will not be referred, and should not be taken into account as distractors by the algorithm (in the example depicted in Figure~\ref{fig:example_chapter3}, it should be clear to the human that, unless specified otherwise, if the robot ask about a cube, it is one on the table and not one in another room). Equally, some relations will be implied (\textit{e.g.} if the robot asks the human to \textit{give} it a cube, it is implied that the cube is not reachable by the robot and reachable by the human). Thus, the algorithm must \textbf{use the context of the ongoing task}.

\subsection{Ontology based REG problem definition}
To formally define the REG problem for HRI, we need to enhance our knowledge base with three functions.
First, we define a class labeling function $\labelfunc_t: T \mapsto str \cup \bot$ where $str$ denotes a set of strings used as words in the vocabulary. We define that a class $t \in T$ is labeled iif $\labelfunc_t(t) \neq \bot$ and call $\labelfunc_t(t) \in str$ the label of $t$. Besides, we require this label to be unique, \textit{i.e.} for any pair of labeled classes $t, t' \in T^2, t \neq t' \Leftrightarrow \labelfunc_t(t) \neq \labelfunc_t(t')$. We define similarly a an entity labeling function $\labelfunc_a: \indivset \mapsto str \cup \bot$ associating some entities to their speakable/writable unique names. These function can be defined in the ontology by using the commonly used property \textit{rdf:label} to the labeled classes and entities. Adding them that way, allows to make these function agent dependent. In the example depicted in Figure~\ref{fig:example_chapter3} we would have among others $\labelfunc_t(Table) = "table"$, $\labelfunc_t(Pickable) = \bot$ and $\labelfunc_t(Cube) = "cube"$.

Moreover, to support the preference ordering we introduce a \textit{comprehension cost function} depending on the agent $\costcompfunc^H: P \mapsto \realset^{+*}$. It allows to represent that some relations are harder to understand for the hearer than others. We will not present in this thesis how to compute these costs. However, some approaches using learning manage to estimate this cost \cite{belke2002tracking, koolen2012learning}.

We are aiming to unambiguously designate, through its relations to other entities, an entity $\goalindiv \in \indivset$ in a knowledge base $\knowledgebase$. We will call the entity we are trying to refer to the \textit{target entity} $\goalindiv$.
However, the RE is meant to be used in the context of a task. As stated previously, the RE needs to account for certain implicit relations. This is why the problem must be given a \textbf{context} $Ctx = (R_{ctx}, C_{ctx})$, a set of relations and direct types that are implicit in the current situation, which will be used to reference $\goalindiv$, but not included in the generated RE. For the interactions of Figure~\ref{fig:chapter3_example}, the context could be defined as $Ctx = (\{ \langle \goalindiv, isOn, table_1 \rangle, \langle \goalindiv, isVisibleBy, human_3 \rangle, \langle \goalindiv, isReachableBy, human_3 \rangle \}, \emptyset)$. With this context, we restrict the disambiguation to the entities present on the table $table_1$ and visible and reachable by \textit{human\_3}, the human partner.

Finally, to be able to run on our ontologies and select only verbalizable properties, we provide the problem with a set of \textbf{usable properties} $\usablepropset \subseteq P$. Because of properties inheritance $Incl$ all the properties inheriting from the ones in $\usablepropset$ are usable in the problem.

We thus defines the REG problem as follows:
\begin{definition}[The referring expression generation problem]
The referring expression generation (REG) problem is a tuple $\regproblem = \langle \goalindiv, \knowledgebase, Ctx, \usablepropset \rangle$ with $\goalindiv \in \indivset$ the target entity, $\knowledgebase$ the hearer's knowledge base as an ontology, $Ctx$ the context and $\usablepropset \subset P$ the set of usable properties.
\end{definition}
To find the more precise and robust RE the considered knowledge base is the estimated hearer's one.


A solution to the REG problem is a set of relations which could be verbalized afterwards.
Because some entities are \textit{not} labeled with a unique name (anonymous) and thus cannot be referred to directly, some of the relations might be under-specified. For instance, the sentence ``the cube is black'' is under-specified in that ``the cube'' does not identify a unique entity but any entity with the class ``cube''.
In addition, it might be the case that a unique, anonymous, entity participates in more than one relation, e.g., ``the cube is black and on the table''. 
To keep track of anonymous entities in underspecified relations, we introduce a variable set $X$, representing the anonymous entities. By convention, variables will be prefixed with a question mark (e.g. $?y \in X$).
An underspecified relation is thus a triple $(s, p, o) \in (X \cup \indivset) \times \usablepropset \times (X \cup \indivset)$, e.g., \textit{(?y, hasColor, black)} where $?y \in X$ is a variable and $black \in A$ is a labeled entity in the knowledge base.


When speaking about anonymous entities, one must know its type to serve as a placeholder in sentences (e.g. "the pen").
Thus, the solution should associate each variable and a type. For simplicity, we chose to represent them also as triplets: $X \times "isA" \times T$ (e.g. \textit{(?y, isA, Cube)}).

\begin{definition}[Reference]
Thus, a \textbf{reference} $E$ is a set of triplets, each triplet in $E$ being either an under-specified relation in $(X \cup \indivset) \times \usablepropset \times (X \cup \indivset)$ or a type ascription in  $(X \times "isA" \times T)$.
\end{definition}

However, a reference may not be verbalizable as is, nor  represent a valid situation of the knowledge base. We thus introduce three constraints:

\begin{constraint}[Nameability of entities]
\label{theo:constraint_1}
Each entity $a \in \indivset$ present in any tuple of a reference $E$ (as first or third component) must have a label: $\labelfunc_a(a) \neq \bot$.
\end{constraint}

\improvement{Example violating C1}

\begin{constraint}[Nameability of variables]
\label{theo:constraint_2}
For each variable $x \in X$ present in any tuple of a reference $E$ (as first or third component) there must also be a unique tuple in $E$ specifying one of its labeled type ($(x, "isA", t) \in E$ with $t \in T$ and $\labelfunc_t(t) \neq \bot$.
\end{constraint}

\improvement{Example violating C2 but not C1}

\begin{constraint}[Correct instantiation of variables]
\label{theo:constraint_3}
For a reference $E$ there must exists at least one substitution function $f: X \mapsto \indivset$ of the variables in $E$ into entities in $\indivset$ such that the types and relations linking entities in $E$ are still present in $T$ and $R$ once $f$ has been applied.
In practice, $f$ transforms the underspecified relations of $E$ into fully specified ones that must appear in the knowledge base.
\end{constraint}

\improvement{Example violating C3 but not C2 nor C1}

We can now define a \textbf{valid reference}:

\begin{definition}[Valid reference]
\label{theo:valid_ref}
A reference $E$ is valid with respect to an ontology $\knowledgebase$ if and only if it respects the constraints \ref{theo:constraint_1}, \ref{theo:constraint_2} and \ref{theo:constraint_3}.
\end{definition}

Besides, we define a solution and a complete solution to a REG problem $\regproblem = \langle \goalindiv, \knowledgebase, Ctx, \usablepropset \rangle$:

\begin{definition}[Referring expression]
\label{def:re}
A solution to a REG problem $\regproblem = \langle \goalindiv, \knowledgebase, Ctx, \usablepropset \rangle$ is called a referring expression and is a tuple $S = \langle E, x_g \rangle$. $E$ is a valid reference and $x_g \in X$ is a variable, such as for each mapping function $f$ respecting the constraint \ref{theo:constraint_3}, $f(x_g) = \goalindiv$.
\end{definition}

\begin{definition}[Complete referring expression]
\label{def:complete_re}
A complete solution to a REG problem $\regproblem = \langle \goalindiv, \knowledgebase, Ctx, \usablepropset \rangle$ is a solution where the mapping function $f$ respecting the constraint \ref{theo:constraint_3} is unique.
\end{definition}

\improvement{Examples}

Finally, we define an optimal solution (referring expression) $S^* = \langle E^*, x_g \rangle$ as being the a solution minimizing $\sum_{(s, p, o) \in E^*}\costcompfunc(p)$ over the set of all possible solutions for a REG problem.


\subsection{Efficient REG algorithm presentation}
\subsubsection{Formalization as a graph search problem}
\label{sec:SCFormalisation}

Let \textbf{node} $\node = \langle \mathcal{T}_\node, X_\node, A_\node, \mathcal{S}_\node$. $\mathcal{T} \subseteq \relationset \cup C$ is a set of triplet relations representing some relations in the knowledge base $\knowledgebase$. $X_\node \subseteq X$ is the variable set used in this node, $A_\node \subseteq \indivset$ is the set of anonymous entities of $\mathcal{T}_\node$ and $\mathcal{S}_\node: X_\node \mapsto A_\node$ is the bijective mapping function linking variables to the anonymous entities they represent. We will note $\mathcal{S}^{-1}(T)$ the resulting \textit{reference} (as defined in Definition~\ref{def:reference}) after the application of $\mathcal{S}^{-1}$ on all the entities in each triplet of $T$ which is also in $A_\node$.
The \textbf{initial node} is specified by the user's query through the $context$ of the problem.
The idea is then to explore these nodes until the reference generated from the node $\mathcal{S}^{-1}(T)$ is valid and solution of the REG problem.
 
To find all substitution functions defined in the Constraint~\ref{theo:constraint_3}, and thus, all the entities which can be bound to the variables in the reference, we use the \sparql{} queries presented previously. From any node $\node$ we can easily construct a \sparql{} query from $\mathcal{S}^{-1}(T)$, and submit it on the knowledge base to know how many entities can bound to the variables of the request.
A node $\node$ is a \textbf{goal node} if $\goalindiv$ is the only solution to the variable $x_g$ of the \sparql{} query created from the node (Definition~\ref{def:re}), and possibly all the variables in the \sparql{} query have only one assignation (Definition~\ref{def:complete_re}).

A \textbf{transition} $\transition$ in the unambiguous reference generation problem consists in the insertion of a new triplet $(s, p, o)$ to the set $\mathcal{T}_\node$ of a node $\node$ resulting in the creation of a new node $\node'$. The inserted relation in a node $\node$ can be a typing relation ($p \equiv isA$) or a relation which differs between ambiguous entities in $\node$. 
We define two kinds of difference between ambiguous entities.
\begin{definition}[Hard difference]
A \textbf{hard difference} ($a_i, \harddiff, a_j$) exists when two entities own the same property towards a different entity (i.e $(a_i, p, b_i) \in \relationset \land (a_j, p, b_j) \in \relationset | b_i \neq b_j$).
\end{definition}

\begin{definition}[Soft difference]
A \textbf{soft difference} ($a_i, \softdiff, a_j$) exists when an entity owns a property that is not owned by another ambiguous entity (i.e $(a_i, p, b_i) \in \relationset \land (a_j, p, \cdot) \notin \relationset$).
\end{definition}

\improvement{Example of soft and hard difference}

As the hard differences respect the \textbf{open-world assumption} but the soft differences do not, we propose to encourage the use of hard difference when possible by adding an extra-cost to transitions coming from soft differences.

Finally, the \textbf{cost} of a node is the sum of the cost of each transition leading to this node. If we assume that each transition $\transition_j$ corresponds to the addition of a triplet $(s_j, p_j, o_j)$ to the set $\mathcal{T}_\node$ of a node $\node$ with a cost $\costcompfunc(p_j)$, the cost to $\node$ is $\costcompfunc_\node = \sum_{(s, p, o) \in \mathcal{T}_\node} \costcompfunc(p)$. 

\subsubsection{Algorithm presentation}
We chose to perform this search and solve the REG problem to use an uniform cost search algorithm on the graph presented before.
From an initial node built from the context of the query, the algorithm generates new nodes by adding possibly disambiguating relation to the current node. We use an uniform-cost search which is \textbf{optimal} and \textbf{complete} with positive transition costs and a finite number of entities and properties in $\knowledgebase$. Just like Dijkstra's algorithm, it expands the nodes in increasing cost order until a solution is discovered or the search space is exhausted.
%On this basis, we can use a transposition table with the hash of the explored states and thus detect if a state has already been explored or not.
%Informed search and bidirectional search have been both discarded because no admissible heuristic can be defined and because we can not sample a goal state directly. A breath-first search is optimal when all steps costs are equal because it always expands the shallowest unexpanded node. However, the cost of our actions are directly linked to the cost of the relation they contain, and thus, not always equals. 
%In this case, the breadth-first search is not optimal and is therefore not suited to our problem.
%Unlike the breadth-first search, the uniform-cost search (just like Dijkstra algorithm) expands the node with the lowest cost.
%Therefore, we chose to use a uniform cost search algorithm to find the optimal solution.
%States are expanded in increasing cost order until a solution is discovered or the search space is exhausted.
%Pseudocode of the uniform-cost search for the REG problem is given in Algorithm \ref{alg:ucs}.


\begin{algorithm}[H]
\begin{algorithmic}[1]
\Function {REG}{$\goalindiv$, $\knowledgebase$, $Ctx$, $U$}
\State $node \leftarrow Ctx$
\State $frontier \leftarrow$ a priority queue of nodes ordered by their \textit{cost}, initialized with $node$ having a cost 0 as only element
\State $explored \leftarrow$ an empty set
\Loop
	\If{\textsc{IsEmpty}($frontier$)} 
		\State \Return failure		
	\EndIf
	\State $node \leftarrow \textsc{Pop}(frontier)$
	\If{\textsc{GoalTest}($node$)} 
		\State \Return $\mathcal{S}^{-1}(\mathcal{T}_{node})$
	\EndIf
	\State $explored \leftarrow explored \cup node$
	\ForEach{$transition$}{\textsc{GetTransitions($node$)}}
		\State $child \leftarrow \textsc{ApplyTransition}(node, transition)$
		\If{$child \notin explored$ and $child \notin frontier$}
			\State \textsc{Insert}($child$, $frontier$)
		\EndIf
	\EndFor
\EndLoop
\EndFunction
\end{algorithmic}
 \caption{Uniform cost search algorithm for referring expression generation}
 \label{alg:reg}
\end{algorithm}

\begin{algorithm}[H]
\begin{algorithmic}[1]
\Function {GetTransitions}{$node$}
\State $transitions\leftarrow$ \textsc{TypingTransitions}($node$)
\If{$transitions \neq \emptyset$}
	\State \Return $transitions$
\EndIf
\State $transitions\leftarrow$ \textsc{DifferenceTransitions}($node$)
\State \Return $additions$
\EndFunction
\end{algorithmic}
 \caption{How to write algorithms}
\end{algorithm}

\begin{algorithm}[H]
\begin{algorithmic}[1]
\Function {TypingTransitions}{$node$}
\ForEach{$(s, p, o)$}{$T_{node}$}
\If{$ \nexists x \text{~s.t.~} (s, $"isA"$,x) \in \mathcal{T}_{node} \land \labelfunc_a(s) = \bot$}
\State \Return $\{\ (s,\text{"isA"},t)\ |\ t \in \textsc{UsableClasses}(s)\ \} $ and the creation of a new variable 
\EndIf
\EndFor
\State \Return $\emptyset$
\EndFunction
\end{algorithmic}
 \caption{Typing transitions pseudocode}
 \label{alg:typingtrans}
\end{algorithm}

\begin{algorithm}[H]
\begin{algorithmic}[1]
\Function {HardDifferenceTransitions}{$node$}
\State $transitions\leftarrow$ an empty set of transitions
\State $\mathcal{M}\leftarrow$ \textsc{SparqlResult}(\textsc{ToQuery}($\mathcal{S}_{node}^{-1}(\mathcal{T}_{node})$))
\ForEach{$x$}{$X_{node}$}
	\ForEach{$a$}{$\mathcal{M}(x)$}
		\If{$a \neq \mathcal{S}_{node}(x)$}
			\ForEach{$r = (\mathcal{S}_{node}(x), p, o)$}{$\mathcal{S}_{node}(x) \Delta a$} \label{line:harddiff}
				\State $r_{inv} \leftarrow (o, Inv(p), \mathcal{S}_{node}(x))$
				\If{$r \notin \mathcal{T}_{node} \land r_{inv} \notin \mathcal{T}_{node} \land p \in U$}
					\State $transitions \leftarrow transitions \cup \{r\}$
				\EndIf
			\EndFor
		\EndIf
	\EndFor
\EndFor
\State \Return $transitions$
\EndFunction
\end{algorithmic}
 \caption{Hard difference transitions pseudocode}
 \label{alg:harddifftrans}
\end{algorithm}

\begin{algorithm}[H]
\begin{algorithmic}[1]
\Function {ApplyTransition}{$node$, $transition$}
\State $newnode \leftarrow$ a copy of $node$
\State $(s, p, o) \leftarrow transition$
\If{$p \equiv$ "isA"}
	\State $x \leftarrow$ a new variable such that $x \in X \land x \notin X_{newnode}$
	\State $X_{newnode} \leftarrow X_{newnode} \cup x$
	\State $A_{newnode} \leftarrow A_{newnode} \cup s$
	\State Update $mathcal{S}_{newnode}$ such that $\mathcal{S}_{newnode}(x) = s$
\EndIf
\State $\mathcal{T}_{newnode} \leftarrow \mathcal{T}_{newnode} \cup (s, p, o)$
\State \Return $newnode$
\EndFunction
\end{algorithmic}
 \caption{Transition application pseudocode}
 \label{alg:transapply}
\end{algorithm}


\textbf{\textsc{ToQuery}:}
Performs a direct translation of a \textit{reference} into a \sparql{} query.

\textbf{\textsc{SparqlResult}:}
The function that takes a \sparql{} query as input and returns a match table $\mathcal{M}: X \mapsto \mathcal{P}(A)$ in the way that $\mathcal{M}(x)$ is the set of entities matching the variable $x \in X$ in the given query.

\textbf{\textsc{GetTransitions}:}
At each step, we consider two kinds of possible transitions. The \textsc{TypingTransitions}~function (Alg.~\ref{alg:typingtrans}) consisting in the addition of an inheritance relation if at least one entity has no label and no inheritance relation in $\mathcal{T}_{\node}$. Otherwise, the \textsc{DifferenceTransitions} concatenates the transitions from the \textbf{hard difference transitions} (Alg.~\ref{alg:harddifftrans}) and the \textbf{soft differences transitions} (Alg.~\ref{alg:harddifftrans} with the $\softdiff$ operator at line~\ref{line:harddiff}. These transitions add relations that differ as hard and soft differences between ambiguous entity for each variable in $\mathcal{M}$.

The $\harddiff$ (resp. $\softdiff$) operator returns all the relations that are hard differences (resp. soft) between two entities as defined in \ref{sec:SCFormalisation}. In the difference actions algorithm, an action can be added only once and must not be present in the current state to avoid redundancy. The inverse relation to the one added by the action is also retrived from the $Inv$ set defined in the knowledge base and checked if not present in the current state and in the current actions set, again to avoid redundancy. 
%In the example of Fig.~\ref{fig:search_example} the relation $(P\_1, isIn, G\_2)$ will be redundant if the relation $(G\_2, hasIn, P\_2)$ has been already used in the current state.

\textbf{\textsc{TypingTransitions}:}
The \textsc{TypingTransitions} function stops at the first entity which has no label nor type. This specificity reduces the branching factor while ensuring that each entity has a label or at least a type. Since typing actions are the first tested in the \textsc{GetTransitions} function, all entities not typed during a first execution will be during the next ones. In the implementation, this function has been optimized by observing that once all the entities from the context are typed, the only entities in $\mathcal{T}_{\node}$ which may be be not typed are added as the \textit{object} of a \textsc{DifferenceTransitions}. Thus, by storing the object entity of a transition and only checking if it is labeled or has already been typed (and is thus present in $A_{\node}$) we reduce the complexity of the \textsc{TypingTransitions} function.

\textbf{\textsc{UsableClass}:}
The function \textsc{UsableClasses} returns the most specific \textit{labeled} classes of an entity $\indiv$, i.e., the set of classes $t \in \classset$ such that $(\indiv, t) \in C$, $\class$ is labeled and there are no labeled subclasses of $t$.

This strategy differs from the one of \cite{dale_computational_1995} that prefers the least specific types (so called basic-level classes).
However, in domain-independent knowledge bases such as ours their scheme could often resolve to "Object" or "Thing" which can lead to confusion.
Furthermore, by being conservative in our estimation of the receiver's knowledge base, we can guarantee that the labels of the considered classes are known to the human partner.
Finally, using the most specific classes might reduce the ambiguities, and thus the branching factor early in the search, without impacting completeness.
Note that the restriction to the most specific classes is not necessary but might reduce the branching factor of the algorithm without impacting completeness.

\textbf{\textsc{ApplyTransition}:}
The \textsc{ApplyTransition} function creates a new node $\node'$ by applying a transition to an existing node $\node$. It always add the triplet of the transition to $\mathcal{T}_{\node'}$ but, in case of a transition coming from the \textsc{TypingTransitions} function, a new variable is created and added to the mapping function $\mathcal{S}_{\node}$. Indeed, if an entity needs to be typed, it means that it is unlabeled and thus need to be represented through a variable in a valid \textit{reference}.

\subsection{Results}
We present hereafter the solutions given by our algorithm to the illustrative examples. Then we provide results involving a large scale knowledge base describing a full apartment in terms of time-execution, solution length and composition. Finally, we provide comparative performance measures with two state-of-the-art methods on their own domains.

\subsubsection{Solutions analysis}
In order to familiarize with solutions, we propose to present some of them. For every presented solution, the variable denoting the entity to refer to will be $x_g = ?0$.
The first setup is for illustration purposes, %it is simulated with a knowledge base which is given and static (figure \ref{fig:search_example}). 
and operates on the static knowledge base illustrated in Fig.~\ref{fig:search_example}.
Since this setup is really small, the context is always empty, all the relations are usable and no entity is labeled.
We only tested with two interesting entities since the others present similar characteristics.
The solutions for P\_1 and G\_1 are respectively \textit{\{(?0, isA, Pen), (?0, isIn, ?1), (?1, isA, Cup), (?1, Color, blue)\}} and \textit{\{(?0, isA, Cup), (?0, Color, blue)\}}, which can be read respectively as "the pen in the blue cup" and "the blue cup". These two solutions are R2 (allowing to read "the" and not "a" in the verbalization), as $?0$ and $?1$ bind to only one entity. Here, we see how referring to another entity lead to interesting solutions.

In order to give the reader a sense of how the context is useful as defined in the problem, we propose to come back to the Fig.~\ref{fig:pens}.
In a knowledge base describing Fig.~\ref{fig:pens}(b), with a labeled entity \textit{Bob}, representing the human, giving a empty context to the problem would lead to the solution \textit{\{(?0, isA, Pen), (?0, isReachableBy, Bob)\}}, which would read as "The pen reachable by Bob". Whereas, if the robot wants the human to give it the pen, the reachablity of the pen is obvious. So the context would become: \textit{\{(pen0, isReachableBy, Bob)\}}, the ensuing solution would be \textit{\{?0, isA, Pen)\}}, simply verbalizable as "the pen", as taking into account the given context resolve the ambiguity.

\subsubsection{Scaling up}

To assess the relevance of our approach, we created a larger, realistically-sized, knowledge base (101 entities, 36 classes, 40 properties and 497 relations), describing an apartment with three rooms including several furniture (tables, shelves) and objects (cups, boxes) linked through geometrical relations (atLeftOf, onTopOf) and attributes (color, weight).
%\footnote{The complete ontology is available at [hidden for review]}. 
We ran our algorithm over all the 77 entities inheriting from the "Object" class, representing physical entities. 

\newcommand{\us}{$\mu$s\xspace}

As this algorithm must be used in a human robot interaction application, we want it not to spoil the interaction when the robot is computing an explanation. In this setup, 100\% of the entities have been referred in under 4.33ms that is well bellow 100ms which is the maximum system response time for the user to get a feeling of instantaneity \cite{miller_response_1968}. More over, 50\% are referred under 357\us and 75\% under 772\us. On average, 10.6 nodes are explored to refer to an object with an average of 67.35\us/node explored. These execution time are promising from a combined use with a task planner, as many requests can be performed while planning without slowing too much the task planner.

Over the 77 entities, 32 (41.56\%) are referred with 2 or less relation meaning that only the type of the entity and one relation is needed to refer to them. We can also note that 25 entities (32.46\%) are referred using 4 or more relations with a maximum of 6 for one of them. Finally, 49.4\% need to be referred by referring to another entity and two of them need to be referred by referring to two other entities. This mean that 49.4\% of the entities can not be referred using approaches like \cite{ros_which_2010} or \cite{dale1995computational}. For this reason we will not compare more of these two works.

These results over a large scale knowledge base highlight the need to be able to refer to an entity through the use of relation linking it with other entities. They also shows that the use of the type of an entity is often sufficient with the use of only one attribute. With this experiment we also demonstrate that our algorithm is suitable for a use with a realistic large scale knowledge base.

\subsubsection{Comparisons with other state-of-the-art algorithms}

\textbf{Longest First} The Longest First (LF)\footnote{http://www.m-mitchell.com/code} algorithm \cite{viethen2013graphs} has been tested on the GRE3D3 Corpus composed of 20 scenes with three objects with different spatial relations relative to one another (onTopOf, atLeftOf). Each object can be referenced by its color, its size (large or small) and its type (cube or ball). The target referent is marked by an arrow and is always in a direct adjacency relation (onTopOf or inFrontOf).
Among the 20 scenes, 8 target objects can be referenced without any ambiguity using only their type, 7 can be referenced using only their type in addition to an attribute (color or size) and the other five can be referenced using their types and both color and size attribute. This means that spatial relations are never necessary to reference the target object. 
We perform the comparison on the 19\textsuperscript{th} case which consists of a small green cube on a large green cube and a small blue cube to the right of the green cubes. We chose this case with only cubes because the LF algorithm does not consider the types when generating the RE and adds them only as a post-process. The other cases requiring only the type are resolved in less than 100\us and those requiring the type and an attribute are resolved in less than 250\us with our algorithm.

Because of their objective of obtaining an over-specification of the RE, their results are strongly impacted by the maximum length parameter. By setting it to 4 as recommended, we get the result which we can read as \textit{"The small green cube on top of a cube"} in 311ms. By setting the maximum length to 3 we obtain the shortest admissible result which can be read as \textit{"The small green cube"} in 109ms. This last result is the one given by our algorithm in just 0.87ms \footnote{Times reported are run on a CPU Intel Core i7-7700 CPU @ 3.60GHz with 32 Go RAM}.

We see here that the results given by the LF algorithm largely depend on the maximum length parameter. This parameter also has a significant impact on the execution time. Besides, in the realistic scenario presented previously, 13\% of the entity need a reference expression length greater than 4. Thus, even if the over-specification is the goal of the LF approach, it can hardly scale-up. Moreover, for a maximum length fixed at the optimal length, the two approaches give identical results.

\textbf{Graph Based Algorithm}
A speed up of the original GBA~\cite{viethen_graphs_2013} is presented in~\cite{li_automatically_2017}. It aims at extracting, from a dedicated entities relations graph $G$, the lowest cost subgraph which is graph isomorphic to one and only one subgraph in $G$ containing the entity to refer to.
Their approach is evaluated on a corpus containing multiple tabletop scenes \cite{li_spatial_2016}, presenting numerous cubes of different colors.

We generated the graph (relations and costs) used for the scene 1, converted it into an ontology, and ran our algorithm on it.
This scene contains 15 cubes, GBA algorithm and ours are able to find a solution for the same 10 of them. In all the 10 cases, as we used the same costs, both algorithms returned the same solution (with the types of used entities added in our approach). 
For the other 5 cases, the two algorithms detect the absence of a solution in a few milliseconds.

On all the 10 cases with a solution, our approach performs faster than theirs (29.4 times faster in average). We can note that the speed increase is more important in cases where there are many solutions (under 4 times faster on 50\% of the cases, but more than 50 times faster for 25\% of the cases, up to 130 times faster). 
Indeed, the GBA approach uses a branch and bound algorithm where the search graph is bounded if the branch exceeds the cost of the current best found solution. Thus, it can explore a large part of the graph if the optimal solution is not found early in the search. Whereas our approach uses an uniform cost search algorithm, ensuring the first found solution is optimal.
Moreover, we think that on cases where the knowledge base contains entities with different types, our approach should work faster, since we prioritize the use of the type. We were not able to test this, as we could not manage to run their approach on other data than their own corpus.

\subsection{Integration}

Our ontology-based REG method has been integrated on a PR2 robotic platform and used in a tabletop scenario. %The used architecture presented in this section is represented if figure \ref{fig:archi}. 
The objects on the tables are detected with the ROBOSHERLOCK\footnote{\url{http://robosherlock.org/}} \cite{beetz_2015} perception system.  
%This software does not require any a priori on the environment such as CAD model, mesh or training. 
It provides the position (not used to extract relations), the shape ("circular" or "rectangular"), the color and the size of the objects ("large", "medium" or "small"). Since the types of objects is not determined by the system, all the objects were set with the labeled type "Object". This allows us to challenge our method with situations where the robot is not able to use high-level concepts and where various ambiguities will be raised.

The knowledge base is managed using the Ontologenius\footnote{\url{https://sarthou.github.io/ontologenius/}} system \cite{sarthou_2019}. 
It uses a custom internal structure to store and manipulate assertions as triplets, and offers reasoning capabilities in the form of plugins. Ontologenius provides a low level API allowing to manipulate the knowledge base as a classical data structure in addition to a \sparql{} interface.
The ontology is dynamically fed to keep it up to date on the basis of a simple situation assessment consisting only of filtering and object tracking.
%At the moment, the situation assessment used is quite simple and consists solely of filtering and object tracking. Based on the confidence on the properties extracted, it dynamically feeds the knowledge base to keep it up to date. To go further, we plan to use a software such as \cite{milliez_framework_2014} to extract higher level relations.
% \cite{sisbot_situation_2011} or

A simple linguistic realization has been made, taking as input a \sparql{} query and generating an English sentence as output. For example, it transforms the query "?0 isA Cup, ?0 isOn ?1, ?1 isA Table, ?1 hasColor black" into "\textit{the cup on the black table}". It is an ad hoc implementation based on a simple grammar and the labels present in the ontology. 
%It will not be further detailed in this paper.

The task \footnote{Commented video available at: \url{https://frama.link/TWU\_VE0o}} involves six objects on a table. The entity to reference is obj\_4 (a white mug). The robot generates the solution \textit{"The white circular object"} since there are other non-white circular objects and other white non-circular objects. Then, a human adds a new object which is a white and circular milk bottle (obj\_5). When the robot is asked to describe the white cup, it generates the sentence \textit{"The white small circular object"}. With this simple task, we show that our REG algorithm can be used within a robotic architecture, can deal with dynamic environment and can adapt its explanation to the current situation.

\improvement{Add small transition sentence here}


\section{Planning Communication Actions Using Referring Expression Generation}



\section{Using Past Actions in Referring Expression Generation}

\section{Conclusion}

\ifdefined\included
\else
\bibliographystyle{acm}
\bibliography{These}
\end{document}
\fi

\ifdefined\included
\else
\documentclass[a4paper,11pt,twoside]{StyleThese}
\include{formatAndDefs}
\sloppy
\begin{document}
\setcounter{chapter}{3} %% Numéro du chapitre précédent ;)
\dominitoc
\faketableofcontents
\fi

\chapter{Emulating the human planning process during task planning}
\label{chapter:doublehtn}
\minitoc

\section{Introduction}
In the previous chapter we successfully integrated a verbal communication planner into a multi agents task planner. This allows to avoid many plan failures, repair actions or unefficiency during the execution. Although we saw that this approach needs a task planner able to maintain one set of beliefs per agent during the planning process, the planner used was only allocating task to the human without considering that the human is not aware of the plan generated.

In this chapter we propose a first step to tackle this issue. We present a novel approach, where, instead of planning and allocating tasks to the human without planning for any plan communication, use a human action model to predict possible human actions, considering their reaction and own planning process.

First, we describe our approach and introduce the notations used in this chapter before detailing the planning process. Then, we demonstrate the planner capabilities on two example situations. Finally, we present a task inspired from psychology which is particularly interesting for HRI and has never been tackled in robotics yet, and we show how our planner has been integrated in a fully functional robotic architecture dedicated to this new task.

\section{Description}
We separate agents involved in a given task into two categories: the controllable agent (\textit{i.e.} the robot) for which the planner needs to select the best course of actions to generate a plan; and the uncontrollable agent (\textit{i.e.} the human) on whom the planner has no direct control but, still, has a representation of their decision and action models. The two agent types are fundamentally different: (1) the robot is controllable since the process is run by the robot, (2) the human agent is not controllable since the process can only "speculate" on her/his decisions and actions, but can model that the robot actions can still influence them, (3) the two agents are not equivalent, the robot agent role is to help, assist and facilitate human and to synthesize pertinent, legible and acceptable behavior.
We want to devise a planner allowing the controllable agent to plan for its actions while anticipating the decisions, actions and reactions of the uncontrollable agent. Moreover, we want the planner to be able to generate plans where the robot actions elicit situations calling for human decision, action and reaction, thus creating and anticipating collaboration and interaction.

This problem may be seen as a classical non deterministic planning problem, but enriched with the ability of the robot to model the actions, beliefs and decision process of the human. Thus, we have to consider distinct action models, beliefs and execution streams for each of the agents involved. Doing so with classical STRIPS-style planning approaches would lead to an intractable search space. Therefore, we chose to use HTN planning for both the controllable and uncontrollable agents. HTN planning aims at decomposing abstract tasks into atomic primitive tasks by choosing from a list of available context-dependent refinements for each abstract task, ensuring that preconditions and effects of refined primitives tasks are respected throughout the created plan. Similarly to HATP~\cite{sebastiani2017dealing}, our planner elaborates a plan with several streams of actions each associated to an agent involved in the task. But while in HATP, all the streams are built starting from on initial root node corresponding to a shared goal of all agents, our planner starts from multiple initial root nodes corresponding to the decision process of the different agents.

\paragraph{\bf Beliefs:}
Let $\statespace$ be the set of all possible world states, we call beliefs of an agent $\agent$ the state $\worldstate_{\agent} \in \statespace$ in which this agent thinks the world is in. It is important to note that the state of the controllable agent is assumed to be the world state estimation for the planner, as we consider the planner as being part of the controllable agent.

\paragraph{\bf Action models:}
We represent the action model of an agent $\agent$ as $\actionmodel_{\agent} = \langle \operators_{\agent}, \abstracttasks_{\agent}, \methods_{\agent} \rangle$ where $\operators_{\agent}$ are the primitive tasks (\textit{i.e.} operators, actions) that the agent $\agent$ can perform, $\abstracttasks_{\agent}$ the set of abstract tasks and $\methods_{\agent}$ are the methods (\textit{i.e.} decompositions) describing how an agent $\agent$ can perform an abstract task though a refinement process.

\paragraph{\bf Agents agendas and plans:}
An agenda $\agenda_{\agent}$ and a plan $\plan_{\agent}$ (this agent only stream of actions) are defined for each agent $\agent$. The agenda $\agenda_{\agent}$ is a list of tasks (abstract or primitive) having to be performed by the agent. The plan $\plan_{\agent}$ is a list of primitive tasks, built from the agenda, which the agent has to perform.
Coordination between agent plans are represented by causal links between streams which correspond to effects of agents actions on the beliefs states of the other agents. 

\paragraph{\bf Agent triggers:}
We then define for each agent $\agent$ a set of so-called \textit{trigger functions} $\triggerset_{\agent}$. These trigger functions aim at representing reactions of agents to certain situations (subsets of worlds states).

\paragraph{\bf Agents:}
Finally, we define an agent state as a tuple $\agentstate_{\agent} = \langle  \agenda_{\agent}, \plan_{\agent}, \worldstate_{\agent} \rangle$, and an agent as being $\agent = \langle \text{name}_{\agent}, \agentstate_{\agent}, \actionmodel_{\agent}, \triggerset_{\agent} \rangle$. Then we define two agent: the controllable one --- the \textit{robot} ---; and the uncontrollable one --- the \textit{human} ---. Let $\agentsstatesset$ be the set of all the possible agents states.

\section{Planning process}
The cooperative agents planning problem consists in two agents $\agents_{start}$ with their respective agenda filled with tasks to achieve and their beliefs about the current world. For the controllable agent, the beliefs correspond to the planner ground truth, for the uncontrollable agent, their beliefs need to be estimated, through, for example, situation assessment component~\cite{milliez2014framework, lemaignan2018underworlds}.

The result is a robot conditional plan $\policy$ being a tree of alternating robot and human primitive tasks. Any path from the root to the leaves is a feasible sequence of primitive tasks (\textit{i.e.} each primitive task application leads to a state where the following one is applicable) leading to a state where all the controllable agents agenda are empty.

To solve such a problem we need to augment the search space from world states $\statespace$ only to all the agents states considered by the planner $\agentsstates$, with their agenda, plan and beliefs. The exploration starts with $\agentsstates^{start}$ and consecutively applies operators associated to the robot and to the human, leading to new agents states $\agentsstates^{i}$ until the controllable agent has an empty agenda: $\agenda_{robot} = ()$.

\subsection{Action models restriction}
Considering the definitions above, for any agent $\agent$ the operators are defined as functions: $\operators \ni o: \agentsstatesset \rightarrow \agentsstatesset \cup \bot$ which produce new agents state, being the effect of the application of the primitive task, or \textit{false} if the task is not applicable.
Methods are defined as tuple, containing an abstract task and a decomposition function: $\methods \ni m = \langle \alpha, \delta \rangle$ with $\alpha \in \abstracttasks$ and $\delta: \agentsstatesset \rightarrow (\operators \cup \abstracttasks)^n \cup () \cup \bot$ with $n \in \intset^*$, which, depending on agents states, decompose the abstract task returning a list of tasks (primitive or abstract), an empty list if the abstract task does not need to be decomposed, or \textit{false} if the task cannot be decomposed in the current state. Multiple methods can address the same abstract task, the goal of the HTN planner is then to choose the right one to create a plan.
Finally triggers function are defined as: $\triggerset \ni t: \agentsstatesset \rightarrow (\operators \cup \abstracttasks)^n \cup ()$ with $n \in \intset^*$, returning a list of tasks to be inserted in an agent agenda as a reaction to specific agent states. 

However, some constraints on these functions must be respected.  Indeed, depending on whether the agent is controllable or not, their planning process will not take decisions based on the same information, and their action will not impact the world state in the same manner. We thus impose restrictions on what a function can read and write (writing means here having effects on agents states and is only in the case of primitive task functions) in the agents state. Then, the function constraints also depend on which agent is performing the action or making the decision (in method and trigger functions). When a function is applied we note \textit{self} the agent which executes it and \textit{other} the other agent. The rules for read and write access are given in Table~\ref{table:function_restrictions}. 
\begin{table}
\centering
\begin{tabular}{|c||c|c|} 
 \hline
 Agent type & Readable & Writable \\
 \hline
 Controllable & \begin{tabular}[c]{@{}l@{}}$\worldstate_{self}, \worldstate_{other},$ \\ $\plan_{self}, \plan_{other}$ (1)\end{tabular} & \begin{tabular}[c]{@{}l@{}}$\worldstate_{self}, \worldstate_{other},$(2)\\ $\agenda_{self}, \agenda_{other}$ (3)\end{tabular}\\
 \hline
 Uncontrollable & $\worldstate_{self}, \plan_{self}$ (4) & $\worldstate_{self}, \worldstate_{other}, \agenda_{self}$ (5)\\
 \hline
\end{tabular}
\caption{Readable and writable elements (belief states, agenda, plan) of the agents state by method, primitive task and trigger functions.}
\label{table:function_restrictions}
\end{table}
\paragraph{(1)}During robot planning, the decision and the action can depend on the beliefs of the robot and on the planned estimated beliefs of the human. Moreover, the current partial plan of the robot and the anticipated plan of human one can also be used to make decisions.

\paragraph{(2)} The effects of robot actions obviously impact its own belief state (considered as the real world state by the planner), but also the beliefs of the human, for example, through their observation process and first order logic reasoning. More elaborate schemes to compute the effects can also be devised such as those described in~\cite{gharbi2015combining}.

\paragraph{(3)} Besides, a robot action can add a new task to the agenda of the human. This is to account for communication actions requesting the human to do something.

\paragraph{(4)} The human decisions and actions can only be done according to her own beliefs and partial plan. Indeed, we cannot add the robot ones as it is, or we would consider that the human estimation of the robot knowledge and past actions is always perfect. This would require a third type of agents, being the robot model as estimated by our estimation of the human. Here, we make the assumption that the human is a naive user, and thus, will not take their decision based on the estimated robot beliefs and past plan.

\paragraph{(5)}The effects of the human actions obviously impact their beliefs and the robot (planner) ones. Moreover, the human agenda could also be updated through, for example, a positive answer of a task request.
%\end{itemize}


\subsection{Exploration algorithm}
Our planner operates in a turn-taking scheme, based on the update of the agents beliefs states, the HTNs of the robot and the human are explored successively.

\begin{algorithm}[H]
\begin{algorithmic}[1]
\Function {SeekPlans}{$robot$, $human$}
\State $solutions \leftarrow$ an empty list of plans
\State $result \leftarrow \textsc{ExploreTree}(robot, human, solutions)$
\If {$result =$ failure} \Return failure \EndIf
\State \Return $solutions$
\EndFunction
\Statex
\Function {ExploreTree}{$r$, $h$, $solutions$}
\If {$\textsc{isEmpty}(\agenda_r)$}
	\State add the plan $\plan_r \cup \plan_h$ in $solutions$
	\State \Return success
\EndIf
\State $\lambda \leftarrow \textsc{Pop}(\agenda_r)$
\If{$\lambda \in \abstracttasks_r$}
	\State $isOneValid \leftarrow$ false
	\ForEach{$\langle \alpha, \delta \rangle$}{$\methods_r$ s.t. $\alpha = \lambda$}
		\State $decomposition \leftarrow \delta(\worldstate_r, \plan_r, \agenda_r, \worldstate_h, \plan_h, \agenda_h)$
		\If {$decomposition \neq \bot$}
			\State $r', h' \leftarrow \textsc{Copy}(r, h)$
			\State $\agenda_{r'} \leftarrow decomposition.\agenda_{r'}$
			\State $result \leftarrow \textsc{ExploreTree}(r', h', solutions)$
			\If {$result =$ success}
				$isOneValid \leftarrow$ true
			\EndIf
		\EndIf
	\EndFor
	\If {$isOneValid$} \Return success \EndIf
\EndIf
\If{$\lambda \in \operators_r$}
	\State $result \leftarrow \lambda(\worldstate_r, \plan_r, \agenda_r, \worldstate_h, \plan_h, \agenda_h)$
	\If {$result = \bot$}
		\Return failure
	\EndIf
	\State $r', h' \leftarrow \textsc{Copy}(r, h)$
	\State $\worldstate_{r'}, \agenda_{r'}, \worldstate_{h'}, \agenda_{h'} \leftarrow \textsc{Apply}(result)$
	\State $\plan_{r'} \leftarrow \plan_{r'}.\lambda$
	\State $\agenda_{r'}, \agenda_{h'} \leftarrow \textsc{ApplyTriggers}(\worldstate_{r'}, \plan_{r'}, \agenda_{r'}, \worldstate_{h'}, \plan_{h'}, \agenda_{h'})$
	\State $humanApplicableOperators \leftarrow \textsc{GetHumanApplicableOperators}(h')$
	\State $isOneValid \leftarrow$ false
	\ForEach{$o$}{$humanApplicableOperators$}
		\State $r'', h'' \leftarrow \textsc{Copy}(r', h')$
		\State $\worldstate_{r''}, \worldstate_{h''}, \agenda_{h''} \leftarrow \textsc{Apply}(o)$
		\State $\agenda_{r''}, \agenda_{h''} \leftarrow \textsc{ApplyTriggers}(\worldstate_{r''}, \plan_{r''}, \agenda_{r''}, \worldstate_{h''}, \plan_{h''}, \agenda_{h''})$
		\State $result \leftarrow \textsc{ExploreTree}(r'', h'', solutions)$
		\If {$result =$ success} $isOneValid \leftarrow$ true \EndIf
	\EndFor
	\If {$isOneValid$} \Return success \EndIf
\EndIf
\EndFunction
\end{algorithmic}
 \caption{Double HTN main exploration algorithm.}
  \label{alg:seek_plans}
\end{algorithm}

\begin{algorithm}[H]
\begin{algorithmic}[1]
\Function {GetHumanApplicableOperators}{$h$}
\State $solution \leftarrow ExploreApplicableOps(h)$
\If {$solution = ()$}
	\Return $(WAIT)$
\EndIf
\EndFunction
\Statex
\Function{ExploreApplicableOps}{$h$}
\If{$\textsc{isEmpty}(d_h)$}
	\Return $(IDLE)$
\EndIf
\State $\lambda \leftarrow \textsc{Pop}(\agenda_h)$
\If {$\lambda \in \abstracttasks_h$}
	\State $applicableOps \leftarrow$ an empty set of operators
	\ForEach{$\langle \alpha, \delta \rangle$}{$\methods_h$ s.t. $\alpha = \lambda$}
		\State $decomposition \leftarrow \delta(\worldstate_h, \plan_h)$
		\If {$decomposition \neq \bot$}
			\State $h' \leftarrow \textsc{Copy}(h)$
			\State $\agenda_{h'} \leftarrow decomposition.\agenda_{h'}$
			\State $applicableOps \leftarrow applicableOps \cup ExploreApplicableOps(h')$
		\EndIf
	\EndFor
	\State \Return $applicableOps$
\EndIf

\If{$\lambda \in \operators_h$}
	\If{$\lambda(\worldstate_h, \plan_h, \agenda_h) \neq \bot$}
		\State \Return $\{\lambda\}$
	\Else
		\State \Return an empty set
	\EndIf
\EndIf
\EndFunction
	
\end{algorithmic}
 \caption{Human HTN exploration algorithm, returning the feasible human actions}
 \label{alg:gethactions}
\end{algorithm}

\subsubsection{Controllable agent HTN exploration}
The robot HTN exploration is a pretty standard depth first algorithm. The first task $\lambda$ from its agenda $\agenda_{robot}$ its popped, then if it is an abstract task $\lambda \in \abstracttasks$, all the applicable methods are applied, and their result are prepended to the agenda, thus giving new agents states (with the same beliefs as the previous ones but with the robot agenda updated) and branching our search space. We iterate with the new task popped from the new robot agenda. Eventually, the popped task will be a primitive one $\lambda \in \operators$, its function will then be applied to the currently explored agent states. If it returns \textit{false}($\bot$), the action is not applicable, and the exploration backtracks to another decomposition of an abstract task. However, if the action is applicable (returns a new agents state), the triggers are run for each agent, updating their agenda if necessary. Then, we question the human HTN to get their possible next actions from this new agents state, and, for each possible new agents state, we apply the triggers of each agent then we continue the robot HTN exploration. This exploration continues until the robot agenda is empty, or all the branches return \textit{false}.

\subsubsection{Uncontrollable agent HTN exploration}
The human HTN exploration differs from classical HTN planner as the goal is not to produce a complete plan, but rather to list all the actions the human is likely to perform in a given agents state. To do so, we recursively decompose the first task of the human agenda $\agenda_{human}$ with every applicable methods, until we reach an applicable operator. All the operators from all the aaplicable decompositions are return to the robot HTN exploration, and applied.

\subsubsection{Default actions} Two special cases are handled during the exploration. If the human agenda is empty whereas the robot one is not, the exploration returns a default action \textit{IDLE} --- which does not modify agents beliefs nor agendas --- for the human. This action represents the non-involvement of the human in a task. Besides, if for the human no applicable action is found a default action \textit{WAIT} --- which does not modify agents nor agendas --- is returned. This action represents the impossibility of the human to act in the current situation, making them wait for the robot to proceed.

Once the robot agenda is emptied, the agents state is set as a success, the plan is added to the policy tree and the search can be continued until no decomposition is left for any task.


\subsection{Conditional plan selection}
Once this exhaustive search has been done, the result is a search tree of alternating robot and human feasible actions leading to a task completion. However, we still have to select the action the robot has to perform at each step (if several of them are applicable). To do so, we define a cost function $cost: \agentsstates \times \operators \mapsto \realset^+$ representing the cost of an action in a specific state. The data structure is now similar to a two players game tree. However, \textit{MinMax} approaches are not suitable here, as we are not in an adversarial setup but more into a collaborative one. Indeed, trying to minimize the maximal possible cost is assuming that the human will always do the actions leading to the worst plan. This defensive behavior could lead to non optimal plans. We assume that given the right indications, the human will do their best to achieve the task with minimal cost. We thus propose to explore this tree differently.
\improvement{also set plan wide (social) costs}

\subsubsection{Minimizing the average cost}
\improvement{TODO !!}
The first approach we propose for plan selection is to minimize the average cost.

\begin{algorithm}[H]
\begin{algorithmic}[1]
\Function{SelectRobotActions}{$agentsstate$, $actualcost$}
\If{$\textsc{IsGoalState}(agentsstate)$}
	\State \Return $actualcost$
\EndIf
\If{$\textsc{NextActionsAgent}(agentsstate) =$ human}
	\State $totalCost \leftarrow 0$
	\ForEach{$action$}{$NextActions(agentsstate)$}
		\State $totalCost \leftarrow \textsc{SelectRobotActions}(\textsc{Apply}(action, agentsstate), actualcost + \textsc{Cost}(action))$
	\EndFor
	\Return $totalCost / \textsc{Card}(NextActions(agentsstate))$
\Else
	\State $minCost \leftarrow +\infty$
	\State $chosenAction \leftarrow$ null
	\ForEach{$action$}{$NextActions(agentsstate)$}
		\State $actionCost \leftarrow \textsc{SelectRobotActions}(\textsc{Apply}(action, agentsstate), actualcost + \textsc{Cost}(action))$
		\If{$actionCost < minCost$}
			\State $minCost \leftarrow actionCost$
			\State $chosenAction \leftarrow action$
		\EndIf
	\EndFor
	\State choose $action$ as the robot plan
	\Return $minCost$	
\EndIf
\EndFunction
	
\end{algorithmic}
 \caption{Conditional plan selection algorithm. Explores a search space (a bipartite tree of alternating robot and human actions) to choose the robot actions minimizing the average of the total plan cost over all the possible human actions.}
 \label{alg:minaverage}
\end{algorithm}

\improvement{add social cost?}

\subsubsection{The human as a limited depth planner}
\unsure{Not implemented, just the idea, maybe not enough time}

\subsubsection{Guiding human choices towards the least costly solution}
\improvement{TODO !!}
\unsure{Even blurrier idea}

\section{Implementation}
The previous section presented the general ideas and concepts behind this new planning paradigm. This short section shows some interesting details about the actual implementation of a prototype of the planner, and how it has been integrated with other components to extend its capabilities and be used in a real robotic architecture.

\subsection{A Python planner}
We chose to implement the prototype of the planner in Python. It is originally based upon the \textit{Python Hierarchical Ordered Planner} (PyHOP) from Nau but as been largely modified, and only remains the general data structures. As in PyHOP, it allows to represents world states as Python objects having dictionary as attributes. For example \verb|s.isReachableBy["cube_23"] = ["human_3", "pr2_robot"]| specifies that the \textit{cube\_23} is reachable by both the \textit{human\_3} and the \textit{pr2\_robot}.
Moreover, like in PyHOP the planning domains (HTNs for both the robot and the human) are written using plain Python functions. While using a separate domain specific language (DSL) for writing planning domains enables some optimizations and advanced search algorithms, it tremendously reduces the expressiveness, and makes representing real worlds scenarios them more complex. Besides, Python being an interpreted language, iterating over these domains is quicker as, unlike HATP, they do not require any compiling in order to be used.
The decomposition functions take the world states (beliefs), partial plans of the agents (depending on the type of the agent~Table~\ref{table:function_restrictions}) and any other optional parameters (\textit{e.g.} goals, entities) as arguments, and return a list of tasks with optional parameters to be put in the agent's agenda. Likewise, the actions expects the world states, the partial plans and the agendas of the agents (also depending on the type of the agent~Table~\ref{table:function_restrictions}) and return new states and agendas.

The search algorithms have also been implemented in Python. A lot of optimization can be done, and it is planned to entirely rewrite the software in C++. The current prototype, while not being efficient compared to others approaches, still allows to find plans in a reasonable time for realistic human robot scenarios.

\subsection{Integrating with other components}
\label{subsec:chap4integratingwithothers}
\subsubsection{Retrieving the current state and beliefs from the knowledge base}
Before any planning process could start, the planner must be initialized with the current world state (robot beliefs) and the estimation of the human's beliefs. To do so, we use the knowledge base presented in the previous chapter: the ontologies. In our architecture, each human the robot is interacting with has a dedicated ontology in addition to the robot own ontology, representing the world state. Thus, each agent beliefs $\agentstate$ is initialized with the facts in its respective ontology.
However, ontologies can contain a lot of information that are not needed for planning, and worse, that can hamper keeping world state consistency if the planning domains do not consider some of these facts (\textit{e.g.} an action deleting a fact but not deleting the inverse relation). To cope with this issue, we define two special attributes in the world state objects: the \textit{types} and the \textit{individuals} attributes.

The \textit{types} attribute must be set by the user before retrieving the current world state. It is a dictionary linking a type name to a list of property names. A world state must be initialized solely with this attributes before being passed to a function retrieving the world state from the ontology. This function takes the agent name as the argument, and will fill the world state with the ontology matching to this name. This function fills the \textit{individuals} attribute of the world state with a dictionary linking the types specified as keys of the  \textit{types} dictionary to a list of entities/individuals inheriting from these types. Then, a world state attribute is created for each property defined in the \textit{type} attribute, and filled with a map linking a subject entity to existing entities list according to the relations in the ontology. Communication with the ontology is done using the Ontologenius~\cite{sarthou2019ontologenius} python API.
\improvement{Write the algorithm ?}

Besides reading an initial state from the knowledge base, the prototype planner is also able to write in it. Indeed, HTNs can not only be used as operational models for planning, but also can be used as a semantic source making verbal communications able to use past plans. To do so however, the knowledge base must be informed with the decompositions of task and with the parameters and their types they require. Our prototype is able to parse a planning domain in order to extract the required information and write them in an ontology friendly format.

\subsubsection{Using REG at planning time}
In the previous chapter, we presented how we integrated referring expression generation algorithm during task planning to precisely evaluate communication feasibility and cost. We showed how we successfully integrated this approach in HATP. However, because of the HATP architecture, its use was restrained to compute only action feasability and cost when an action was explored by the planner.
With this new planning scheme, we are able to make a REG request and fully use the returned RE at any time in action or decomposition functions (as they are Python functions). For example, in a decomposition, we can request a REG for multiple entities and return sub-tasks concerning only the least costly one. 
Besides, we can also use the REG failure information to know which entities prevent another from being referred and select decomposition accordingly. For example, if an entity is not distinguishable from another through the REG, we can try to rely on an other communication mean, such as pointing, or to move the distractor entity to remove it from the context entities.

\subsubsection{Communicating through ROS}
Finally, even if the planning process can be initialized and started through a Python script, it is not enough to make it useful in a real robotic architecture. Thus, we integrated planning requests support via the ROS framework. 
Once the planning service is launched, the specified domains are loaded for controllable and uncontrollable agents, then a service server is started, waiting for planning requests. 
The requests need to be filled with the controllable and uncontrollable agents name, used to retrieve their belief and to run REG on their ontologies if needed. 
They also need to be given the tasks list to decompose along with their parameters for both the controllable and the uncontrollable agents. While the tasks for the controllable agent are simply the tasks a classical HTN planner would have to decompose, the uncontrollable agent's ones represent the tasks that the robot estimates the human is performing. Such information can be given via actions and intentions recognition components. The parameters of the task can be given as strings if they are simple enough (\textit{e.g.} entity names, agent names) but they also can be formatted through JSON, allowing for more complex parameters (\textit{e.g.} goals --- which are represented as partial world state). The complex parameters are then deserialized into corresponding Python objects.
Once a request is received the planning process can start, and, once a conditional plan has been elaborated is sent back as an response to the request. The response is constituted of a list of tasks each having an unique id, a list of their parameters, the name of the agent executing it, its type (either abstract or primitive/action), an id of the previous task (if any), an id of the task it decomposes from (if any), a list of id of the following tasks (if any) and a list of id of the task it decomposes into (for abstract task only). Using this information a component can easily reconstruct the conditional plan computed.

\unsure{Speak about the (not implemented) anytime ? Replan ?}

\unsure{Speak about the GUI ?}


\section{Examples}
In this section, we will present some case studies in which are presented small task examples. Each example will present some features of the planner and comparisons between multiple plans depending on some initial conditions (beliefs or action costs). Besides, they will give insights into the rationale used when creating the domains for both the robot and the human.

The two following cases are set in the same context. We envision a super-scenario in a company office where a robot assistant is verbally commanded by a human worker to bring her a coffee. The robot must take her mug, go to the coffee machine, manage to fill the mug with coffee and bring back the filled mug to the worker.
We decline this scenario into three precise subtasks highlighting multiple features of the presented planner.
% Triggers vs. explicit communication
%Human asks for the robot to get a coffee. Two mugs nearby, two possible decomposition for the robot : ask for which one to take or take one. One adds "answers" to the human -> two possibilities 
\subsection{Plan for robot unknown human knowledge}
\improvement{3d illustration ?}
First of all, after the robot has been commanded by the worker to bring her a coffee, it must pick her mug. We want to illustrate how we can represent human knowledge that is unknown to the robot (but with a small number of possibilities), and how the planner can elaborate different plans depending on action costs and the number of possibilities. To do so, we place the robot in a scenario where the worker and it are face to face when she asks it for a coffee. There is a table between them and $n \in \intset$ mugs are placed on it. Only one mug belongs to the worker, the robot knows she knows which one it is ($\humanmodel$) but the robot does not know it ($\robotmodel$). The goal of the robot is thus to take the right mug, and to go to the coffee machine. The general idea is that we want the robot to have two ways of grabbing the right mug. Either it can take a random one, and if the human is not protesting, the robot can proceed with the rest of the task, else it tries again while knowing this mug was not the right one; or the robot can ask the human for the right mug and take it. However, as the specifying the right mug through verbal communication can be costly for the human (\textit{e.g.} highly resembling mugs, a noisy environment), asking her might not always be feasible not be the optimal solution.

First, we go through the design of the robot HTN. The robot agenda is initialized with two tasks \verb'get_right_mug' and \verb'go_to_coffee_machine'. The robot abstract tasks and their decompositions are presented here:
\begin{itemize}
\item \verb'get_right_mug' aims at making the robot pick the mug belonging to the human. It has two decompositions:
	\begin{itemize}
	\item \verb'robot_ask_and_take_mug' representing the robot asking the human which one is her mug. It returns either \verb'pick_mug' if the human's mug is known or \verb'ask_mug_to_take' and \verb'get_right_mug'
	\item \verb'take_one_random_mug' representing the robot go through trial and errors. It returns either \verb'pick_mug' if the human's mug is known or \verb'pick_mug' with all the mugs potentially belonging to the human.
	\end{itemize}
\end{itemize}
The primitive tasks of the robot and their effects are presented here:
\begin{itemize}
\item \verb'pick_mug' updating the beliefs of all the agents in the room by removing the mug given as parameter from the table and adding it to the robot gripper.
\item \verb'ask_mug_to_take' adding the task \verb'answer_right_mug_a' to the human agenda.
\item \verb'drop_mug' updating the beliefs of all the agents in the room by removing the mug they believe the robot is currently holding and adding it on top of the table.
\item \verb'go_to_coffee_machine' updating the beliefs of all agents in the room that the robot has left the room and is now in the coffee room. Updates all agents in the coffee room that the robot is in the coffee room and updates their beliefs of what it is carrying.
\end{itemize}
Moreover, if the human is complaining about the mug the robot is currently holding, we want the robot to drop it, and to know that this mug is not the right one. To do so, we use the triggers mechanisms and we define a trigger function for the robot:
\begin{itemize}
\item \verb'drop_wrong_mug' checking if the human just complained about the mug we taken (\verb'complain_mug'~$\in \plan_h$) and if so, adds \verb'drop_mug' and \verb'get_right_mug' to the robot agenda.
\end{itemize}

Then, we go through the human task model. We assume the order of getting a coffee has already been given to the robot, and thus assume the human has no task to decompose initially in her agenda. When asked for which mug belongs to her, we model that she can answer whatever mug she has not ruled out. Moreover, we model that when the robot pick a mug, she will complain if it is not the right one. Here are the two abstract tasks and their decompositions we used to model this human behavior:
\begin{itemize}
\item \verb'answer_right_mug_a' representing the task of answering the robot for the right mug. For this example, it has only one decomposition:
	\begin{itemize}
	\item \verb'answer_right_mug' if the right mug is present in the human beliefs, it returns only the \verb'verbally_answer_right_mug' primitive task, else, it returns $m \in \intset$ alternatives of \verb'verbally_answer_right_mug' with $m$ being the number of mug not having been ruled out in the plan at this state.
	\end{itemize}
\item \verb'check_mug_taken' models the human expectation of the robot taking the right mug. It has two decompositions:
	\begin{itemize}
	\item \verb'agree_mug_taken' when the robot has the right mug in its gripper. It always returns an empty task list.
	\item \verb'complain_mug_taken' when the robot has a wrong mug in its gripper. It return the primitive task \verb'complain_mug'.
	\end{itemize}
\end{itemize}
The following presents the modeled human primitive tasks:
\begin{itemize}
\item \verb'verbally_answer_right_mug' updating the beliefs of all the agents in the room with the human being the owner of the mug passed as parameter. To estimate the feasibility and the cost of this action we run our REG component as detailed in the previous chapter. 
\item \verb'complain_mug' updating the beliefs of all the agents in the room with the human not being the owner of the mug passed as parameter.
\end{itemize}
Finally, to model the human reaction when the robot grab the wrong mug, we use the triggers system for the human:
\begin{itemize}
\item \verb'check_mug' adding the task \verb'check_mug_taken' to the human agenda each time the robot takes a mug that has not been specially designated by the human (\textit{i.e.} the robot has a mug in its gripper and \verb'verbally_answer_right_mug'~$\notin \plan_h$).
\end{itemize}

\todo{caption}
\begin{figure}[hbtp]
\centering
\includegraphics[width=\textwidth]{figures/chapter4/mug_selection_search_space.png}
\caption{CAPTION TODO}
\label{fig:chap4mugsss}
\end{figure}
\improvement{where can I introduce how to read this graph?}

The search space for $n=2$ mugs is presented in Figure~\ref{fig:chap4mugsss}. In this example, both mugs are distinct and RE can be computed. On the right hand side of the figure is the decomposition where the robot explicitly asks the human to designate her mug. The answer can either be \verb'mug_0' or \verb'mug_1'. The robot then pick the right one and go to the coffee machine, leaving no task to decompose.
On the left hand side of the figure is the decomposition where the robot proceeds via trials and errors. The robot can either pick \verb'mug_0' or \verb'mug_1' and the human will either react by doing nothing (if the robot took the right mug) or by complaining, in which case the robot will drop the mug, take the other one, and leave. Interestingly, we model the human reaction such as not expecting her to complain when taking the second mug after a first failure.

Next, we will compare different action costs and conditional plan selection criteria based on the same search space. To select a plan we used the Algorithm~\ref{alg:minaverage}. First, we set the cost of \verb'complain_mug' action much lower than the \verb'verbally_answer_right_mug' action. Here, the costs might have been set by the supervision component, estimating we are interacting in a noisy environment, where verbal communications are difficult to make, or that the human does not bother to correct the robot. The conditional plan returned is presented in Figure~\ref{fig:chap4mugtrialerror}. The chosen plan is the one containing trials and errors. Indeed, as it can lead to much shorter and thus less costly plans, it is the minimum average. 
\todo{caption}
\begin{figure}[hbtp]
\centering
\includegraphics[width=0.8\textwidth]{figures/chapter4/mug_selection_trials.png}
\caption{CAPTION TODO}
\label{fig:chap4mugtrialerror}
\end{figure}

However, imagine that the human (who has still not had her coffee) is in a hurry, or that the mugs are really easy to distinguish from one another (\textit{e.g.} different color) and thus, we decrease the cost of the \verb'verbally_answer_right_mug' action and increase the cost of the \verb'complain_mug' action. The conditional plan selected is presented in Figure~\ref{fig:chap4mugask}. The robot know, prefers to ask for the right mug rather than trying to pick one at random.
\todo{caption}
\begin{figure}[hbtp]
\centering
\includegraphics[width=\textwidth]{figures/chapter4/mug_selection_ask.png}
\caption{CAPTION TODO}
\label{fig:chap4mugask}
\end{figure}

As we increase the number of mugs $n$, the cost of \verb'verbally_answer_right_mug' has to also increase to make the robot choose the trials and errors decomposition, as the average of this decomposition increases since the number of potential errors increases. 
%Finally, if we make two mugs for which the REG does not find any solution (the mugs are not distinguishable from one another) the planner is able to generate conditional plans where it first tries to <-- not true for now, as we do not take into account failed actions in cost computation....

\improvement{Adds compute time for n=2,3,4,5}

\improvement{Add link to domain}

In this example we show one really interesting feature of our planner: representing human knowledge that is not known by the robot. While not being tractable when there are a lot of possibilities, it allows to select more or less conservative conditional plans depending on the cost of each actions. Moreover, this example allowed to see how updating the human agenda and how triggers can be used to model the agents interaction in the HTNs planning.
However, the human model was pretty simple, and we propose to challenge our planner in the next example with a task where the human is more involved.

\subsection{Balance difficult communications, decomposition cost and task attribution}
The robot is now heading to the coffee machine with the right mug in its gripper. On its way it detects another human taking a break near the coffee machine. The coffee has to be made. To brew coffee, ground coffee and water must be put in the coffee machine, and then the coffee can be served. While water is considered as always available, ground coffee is not. There are two places where ground coffee can be retrieved: either in the kitchen cupboard (close to the coffee machine) or in the pantry cupboard. 

\subsubsection{Handling the robot only case}
\label{subsubsec:chap4coffeerobotonly}
 First, we want the robot to be able to make coffee by itself, without requiring human help. To do so, we implement the following abstract tasks tree in the robot model (here we prepend the task names with \verb'r' where task are different in the robot and the human model):

\begin{itemize}
\item \verb'r_make_coffee' only having one decomposition (for now):
	\begin{itemize}
	\item \verb'r_make_coffee_alone' returning, in both orders (to represent partially ordered task tree), the tasks \verb'get_water', \verb'pour_water_in_machine' and \verb'r_get_coffee', \verb'put_coffee_in_machine'. Only \verb'r_get_coffee' is an abstract task.
	\end{itemize}
\item \verb'r_get_coffee' representing the ways for the robot to obtain coffee. It has only one decomposition:
	\begin{itemize}
	\item the decomposition returns $()$ if the robot has already coffee in its gripper. Else, it selects the closest cupboard and returns \verb'r_pick_coffee' with it as parameter.
	\end{itemize}
\end{itemize}

The robot primitive tasks as are follow:
\begin{itemize}
\item \verb'get_water' returning $\bot$ if the robot is already holding something; updating the beliefs of all the agents in the room with the fact that the robot holds water otherwise.
\item \verb'pour_water_in_machine' updating all the agents in the room beliefs with the machine being filled with water.
\item \verb'r_pick_coffee' returning $\bot$ if the robot is already holding something or if the cupboard passed as parameter does not contains coffee (in the robot beliefs); updating the beliefs of all agents in the room with the fact the the robot holds coffee otherwise.
\item \verb'put_coffee_in_machine' updating all the agents in the room beliefs with the machine being filled with coffee.
\item \verb'r_serve_coffee' updating all the agents in the room with the mug being filled with coffee.
\end{itemize}

Now, for the initial conditions we set that the robot knows there is coffee in the kitchen cupboard (the closest) and we add two tasks in its agenda: \verb'r_make_coffee' and \verb'r_serve_coffee'. The human has nothing in its agenda. The two possible plans for this really simple case are presented in Figure~\ref{fig:chap4coffeesimple}(a) and (b). The plan selection would then choose one of the plan based on robot action costs.

\todo{caption}
\begin{figure}[hbtp]
\centering
\includegraphics[width=\textwidth]{figures/chapter4/Chap4CoffeeSimplePlan.png}
\caption{CAPTION TODO}
\label{fig:chap4coffeesimple}
\end{figure}

\subsubsection{Incorporating the human planning process}
As we also want the robot to be able to ask the idling human to help it, we add, to its action model, a decomposition to the abstract task \verb'r_make_coffee' and a new abstract task \verb'help_make_coffee':

\begin{itemize}
\item \verb'r_make_coffee' containing the previous decomposition and the new one:
	\begin{itemize}
	\item \verb'r_make_coffee_collaboratively' returning the primitive task \verb'r_ask_human_for_help' and the abstract task \verb'r_help_make_coffee'
	\end{itemize}
\item \verb'help_make_coffee' representing the ways for the robot to help another agent to make coffee. It has only one decomposition:
	\begin{itemize}
	\item It returns \verb'get_water' and\verb'pour_water_in_machine' if there is no water in the machine and the human is doing a task related to bringing coffee. Likewise, it returns \verb'r_get_coffee' and \verb'put_coffee_in_machine' if there is no coffee in the machine and the human is doing a task related to fill the machine with water. Then, if the human is not doing any task, we add to the exploration \verb'r_get_coffee', \verb'put_coffee_in_machine' and \verb'help_make_coffee' if there is no coffee in the machine and \verb'get_water', \verb'pour_water_in_machine' and \verb'help_make_coffee'. The idea here is to complete the human actions if they take the initiative of a task, but to be proactive by exploring both possible alternatives if they are not. The recursion allows to reevaluate the need of this task later in the planning process.
	\end{itemize}
\end{itemize}

The primitive action added to the robot model is:
\begin{itemize}
\item \verb'r_ask_human_for_help' adding the task \verb'help_make_coffee' to the human agenda. Here we could have represented the possible refusal of the human by adding an abstract task leading to two possible decomposition for the human, accepting or declining, leading in similar schemes as in \ref{subsubsec:chap4coffeerobotonly}. However, to keep this example as simple as possible, we assume the human will always help the robot if asked to do so.
\end{itemize}

We model the human actions similarly to the robot ones. Their primitive tasks are defined as:
\begin{itemize}
\item \verb'help_make_coffee' representing the ways for the human to help another agent to make coffee. It has only one decomposition, which is the same as the robot one.
\item \verb'h_get_coffee' representing the ways for the human to obtain coffee. It has only one decomposition:
	\begin{itemize}
	\item the decomposition returns $()$ if the human is already holding coffee. Else, it selects the closest cupboard and returns \verb'h_try_pick_coffee' with it as parameter and \verb'h_get_coffee'. It differs from the robot one, indeed, whereas the knowledge of the robot is assumed to be the world state, the human's one can be false. Thus, the human might try to perform \verb'h_try_pick_coffee' on a cupboard not containing coffee. We take this into account with the recursion of this abstract task, and with the primitive task \verb'h_try_pick_coffee' described hereafter.
	\end{itemize}
\end{itemize}

The model of the human primitive actions are:
\begin{itemize}
\item \verb'get_water' as defined for the robot
\item \verb'pour_water_in_machine' as defined for the robot
\item \verb'h_try_pick_coffee' differs from the one defined for the robot as it checks if the cupboard passed as parameter really contains coffee (\textit{i.e.} in the robot beliefs). If it does not, the human's beliefs about this cupboard are updated to match the robot ones (modeling the human going in front of the cupboard, opening it and seeing the absence of coffee). If the cupboard does contain coffee in the robot beliefs, all the agents in the room beliefs are updated with the human having coffee in their hand.
\item \verb'put_coffee_in_machine' as defined for the robot.
\end{itemize}

The initial conditions are the same as presented before, but we also add that the kitchen cupboard contains coffee in the human beliefs. In addition to the two plans where the robot does not seek help to the human, another valid plan is found. This plan is presented in Figure~\ref{fig:chap4coffeesimple}(c). This conditional plan has two alternatives, depending on the initiative taken by the human.

\subsubsection{Updating human beliefs}
We can also change the initial conditions to elicit new behaviors. We keep the same action models for both the robot and the human, but we change the estimation of the human beliefs given as initial conditions to the planner. In a real robotic architecture, the human knowledge base would be updated with an estimation provided by situation assessment components. We specify that the human believes that both the kitchen and the pantry cupboard contain coffee. However, the robot knows (\textit{e.g.} using specific sensors or having been told about) that there is coffee only in the pantry cupboard. With these conditions, the search space extends to the three plans presented in Figure~\ref{fig:chap4coffeesimple}(a), (b), being when the robot prepare the coffee by itself, and in Figure~\ref{fig:chap4beliefsdiv}(a). In this last plan, we indeed model that the human will tend to first go to the nearest cupboard he thinks contains coffee. If this cupboard does not contain coffee, he will go to the next one. We can also note that only the left branch of the plan in Figure~\ref{fig:chap4beliefsdiv}(a) is impacted by this beliefs divergence. However, this branch choice is not up to the robot without any communication as we modeled the human as having the initiative of selecting a task. This subtlety cannot be represented in HATP.

Depending on the cost of the human being deceived and of the actions, the plan selected can be that the robot does all the task, as the human "mistake" can increase too much the average cost.

\todo{caption}
\begin{figure}[hbtp]
\centering
\includegraphics[width=\textwidth]{figures/chapter4/Chap4CoffeeBeliefDiv.png}
\caption{CAPTION TODO}
\label{fig:chap4beliefsdiv}
\end{figure}

To improve our robot, we want to make it able to realign the beliefs of the human, so, whatever the task he chooses, he will not make a mistake. To do so, we add a third decomposition to the \verb'r_make_coffee' abstract task and one new primitive task to the robot models.
\begin{itemize}
\item \verb'r_make_coffee' containing the previous two decompositions and the new one:
	\begin{itemize}
	\item \verb'r_align_and_make_coffee_collaboratively' returning $\bot$ if no belief divergence is detected between the robot and the human. The decomposition returns the new primitive task \verb'r_update_human_inventory' along with \verb'r_ask_human_for_help' and \verb'r_help_make_coffee'. An alternative for \verb'r_update_human_inventory' is returned with as parameter each cupboard in diverging beliefs between the robot and the human.
	\end{itemize}
\item \verb'r_update_human_inventory' being a primitive task. It updates the human beliefs concerning the cupboard passed as parameter with the beliefs of the robot.
\end{itemize}

With this new decomposition the new plan presented in Figure~\ref{fig:chap4beliefsdiv}(b) is added to the search space. In this plan the human beliefs are updated before asking him to help the robot to make coffee. The human does not make the mistake of going first to the kitchen cupboard.

Depending on the communication cost (estimated using the REG approach presented in the previous chapter), the human deception cost and the other actions costs any one of the four possible plans can be selected. For example, to minimize the human involvement and if the communication has a high cost, the selected plan would be Figure~\ref{fig:chap4coffeesimple}(a) or (b). If the communication is costly but the pantry and kitchen cupboard are not too far away, the selected plan is Figure~\ref{fig:chap4beliefsdiv}(a), finally, if we represent that the human would be upset if he makes a mistake or if the communication for aligning beliefs is not expensive, the plan Figure~\ref{fig:chap4beliefsdiv}(b) would be returned.

Through all these examples we show that this task planning approach, which separates human and robot beliefs and action models, can be suitable for multiple problems. We are able to plan for robot unknown human beliefs, to rely on the human planning process while keeping inherent uncertainties (\textit{i.e.} not making choices for them, without communicating them) and also to plan diverging beliefs and balance the actions of realigning them with plans containing mistakes. In the next section, we present a HRI task, inspired from psychology, that has never been tackled in robotics and we show how our planner is integrated in a fully functional robotic architecture dedicated for this task.


\section{The Director Task}
This work has been made in close collaboration with two other PhD. students Amandine Mayima and Guillaume Sarthou. 
\unsure{Dire qu'il y a un papier ?}

\subsection{A task used in psychology}
The director task is an experiment setup largely used and derived in psychology. It places two agents, the director and the receiver, in front of each other with a shelf in between. Usually, only the receiver is the participant, and the director is either an accomplice or a remote controlled agent (for computer-based experiment). The shelf consists in several compartments possibly containing objects. Each compartment can either be open on both opposite faces or open on only the receiver's side (hiding any contained object from the director, and being obvious for the receiver that the director cannot see inside it).

\todo{caption}
\begin{figure}[hbtp]
\centering
\includegraphics[width=0.5\textwidth]{figures/chapter4/dt_apple.png}
\caption{CAPTION TODO}
\label{fig:chap4dtapple}
\end{figure}

The director asks the receiver to move some objects by describing them. However, some descriptions of the director will also match other objects ---called competitors--- that are only visible by the receiver (Figure~\ref{fig:chap4dtapple}). Hence, the receiver must think that the object matching the description cannot be the one referred by the director as they are not aware of it, and find the object matching the description in the director's beliefs. This process must then be maintained all along the interaction as the situation evolves.

This task is used in psychology to study how perspective-taking is used for communication understanding while performing a task with another agent \cite{keysar2000taking}. Results show that, even if the receiver considered or took a competitor for the first trial, they are able to take the one designated by the director during subsequent trials. This shows that even if participants understand language in an egocentric way, they are able to do perspective-taking to successfully perform the task \cite{keysar2003limits}. 

Even if this task is well known by psychologists, to our knowledge, no robot managed to handle it. A robot that does would prove its architecture being able to maintain self-other distinction of beliefs but also to perform perspective taking on its human partner. We propose a robotic architecture integrating our planner, that is able to handle both side of the director task. Besides, we propose some changes for both the director and receiver roles to be interesting along with planning challenges some modifications of the task can arise.

\subsection{Setup}
The setup we propose is a slight variation of the original director task used in psychology. First, instead of moving objects between compartments, the high level known goal is to remove a subset of the objects in the compartments and to place them in a receiver accessible only area (on top of a table). 
Then, objects are replaced with blocks having four special attributes: their color, the color of their border, the shape drawn on them and the color of this shape. The colors can either be blue or green, and the shapes either a triangle or a circle. This allows for a maximum of ambiguity between the blocks. 
Besides, to make the task interesting for both the director and receiver roles, compartment are not only either fully opened or hiding content from the director, they can also be hiding content from the receiver while being opened towards the director. Thus, not only the receiver must take the perspective of the director to understand the right block, but the director has also to take the perspective of the receiver to make the smallest instruction possible to respect the maxim of quantity~\cite{grice1975logic}.
In addition, to increase the number of ambiguous situations, we prohibit the use of geometrical relations during the communications (\textit{e.g.} "the leftmost block", "the block above the green one") to only allow the use of the block attributes. Likewise, pointing at a block is forbidden.
Finally, every blocks and compartments are equipped with AR-tags (different on each face) allowing the robot to easily detect them and to make an accurate representation of the environment.

\subsection{The robotic architecture}
The robotic architecture is composed of several elements.
First, the knowledge base chosen is an ontology. As we have shown before, ontologies are more and more common in robotics as they allow for rich and complete reasoning mechanisms and efficient requests of symbolic facts. The software used is \textit{Ontologenius}~\cite{sarthou2019ontologenius}. An ontology is created for the robot ($\robotmodel$, which we use as the true state of the environment) and another one is created for the human ($\humanmodel$) representing the robot estimation of the human knowledge. As stated in the previous chapter, these ontologies can be queried or updated through an efficient low-level API or higher-level \sparql{} queries. The interfaces between the ontologies and our planner are presented in \ref{subsec:chap4integratingwithothers}. The ontology is initialized with static facts being (1) the link between each AR-tag and their matching block and compartment, (2) the attributes of each block, (3) the 3D models of each block and compartment.

To gather all the robot sensing data, create a geometric representation of the environment, compute symbolic facts from it for both the robot and the human and update the ontologies with them, a situation assessment component is added to the architecture. This component is based on underwolds~\cite{lemaignan2018underworlds} allowing for modular and reusable reasoners. It first gathers the data from perception algorithm (AR-tags positions for the objects and motion capture for the human), then creates a geometrical scene of the environment. Based on it, the component is able to compute symbolic facts: \textit{isIn}, \textit{isVisibleBy}, \textit{isReachableBy}, \textit{isOnTopOf}; and feed them in real time to the ontology. This component also estimates the geometrical scene viewed and known by the human, computes symbolic facts on it and feed the corresponding ontology.

Then, to be able to give and understand instructions, the REG component presented in the previous chapter is also included. A grammar-based verbalization component allows to transform the generated RE into natural language. Similarly, a natural language request can be interpreted as a \sparql{} query and matched against an ontology.

To orchestrate all the components, the supervision system called JAHRVIS (Joint Action-based Human-aware supeRVISor) is dedicated to manage the interaction. It not only handles the robot actions but also estimates the human mental state, monitors the human actions and manages the communication with them. It manages five facets of the interaction: (1) interactions sessions, (2) communication, (3) human, (4) task and (5) quality of interaction. It is responsible for task planning requests and plan execution and monitoring.

Finally, the task planning component used in this architecture is the one presented in this chapter. Task planning is only required when the robot is the director, since when the robot is receiver it only has to execute commanded instructions. When the robot is the director, the supervision system is given a list of blocks (via their ids) to remove from the shelves. This list is then passed as the goal parameter of a task to decompose to the planner.
The task to decompose is called \verb'clear_blocks' and is filled with a parameter representing the goal and another being the human id. The goal is simply an under specified world state, composed of triplets containing for example \verb'(block_23, isIn, disposalArea)'.
The robot task planning domain ($\robotmodel$) is presented hereafter. 
\begin{itemize}
\item The \verb'clear_blocks' abstract task has only one (recursive) decomposition. The decomposition returns $()$ if the blocks specified in the goal matches the relations also specified in the goal. Otherwise, a REG request is performed for all the misplaced blocks, and the tasks \verb'clear_one_block' and \verb'clear_blocks' are returned, with the easiest block to refer to (\textit{i.e.} lowest RE cost) as parameter of the \verb'clear_one_block' task.
\item The \verb'clear_one_block' task has also only one decomposition returning the primitive task \verb'tell_human_to_clear_block' and the abstract one \verb'wait_for_human_to_clear_block'.
\item The \verb'wait_for_human_to_clear_block' abstract task has only one decomposition. It aims at recursively planning to wait until we planned the human has removed and put the right block away. It recursively decomposes into $()$ if the block passed as argument is in the right place; and returns \verb'wait' and \verb'wait_for_human_to_clear_block' otherwise.
\end{itemize}

The robot primitive tasks are defined as follow:
\begin{itemize}
\item The \verb'tell_human_to_clear_block' primitive task returns $\bot$ if the block specified as parameter is not reachable by the human, or if it is not present in the human beliefs $\humanmodel$. Else, it adds the abstract task \verb'clear_told_block' to the human agenda passing the specified block as parameter and does not update any beliefs.
\item The \verb'wait' primitive task does not update any beliefs.
\end{itemize}

On the human tasks side, we modeled a cooperative human, non declining asked tasks. Thus, the task model is as follows:
\begin{itemize}
\item The abstract task \verb'clear_told_block' has only one decomposition, returning the primitive tasks \verb'pick_block' and \verb'place_block'.
\item The primitive task \verb'pick_block' return $\bot$ if the block passed as parameter is not reachable by the human in their beliefs; and updates the beliefs of all the agents in the room with the human holding the block otherwise.
\item The primitive task \verb'place_block' return $\bot$ if the human is not carrying anything; else it updates the beliefs of all the agents in the room with the block being placed in the area specified as parameter, and the human not holding anything.
\end{itemize}

In the domain described previously, we compute in the abstract task the least costly block to refer to to decompose the task. This greedy approach can be used because we remove the block during the task. The referring cost of blocks thus can only decrease at each iteration of the task. However, it would be completely different if some blocks were to be added to the shelf along with others to remove. There, our greedy approach would not work as an not costly action would make future actions more costly leading to an sub-optimal plan. Another approach could be to have the decomposition of \verb'clear_blocks' to return all the orders of the block removing instructions to explore.

In the proposed version of the director task, only the cube order can be planned. Thus, in the following subsection, we propose slight variations of the task increasing the planning complexity and interest, along with some possible modeling directions to overcome these challenges.

\subsection{Challenges for planning}
To spice up the planning challenge in the director task, we propose some extensions to the presented task.
First, by adding blocks with identical visual features to the shelf, and asking to remove a precise one of them, we introduce situations where verbal communication (and REG) is not enough to refer to a block. To solve this problem, we could for example add a decomposition to \verb'clear_one_block'. In this decomposition, we could have the robot moving one distractor of the REG in a non visible compartment to be able to refer to the intended block. The planner would then have to balance between making the referring easier by moving a block before giving the instruction and giving a long and complex instruction (when feasible).

With the same scenario, using the REG extension presented in the previous chapter, allowing the RE to contain relations to past common actions, we could introduce another way for the robot to refer to a block. The planner would need to balance previous communication means with creating a unique past experience with a block to easily refer to it (\textit{e.g.} "the blue block that I just moved").

Besides, we can add multiple distinct disposal areas for blocks. The director would then have to instruct the receiver not only the block to pick but also which area they have to place it in. Furthermore, we could also add a decomposition where the robot ask to pick an under specified block, ensuring that all the matching ones need to be removed, and plan for all the possible blocks picked by the human matching this description. Depending on the block picked, it would be planned to ask to place the cube in its corresponding area. This may result in a better efficiency as it would lead to less complex referring expressions.

Finally, performing this task will necessarily bring errors, either from the human (\textit{e.g.} the wrong block may be picked) or the robot (\textit{e.g.} a block may fail to be picked). Even if some errors can be planned for (\textit{e.g.} modeling all the blocks the human can pick and planning accordingly) doing so for all the possible contingencies is not possible. Therefore, a strong link with supervision must be envisioned as it will ask for replanning. Replanning is not as simple as passing the current state and the same goal and starting the process all over again, since some task decomposition may have been executed partially and cannot be changed in the replanning process.

% Code words + Multiple sessions
% Communicating about multiple blocks at once

\section{Conclusion and Future Works}
In this chapter we proposed a new task planning approach for human robot interaction. This approach not only explicitly represent and plan upon both the human and the robot beliefs but also use two separate action models as HTNs for the human and the robot.

We proposed a formalism along with an algorithm allowing to plan for robot actions, while considering the possible human actions according to their task model. By doing so, we are able to represent and to account for the human planning and reaction processes.

Then, we presented a successful implementation of this approach in Python and showed first results along two examples. These examples highlighted both the features of the prototype planner and the rationale behind the action models crafting. The planer is able to represent and to plan for robot unknown human knowledge, human reactions to robot actions, multiple human possible plans and intricate human robot tasks. It is also capable of balancing between communicating, letting the human perform a mistake and attributing different roles to the robot or the human.

Finally, we introduced a simple to reproduce while challenging collaborative task inspired from psychology studies: the director task. To be performed by a robot, this task includes several prerequisites such as being able to take the perspective of the human partner and to refer objects in a dynamically evolving environment. The robot architecture built to handle the task allowed us to show how our prototype planner can be used on a real task in a complete robotic architecture. Some extensions of this task challenging for task planning were introduced along with hints to complete them.

We look forward to continue exploring this approach, find its benefits and its limits. More importantly we aspire at improving it, especially on the following topics.

\subsection{Representing explicitly observation processes}
A lot of code is common between the primitive tasks not only in a planning domain but also between multiple ones. Indeed, the part where we update the beliefs of the agent performing the action, but also the ones of the other agents in the room is present in almost all the primitive actions. This commonality raises the question of observability of actions. Indeed, even if we can assume that when an agent is planned to do an action they will be aware of its effects, and that if the human is planned to perform an action, even if it is not observable from the planned robot position, we update the beliefs of the robot (as it is emulating the human actions), it is not the case for a robot action and the update of human beliefs. We can represent is coarsely in the primitive task effects by relying on heuristics such as the presence of the agents in the same room. Still, observabilty of actions may need a specific representation for belief update in the planner core.

Besides, even if we plan that the human will do one action or the other at one point in the plan, recognizing and distinguishing between them may not be possible during the execution. This can endanger the interaction as the wrong branch of the plan may be executed. Thus, the supervision component must be informed, along with the plan, of the observability of the human actions, which can, in turn, impact the plan selection process.

%\subsection{Leveling-up the theory of mind}

\subsection{Pruning during the search space exploration}
For now, all the search space is explored to find valid plans, and only then the cost of actions are evaluated to select the optimal conditional plan. However, even if the search is guided by the HTNs, the branching factor can become large, especially if the planning process is executed in a robotic architecture while interacting with a human. Thus, it is possible to evaluate the plan cost during exploration and to prune some branches in the search space according to the best plan found so far. Yet, doing so would prevent from applying plan wide costs, as they could change the optimality of the plan used during in the search.

Another approach would be to learn which actions a human is more likely to perform in a certain state. We could then prune all the least probable human actions (still returned by their modeled task tree). Although this approach can be easy to implement, learning must be done per human (as two humans may react differently in the same situation) and on a few learning samples as the same world state seldom appears during specific tasks on typical short term laboratory scenarios and generalization could be difficult.

Finally, we can also imagine that the robot might ask the human which tasks they are likely to perform at a specific step in the plan, while elaborating the plan. This would lead to negotiations with the human making the final plan more acceptable while also reducing the branching factor as only the tasks answered by the human would be explored. However, for long and complex domains, this solution may confuse the human because it would require them to project on the long-term among multiple conditional eventualities.

\ifdefined\included
\else
\bibliographystyle{acm}
\bibliography{These}
\end{document}
\fi

\ifdefined\included
\else
\documentclass[a4paper,11pt,twoside]{StyleThese}
\include{formatAndDefs}
\sloppy
\begin{document}
\fi


\chapter*{Conclusion}
\addstarredchapter{Conclusion} %Sinon cela n'apparait pas dans la table des matières
In this thesis we presented multiple contributions explicitly planning for the human partner in addition to the robot. Indeed, several approaches to human-aware navigation planning (and motion planning in general) account for human presence by including social costs influencing the trajectory, but only plan a trajectory for the robot. While being successful when humans are static or when evolving in large environment, these approaches break in intricate interactions, where the robot and the human must collaborate to resolve the situation. In robotic task planning, where human robot intricate interactions are easier to explore, common approaches do include planning for both the human and the robot. However, these approaches seldom maintain different beliefs for both agents during the planning process, whereas it is what a human is expecting according to joint action theory. Moreover, these approaches consider the human as a totally controllable agent, close to what is done in multi robot planning. They do not account for communications needed to align beliefs or to share the plan.

In chapter~\ref{chapter:navigation}, we showed why planning for both the human and the robot is important for navigation planning in intricate interaction scenarios. Not only it allows to find valid solutions where other approaches would have not, but, by anticipating the possible human trajectory we can make the robot one more efficient and satisfactory for the human.  Indeed, by estimating the human future positions and speed, we are able to make the robot trajectory less threatening, more legible and to enhance the mutual manifestness of the robot. These results were proved through a user study, involving a totally autonomous PR2 robot crossing a human in a narrow corridor. We also presented how this navigation scheme has been implemented on other robots, including a HRP2 humanoid robot and a Pepper robot which ran autonomously for several weeks providing route description to customers in a mall.

Alleviating from ephemeral nature of human robot navigation interaction, we presented a new way of planning for communication during task planning in chapter~\ref{chapter:comm}. We chose to make an hybrid planning approach where a domain-independent HTN planner (HATP) delegate the resolution of feasibility and cost of communication actions to a domain-specific communication planner. We focused on communication actions needing to designate objects to the other agent. Thus, we needed a domain-specific planner able to determine the content of such communications. This problem is called referring expression generation and, albeit studied for a long time, we did not find any suitable existing work to our application. Indeed, to be used in task planning, such a planner must be efficient and some specific constraints imposed by human robot interaction have to be respected. We formalized the REG problem for HRI using an ontology as a knowledge base and we proposed an efficient algorithm solving it. This algorithm has been shown to be the most efficient to our knowledge while being designed for HRI scenarios. It then has been integrated in HATP, an HTN planner able to maintain beliefs of multiple agents during task planning. Resolving the content of communication actions at this stage is only doable if the planner plans for both human and robot, it allows to prevent execution deadlocks and improve the quality of plans.

However, HATP relies on exploring only one hierarchical task network (HTN) and allocate task to either the robot or the human depending on the task constraints to find an optimal plan. Representing interactive tasks this way leads to execution where the human is considered as knowing the plan before it begins. Indeed, several works use similar approaches and solve contingencies (\textit{e.g.} beliefs divergence, plan communication) during the plan execution. In chapter~\ref{chapter:doublehtn} we propose a new task planning scheme. The general idea is to not only keep distinct beliefs between the robot and the human but also have two separated action models. Both models are HTN, but does not represent the same concepts. The robot HTN, as in classical HTN planning, is designed to give expert knowledge to the planner on the different ways for the robot to perform a task. The human HTN on the other hand, is closer to a human task model as used in interactive systems engineering in human computer interaction. It aims at representing how a human may achieve a task (\textit{i.e.} emulating parts of their planning process) and how they may react to a particular world state or to a robot request. The presented planner uses both HTNs to elaborate valid conditional plans then selects the optimal one. These conditional plans contain the possible human actions deduced via their task model. We presented our approach in scenarios involving intricate human robot interactions and showed how it is suitable and results in interesting plans. Finally, we introduced a new task for HRI inspired from psychology experiments along with an implemented robotic architecture including our planning scheme, allowing to tackle some of the challenges of this task.

\section*{On the human agent interaction guidelines and joint action theory principles}
We presented in chapter~\ref{chapter:sota} maxims coined by Bradshaw \textit{et al.}~\cite{bradshaw2011human} for human-agent interaction. We propose here to sum up which maxims have guided the work presented in this thesis and to what extent.

By completely exploring the search space with the proposed planning method of chapter~\ref{chapter:doublehtn}, we can increase the \textit{progress appraisal} of the robot. Indeed, by returning all the conditional plan to the supervision it can compute how much the task is progressing, and communicate it to the human if needed. Besides, we proposed a plan post-processing step where robot actions are added to guide human actions away from potential errors and to the optimal plan.
Then, we showed our approach can be used to balance plans where the robot is proactive and ones where it let the human choose the tasks attributions. This matches to the agent \textit{knowing its limits} maxim.
Moreover, by representing robot unknown human knowledge, we can plan to ask question and predict possible human answers to them, making the robot more \textit{directable}. A strong link between the task planning process and the supervision is required to explore further this maxim.
Similarly, a first step has been made towards the negotiation and deconfliction of plans and beliefs alignment, increasing the robot \textit{coordination}. Some leads where presented in chapter~\ref{chapter:doublehtn} to increase it even more by interacting with the human during the plan elaboration process, in addition to the human planing process emulation. This also requires a stronger link with supervision, currently being explored.

\section*{Limitations and future work}
The task planning approach presented in chapter~\ref{chapter:doublehtn} must be refined and enriched. While it seems promising, as it allows to represent and find plans for intricate interaction scenarios, has been integrated with a domain-specific communication planner and used into a robotic architecture, limitations can be identified. First, the turn-taking approach used does not translate the duration of actions. Indeed, some actions will be longer than others, and not accounting for it may lead to suboptimal plans and wrong prediction of human actions. Then, by not pruning some part of the exploration graph during the search, the approach is not efficient for large HTN domains. Efficiency can also by gained by a better implementation in C++ rather than in Python. However, pruning while searching would prevent applying plan-wide costs.


\subsection*{Bringing planning and supervision closer}
Some future works have already been identified to extend the approaches presented in this thesis. The first one, already mentioned before, is to build a stronger link with supervision. In common robotic architectures, the link between the supervision and geometric and task planning come down to planning request and plan response. However, this link may not be enough for long term interaction or intricate and dynamic situations usually found in HRI.

For HATEB, the navigation scheme presented in chapter~\ref{chapter:navigation}, the solution proposed assume the human will respect the model provided, and adjust the robot trajectory accordingly. For example, if the robot is following the human in a narrow corridor, if the human goes slower than what is set in the navigation planner parameters, the robot will never overtake them. Indeed, the planned trajectory for the human is for them to accelerate making the robot following the human permanently. The human model provided to the planner ($\humanmodel$) must be as accurate as possible. Providing a perfect model for each human encountered is obviously impossible, that is why the supervision must not only provide an initial model to the planner with the planning request but must also update this model during the execution, especially if contingencies are detected to happen while following the plan. Here for example, a supervision system might decrease the human speed parameter if they are repeatedly detected to move slower than expected.

For the REG algorithm presented in chapter~\ref{chapter:comm} this model update is also crucial to have a good estimate of communication capabilities of the human and the associated difficulties to understand. As presented, some relations to describe an object are more difficult to understand than others. In our approach, we represented it by a cost associated to each properties. This difficulty depends on the person the robot is interacting with. For example, using color to refer to an object is less efficient or even impossible when speaking to a color blind person. Our costs are indeed defined per human we are interacting with. Besides, the difficulty to understand is also context dependent. For instance, colors relation can be hard or even impossible to perceive if the scene is lit with colored lights. Again, to cope with these issue, the supervision must update on the human model used for planning during the execution. Moreover, it can request and iterate with the planner during the planning process to allow for more or less risky communications, leading to more or less efficient plans.

The same is true for the planning approach depicted in chapter~\ref{chapter:doublehtn}. The human action model along with associated costs must be updated on a per-human basis all along the interaction. By refining the human model the best prediction would be made for them, leading to more efficient plans. Besides, some task or actions can be enabled or disabled depending on the human and their level of expertise in the task and for robot collaboration. Heuristics can also be learned as to which decomposition a human may use for a task in a specific context. Highly probable decompositions can then be explored first, resulting in a more efficient planning process.

To reduce the branching factor in human HTN exploration, we can also try to negotiate the plan while elaborating it. For example, if too many human actions are returned during the search, the planning process can request the supervision to propose the different task alternatives to the human and ask which one they would perform in the specific state the planner is in. Only the answered alternatives can then be explored by the planner. Not only it would help reducing the branching factor, but the robot may appear more \textit{predictable} and the plan more \textit{explainable} as some actions would have been chosen by the human. Thanks to the HTN structure, communication about the tasks made easier. This negotiation may need multiple iterations as the alternatives proposed by the human may lead to unfeasible plans. However, if the choice is proposed for a point too far in the future, the human may have a hard time projecting themselves in that situation.

In addition to returning the possible human actions in the conditional plan, the planner could also return the effects of the actions and more precisely the observable effects of them. By doing so, the supervision would know what to expect from the human and what to monitor to determine which action the human did, influencing the branch of the plan executed. Going further, the supervision may use the entire human action model to be able to also predict human behavior, in case of plan repair for example.

\subsection*{Theory of mind level up}
To make a better prediction of the human decisions and actions, some advanced scenarios require the robot to represent the model the human has made of it ($\robotinhumanmodel$). Indeed, as shown by Chakraborti \textit{et al.}~\cite{chakraborti2017plan}, using it can lead to more legible and predictable plans. As described in the chapter~\ref{chapter:sota}, joint action theory informs about the capabilities a human is using when interacting with another agent. Especially, humans can \textit{predict} other's actions and \textit{integrate them into their own plan}. Thus, the model the human is making about the robot will influence their decision process and how they will perform a task. This is why it is important to build this model. One approach to do so is to analyze the actions performed by the robot by taking the perspective of the human and emulate an inferring process to build a robot model.

For example, in the navigation scheme from chapter~\ref{chapter:navigation}, integrating this model would lead to a better prediction of human trajectory. The robot trajectory costs would also be more accurate, as more constraints could be added. A surprise cost for instance can be estimated by comparing the planned robot trajectory ($\robotmodel$), with the one expected by the human ($\robotinhumanmodel$). This could, in turn, be used to better respect and evaluate the respect of the maxim of \textit{predictability} and \textit{dependability}.

Likewise, in the planning approach presented in chapter~\ref{chapter:doublehtn}, using the estimation of the robot model the human has would result in more accurate predictions of human actions. Indeed, we know that the human will integrate actions of other agents  in their own plan, as showed by joint action theory. Until now, we assumed that it would not be the case as we envisioned interaction with users who have never interacted with a robot before. But, as they would gain experience, trust and habits, human partners will expect the robot to perform in a certain way. Plans quality would increase by integrating these expectations in the emulation of human planning process. Moreover, we could improve the \textit{directability} and \textit{observability} of the robot by adding possible actions explicitly updating the human's robot model.


\ifdefined\included
\else
\bibliographystyle{acm}
\bibliography{These}
\end{document}
\fi

\nomenclature{$\robotmodel$}{The global model of the robot}

\newpage
\listoftodos[Notes]


\appendix

\chapter{Navigation User Study Questionnaires}
\label{annex:questionnaires}
\section{Original PeRDITA Questionnaire Without Verbal Dimension (French)}
\sectionmark{PeRDITA (French)}
\begin{center}
\includegraphics[page=1, width=\textwidth]{Annexes/PeRDITA_vParticipant_sansDimVerbale.pdf} 
\end{center}

\begin{center}
\includegraphics[page=2, width=\textwidth]{Annexes/PeRDITA_vParticipant_sansDimVerbale.pdf} 
\end{center}

\section{Unofficial Translation of the PeRDITA Questionnaire}
\sectionmark{PeRDITA (Translated)}
\begin{center}
\includegraphics[page=1, width=\textwidth]{Annexes/perdita_translation_en_thesis.pdf} 
\end{center}

\begin{center}
\includegraphics[page=2, width=\textwidth]{Annexes/perdita_translation_en_thesis.pdf} 
\end{center}

\section{Situation Assessment Questionnaire (French)}
\begin{center}
\includegraphics[page=1, width=\textwidth]{Annexes/SAQuestionnaireFr.pdf} 
\end{center}

\begin{center}
\includegraphics[page=2, width=\textwidth]{Annexes/SAQuestionnaireFr.pdf} 
\end{center}

\section{Translation of Situation Assessment Questionnaire Items}
\begin{itemize}
\item J'ai perçu les éléments pertinents indiquant les déplacements du robot.

I perceived the relevant elements indicating the robot motion.

\item J'ai remarqué les signaux envoyés par le robot lors de son déplacement.

I noticed the signals sent by the robot during its motion.

\item J'ai compris le comportement de déplacement du robot.

I understood the robot motion behavior.

\item Les indications du robot concernant son déplacement avaient du sens pour moi.

The robot signs about its motion made sense to me.

\item J'étais en capacité d'anticiper quel déplacement le robot allait effectuer.

I was able to anticipate which motion the robot was going to do.

\item J'ai trouvé les déplacements du robot prévisibles.

It seemed to me that the robot motions were predicatables.
\end{itemize}

\section{AttrakDiff Questionnaire (French)}
This questionnaire is the official translation of the AttrakDiff questionnaire \cite{lallemand_creation_2015}.

\begin{center}
\includegraphics[page=1, width=\textwidth]{Annexes/AttrakDiff.pdf} 
\end{center}





\bibliographystyle{StyleThese}
%\bibliographystyle{plain}
\bibliography{These}

\cleardoublepage
\begin{vcenterpage}
\noindent\rule[2pt]{\textwidth}{0.5pt}

\textbf{Abstract:}
%%%%%%%%%%%%%%%%%%%%%% ATTENTION GUILHEM ! SI MODIFICATION => MODIF SUR ADUM AUSSI !!!
Robots are capable to handle autonomously more and more complex tasks. However, today's robots have either their workspaces are physically separated from the humans ones or their abilities severely restricted when evolving nearby humans. In this thesis, we propose several approaches allowing to plan not only for the robot but also for the human, enabling to predict and elicit their decision process and actions, leading to better human robot interactions (HRI). 

First, we show through a user study, why such a scheme applied to robot navigation is crucial for efficient and satisfactory interaction. Planning for the robot and the human allows to find solution in intricate situations where collaboration is necessary but also for the robot to be proactive and legible while navigating. Secondly, we alleviate from inherent ephemeral nature of interaction in collaborative navigation to explore how co-planning can be applied for task planning. We introduce a new referring expression generation algorithm, working on an ontology as knowledge base, and show that it is the fastest one to date while being designed for HRI application. We use it in a human-aware task planner to estimate feasibility and cost of communication during task planning, preventing deadlock or suboptimal plans. Finally, a novel approach to human-aware task planning is proposed where action models of the robot and the human are distinct and used to produce conditional plans.
\begin{comment}
As technology progresses, more and more complex tasks can be automated. However, robots seldom consider the nearby humans in their decision making processes. This result in today robots usage to be separated from human environment or to be overdefensive when evolving close to humans. We claim that bringing the robot and the human workspace closer would allow to use their complementarity to perform more complex tasks in collaboration, more efficiently and with more satisfaction for the human.

In this thesis we show why it is crucial for a robot not only to plan for itself but also for the nearby human in order to predict, elicit and guide human decisions and actions. 
First, we prove, through a user study, planning both for the robot and the human trajectories is important to make a successful and efficient crossing with a human in a narrow corridor. The resulting robot trajectory are not only more acceptable to the human but also allows for the robot to communicate about the proposed collaborative plan.
Secondly, we alleviate from inherent complexity and ephemeral nature of collaborative motion planning to propose a method for planning communication in human robot task planning. An novel approach for referring expression generation is described and shown to be the most computationally efficient method to date while being designed for human robot interaction scenarios. This algorithm is then integrated in a human robot task planner: the Hierarchical Agent-base Task Planner (HATP), enabling for the estimation of the feasibility and the cost of communication actions by resolving their content during task planning. 
%This require to plan for both robot and human actions while also keeping track of both agents beliefs while planning. Consequently, it prevents deadlocks and results in more efficient plan than if communication actions were resolved during execution.
Finally, we introduce a new task planning scheme reasoning on separate human and robot beliefs and action domains. By exploring two HTNs corresponding respectively to the planning domain of the robot and the task model of the human, the planner is able to predict and elicit human actions by emulating their decision and reaction processes. The pertinence of the scheme is showed on example and a more complete challenging task is presented: the director task.
\end{comment}

\textbf{Résumé :}
%%%%%%%%%%%%%%%%%%%%%% ATTENTION GUILHEM ! SI MODIFICATION => MODIF SUR ADUM AUSSI !!!
Les robots sont capables d'effectuer de manière autonome des tâches toujours plus complexes. Cependant, ils sont encore soit utilisés avec une séparation physique des humains, soit limités lorsqu'ils évoluent proches d'humains. Dans cette thèse, nous proposons plusieurs approches visant à planifier pour le robot mais aussi pour l'humain. Les plans alors générés prennent en compte l'humain en prédisant et provoquant leur actions, menant à de meilleures interactions humains-robots.

Premièrement, nous montrons au travers d'une étude utilisateur, pourquoi une telle approche appliquée à la navigation est cruciale pour une interaction efficiente et satisfaisante. Planifier pour le robot et l'humain permet en effet de trouver des solutions dans des situations d'interaction complexes mais aussi au robot d'avoir un comportement plus lisible et proactif. Deuxièment, nous présentons un algorithme de génération d'expression de référencement, le plus rapide aujourd'hui et pensé pour l'interaction humain-robot. Nous utilisons ensuite cet algorithme pour estimer la faisabilité et le coût des actions de communication pendant la planification de tâches, permettant d'éviter des impasses et des plans sous optimaux. Enfin, nous proposons une approche novatrice à la planification de tâches humain robot, dans laquelle les modèles de l'action des deux agents sont distincts et utilisés pour produire des plans conditionels.

\textbf{Keywords:}

\textbf{Mots clés :}
mots, clefs
\\
\noindent\rule[2pt]{\textwidth}{0.5pt}
\end{vcenterpage}

\end{document}
