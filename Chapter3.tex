\ifdefined\included
\else
\documentclass[a4paper,11pt,twoside]{StyleThese}
\include{formatAndDefs}
\sloppy
\begin{document}
\setcounter{chapter}{2} %% Numéro du chapitre précédent ;)
\dominitoc
\faketableofcontents
\fi

\chapter{Evaluating communication needs at task planning level}
\minitoc

\section{Introduction}


\section{Related Work}
\subsection{Referring Expression Generation}

Il faut utilizer au moins une citation (exemple: \cite{goossens93}) pour bien
compiler le document. Regardez le document These.bib pour les details de
comment organizer la bibliographie.

\subsection{Task Planning with Communication Actions}

Pour ajouter un symbole à la liste des abréviations il faut utiliser
\verb|\nomenclature{<symbole>}{<description>}|. Par exemple, je peux ajouter
$\beta$\nomenclature{$\beta$}{La deuxième lettre de l'alphabet grec} et
$\alpha$\nomenclature{$\alpha$}{La première lettre de l'alphabet grec} comme
symboles dans ce document.

\section{Ontology based Referring Expression Generation}

\section{Planning Communication Actions Using Referring Expression Generation}

\ifdefined\included
\else
\bibliographystyle{acm}
\bibliography{These}
\end{document}
\fi
